\documentclass[11pt]{report}

\title{Stan Manual: \\[8pt] {\bf Directed Graphical Model Compiler}}
\author{{\it The Stan Development Team}}
\date{\footnotesize \today}

\begin{document}

\maketitle

\begin{abstract}
  Stan's graphical model compiler converts a model specifying a joint
  probability density into an executable sampler from that density.
  The motivating use case is sampling from Bayesian posterior densities.
  
  This manual provides a getting-started guide and complete reference
  for Stan's graphical model compiler.  The compiler converts a Stan
  program specifying a directed graphical model into executable code
  through the generation of a C++ program for the model written using
  Stan's C++ modeling framework.  Stan's primary use is sampling
  parameter values from the Bayesian posterior given a data set.

  Stan generates samples using the no-U-turn sampler (NUTS), a variant
  of Hamiltonian Monte Carlo sampling (HMC) that automatically tunes
  its step size and number of leapfrog steps.  Variables are declared
  with types and automatically transformed for use with HMC.  Each
  variable in the program is declared as being data, derived data, a
  parameter, or a derived parameter.  The sampler producs sample from
  the model's distribution over the parameters.  

  


  graphical model description into C++ code, which may then be
  compiled to perform sampling.  

\end{abstract}

\chapter{Getting Started}

This section provides a quick demonstration of the end-to-end
use of Stan's graphical model compiler.

\end{document}