\part{Programming Techniques}\label{programming-techniques.part}



\chapter{Missing Data \& Partially Known Parameters}

\noindent
\BUGS and \R support mixed arrays of known and missing data.  In
\BUGS, known and unknown values may be mixed as long as every unknown
variable appears on the left-hand side of either an assignment or
sampling statement.  

\section{Missing Data}


\Stan treats variables declared in the \code{data} and
\code{transformed data} blocks as known and the variables in the
\code{parameters} block as unknown.

The next section shows how to create a mixed array of known and
unknown values as in \BUGS.  The recommended approach to missing data
in \Stan is slightly different than in \BUGS.  An example involving
missing normal observations%
%
\footnote{A more meaningful estimation example would involve a
  regression of the observed and missing observations using predictors
  that were known for each and specified in the \code{data} block.}
%
could be coded as follows.
%
\begin{quote}
\begin{Verbatim}
data {
  int(0,) N_obs;
  int(0,) N_miss;
  real y_obs[N_obs];
}
parameters {
  real mu;
  real(0,) sigma;
  real y_miss[N_miss];
}
model {
  for (n in 1:N_obs)
    y_obs[n] ~ normal(mu,sigma);
  for (n in 1:N_miss)
    y_miss[n] ~ normal(mu,sigma);
}
\end{Verbatim}
\end{quote}
%
The number of observed and missing data points are coded as data with
non-negative integer variables \code{N\_obs} and \code{N\_miss}.  The
observed data is provided as an array data variable \code{y\_obs}.
The missing data is coded as an array parameter, \code{y\_miss}.  The
ordinary parameters being estimated, the location \code{mu} and scale
\code{sigma}, are also coded as parameters.

The model contains one loop over the observed data and one over the
missing data.  This slight redundancy in specification leads to much
more efficient sampling for missing data problems in \Stan than the
more general technique described in the next section.


\section{Partially Known Parameters}\label{partially-known-parameters.section}

In some situations, such as when a multivariate probability function
has partially observed outcomes or parameters, it will be necessary to
create a vector mixing known (data) and unknown (parameter) values.
This can be done directly in \Stan by creating a vector or array in the
\code{transformed parameter} block and assigning to it.

The following example involves a bivariate covariance matrix in which the
variances are known, but the covariance is not.
%
\begin{quote}
\begin{Verbatim}
data {
  int(0,) N;
  vector(2) y[N];
  double(0,) var1;  double(0,) var2;
}
transformed data {
  double(0,) max_cov;  double(0,) min_cov;   
  max_cov <- sqrt(var1 * var2);  
  min_cov <- -max_cov;
}
parameters {
  vector(2) mu;
  real(min_cov, max_cov) cov;
}
transformed parameters {
  matrix(2,2) sigma;
  sigma[1,1] <- var1;     sigma[1,2] <- cov;
  sigma[2,1] <- cov;      sigma[2,2] <- var2;
}  
model {
 for (n in 1:N)
   y[n] ~ multi_normal(mu,sigma);
}
\end{Verbatim}
\end{quote}
%
The variances are defined as data in variables \code{var1} and
\code{var2}, whereas the covariance is defined as a parameter in
variable \code{cov}.  The $2 \times 2$ covariance matrix
\code{sigma} is defined as a transformed parameter, with the variances
assigned to the two diagonal elements and the covariance to the two
off-diagonal elements.

The constraint on the covariance declaration ensures that the
resulting covariance matrix \code{sigma} is positive definite.  The
bound, plus or minus the square root of the product of the variances,
is defined as transformed data so that it is only calculated once.

\section{Efficiency Note}

The missing-data example in the first section could be programmed with
a mixed data and parameter array following the approach of the
partially known parameter example in the second section.  The behavior
will be correct, but the computation is wasteful.  Each parameter, be
it declared in the \code{parameters} or \code{transformed parameters}
block, uses an algorithmic differentiation variable which is more
expensive in terms of memory and gradient-calculation time than a
simple data variable.  Furthermore, the copy takes up extra space and
extra time.


\chapter{Truncated or Censored Data}

\noindent
Data in which measurements have been truncated or censored can be
coded in \Stan following their respective probability models.

\section{Truncated Data}

Truncated data is data for which measurements are only reported if
they fall above a lower bound, below an upper bound, or between a
lower and upper bound.  

Truncated data may be modeled in \Stan using truncated distributions.
For example, suppose the truncated data is $y_n$ with an upper
truncation point of $U = 300$ so that $y_n < 300$.  In \Stan, this
data can be modeled as following a truncated normal distribution for
the observations as follows. 
%
\begin{quote}
\begin{Verbatim} 
data {
   int(0,) N;
   real U;
   real(,U) y[N];
} 
parameters
   real mu;
   real(0,) sigma;
} 
model {
  for (n in 1:N)
       y[n] ~ normal(mu,sigma) T(,U);
}
\end{Verbatim}
\end{quote}
% 
The model declares an upper bound \code{U} as data and constrains
the data for \code{y} to respect the constraint;  this will be checked
when the data is loaded into the model before sampling begins.

This model implicitly uses an improper flat prior on the scale and
location parameters; these could be given priors in the model.

\subsection{Unknown Truncation Points}

Nothing prevents the truncation points from being estimated as
parameters.  This can be done with a slight rearrangement of the
variable declarations from the model in the previous section with known
truncation points.
%
\begin{quote}
\begin{Verbatim}
data {
   int(1,) N;
   real y[N];
}
parameters
   real(, min(y)) L; 
   real(max(y), ) U;
   real mu;
   real(0,) sigma;
}
model {
  L ~ ...;  
  U ~ ...;
  for (n in 1:N)
       y[n] ~ normal(mu,sigma) T(L,U);
}
\end{Verbatim}
\end{quote}
%
Here there is a lower truncation point \code{L} which is declared to
be less than or equal to the minimum value of \code{y}.  The upper
truncation point \code{U} is declared to be larger than the maximum
value of \code{y}.  This declaration, although dependent on the data,
only enforces the constraint that the data fall within the
truncation bounds.  With the constraint that there is at least one
data point \code{y[n]}, the constraint that \code{L} is less than
\code{U} is enforced indirectly.

The ellipses where the priors for the bounds \code{L} and \code{U}
should go should be filled in with a an informative prior in
order for this model to not concentrate \code{L} strongly around 
\code{min(y)} and \code{U} strongly around \code{max(y)}.


\section{Censored Data}

Censoring hides values from points that are too large, too small, or
both.  Unlike with truncated data, the number of data points that were
censored is known.  The textbook example is the household scale which
does not report values above 300 pounds.  

\subsection{Estmating Censored Values}

One way to model censored data is to treat the censored data as
missing data that is constrained to fall in the censored range of
values.  Because \Stan does not allow unknown values in its arrays or
matrices, the censored values must be represented explicitly.
%
\begin{quote}
\begin{Verbatim}
data {
   int(0,) N_obs;
   int(0,) N_cens;
   real(0,) y_obs[N_obs];
   real(max(y_obs),) U;
}
parameters {
  real(U,) y[N_cens];
  real mu;
  real(0,) sigma;
}
model {
   for (n in 1:N_obs)
       y_obs[n] ~ normal(mu,sigma);
   for (y in 1:N_cens)
       y_cens[n] ~ normal(mu,sigma);
}
\end{Verbatim}
\end{quote}
%
Because the censored data array \code{y} is declared to be a parameter, it
will be sampled along with the location and scale parameters \code{mu}
and \code{sigma}.  Because the censored data array \code{y} is
declared to have values of type \code{real(U,)}, all imputed values
for censored data will be greater than \code{U}.  The imputed censored
data affects the location and scale parameters through the last
sampling statement in the model.  

\subsection{Integrating out Censored Values}

Although it is wrong to ignore the censored values in estimating
location and scale, it is not necessary to impute values.  Instead,
the values can be integrated out.  Each censored data point has a
probability of
%
\[
\mbox{Pr}[y > U] 
= \int_U^{\infty} \distro{Normal}(y|\mu,\sigma) \, dy
= 1 - \Phi\left(\frac{y - \mu}{\sigma}\right).
\]
%
With $M$ censored observations, the total probability on the log scale
is
\[
\log \prod_{m=1}^M \mbox{Pr}[y_m > U]
= \log \left( 1 - \Phi\left(\frac{y - \mu}{\sigma}\right)\right)^{M}
= M \, \log \left( 1 - \Phi\left(\frac{y - \mu}{\sigma}\right)\right)
\]

% 
This can be implemented directly in \Stan using the cumulative normal
distribution function \code{normal\_p}.  The following model assumes
that the censoring point is known, so it is declared as data.
%
\begin{quote}
\begin{Verbatim}
data {
   int(0,) N_obs;
   int(0,) N_cens;
   real(0,) y_obs[N_obs];
   real(max(y_obs),) U;
}
parameters {
  real mu;
  real(0,) sigma;
}
model {
   for (n in 1:N_obs)
       y_obs[n] ~ normal(mu,sigma); 
   lp__ <- lp__ + N_cens * log1m(normal_p(U,mu,sigma));
}
\end{Verbatim}
\end{quote}
%
For the observed values in \Verb|y_obs|, the normal sampling model is
used directly (without truncation).  The log probability accumulator
\Verb|lp__| is directly incremented using the calculated log
cumulative normal probability of the censored data items.  The
built-in function \code{log1m} is used, which maps $x$ to $\log (1 -
x)$ in an arithmetically stable way for values of $x$ near zero.

To deal with situations where the censoring point variable \code{U} is
unknown, the declaration of \code{U} should be moved from the data
block to the parameters block.



\chapter{Mixture Modeling}\label{mixture-modeling.chapter}

\noindent
Mixture models of an outcome assume that the outcome is drawn from one
of several distributions, the identity of which is controlled by a
categorical mixing distribution. Mixture models typically have
multimodal densities with modes around the modes of the mixture
components.  Mixture models may be parameterized in several ways,
as described in the following sections.


\section{Latent Discrete Parameterization}

One way to parameterize a mixture model is with a latent categorical
variable indicating which mixture component was responsible for the
outcome. For example, consider $K$ normal distributions with locations
$\mu_k \in \reals$ and scales $\sigma_k \in (0,\infty)$.  Now consider
mixing them in proportion $\theta$, where $\theta_k \geq 0$ and
$\sum_{k=1}^K \theta_k = 1$ (i.e., $\theta$ lies in the unit $K$-simplex).
For each outcome $y_n$ there is a latent variable $z_n$ in
$\setlist{1,\ldots,K}$ with a categorical distribution parameterized
by $\theta$,
%
\[
z_n \sim \distro{Categorical}(\theta).
\]
%
The variable $y_n$ is distributed according to the parameters
of the mixture component $z_n$, 
\[
y_n \sim \distro{Normal}(\mu_{z[n]},\sigma_{z[n]}).
\]
%
This model is not directly supported by \Stan because it involves
discrete parameters $z_n$, but it may be implemented as described in
the next section.

\section{Summing out the Responsibility Parameter}

To implement the normal mixture model outlined in the previous
section in \Stan, the discrete parameters can be summed out of the
model. If $Y$ is a mixture of $K$ normal distributions with 
locations $\mu_k$ and scales $\sigma_k$ with mixing proportions
$\theta$ in the unit $K$-simplex, then 
\[
p_Y(y) = \sum_{k=1}^K \theta_k \, \distro{Normal}(\mu_k,\sigma_k).
\]

For example, the mixture of $\code{Normal}(-1,2)$ and
$\code{Normal}(3,1)$ with mixing proportion $\theta =
(0.3,0.7)^{\top}$ can be implemented in \Stan as follows.
%
\begin{quote}
\begin{Verbatim}
parameters {
  real y;
}
model {
   lp__ <- log_sum_exp(log(0.3) + normal_log(y,-1,2),
                       log(0.7) + normal_log(y,3,1));
}
\end{Verbatim}
\end{quote}
%
The log probability term is derived by taking
\begin{eqnarray*}
\log p_Y(y) & = & \log \, \left( 0.3 \times \distro{Normal}(y|-1,2) \, + \,
  0.7 \times
  \distro{Normal}(y|3,1) \, \right)
\\[2pt]
& = & \log(\! \begin{array}[t]{l}
                 \exp(\log(0.3 \times \distro{Normal}(y|-1,2))) \\
                 + \exp(\log(0.7 \times \distro{Normal}(y|3,1))) \ )
              \end{array}
% \\[4pt]
% & = & \log( \! \begin{array}[t]{l}\exp(\log(0.3) + \log \distro{Normal}(y|-1,2))
%             \\
%            + \exp(\log(0.7) + \log \distro{Normal}(y|3,1)) \ )
%             \end{array}
\\[2pt]
& = & \mbox{log\_sum\_exp}(\! \begin{array}[t]{l}
                         \log(0.3) + \log \distro{Normal}(y|-1,2),
                         \\                  
                         \log(0.7) + \log \distro{Normal}(y|3,1) \ ).
                       \end{array}
\end{eqnarray*}

Given the scheme for representing mixtures, it may be moved to an
estimation setting, where the locations, scales, and mixture
components are unknown.  Further generalizing to a number of mixture
components specified as data yields the following model.
%
\begin{quote}
\begin{Verbatim}
data {
  int(1,) K;           // number of mixture components
  int(1,) N;           // number of data points
  real y[N];           // observations
}
parameters {
  simplex(K) theta;    // mixing proportions
  real mu[K];          // locations of mixture components
  real(0,) sigma[K];   // scales of mixture components
}
model {
  real ps[K];          // temp for log component densities
  for (k in 1:K) {
    mu[k] ~ normal(0,10);
    sigma[k] ~ uniform(0,10);
  }
  for (n in 1:N) {
    for (k in 1:K) {
      ps[k] <- log(theta[k]) 
               + normal_log(y[n],mu[k],sigma[k]);
    }
    lp__ <- lp__ + log_sum_exp(ps);    
  }
}
\end{Verbatim}
\end{quote}
%
The model involves \code{K} mixture components and \code{N} data
points.  The mixing proportions are defined to be a unit $K$-simplex
using \code{simplex(K)}, the components distributions locations
\code{mu[k]} are unconstrained and their scales \code{sigma[k]}
constrained to be positive.  The model declares a local variable
\code{ps} of type \code{real[K]}, which is used to accumulate the
contributions by each mixture component.

The locations and scales are drawn from simple priors for the sake of
this example, but could be anything supported by \Stan.  The mixture
components  could even be modeled hierarchically.

The main action is in the loop over data points \code{n}.  For each
such point, the log of $\theta_k \times
\distro{Normal}(y_n|\mu_k,\sigma_k)$ is calculated and added to the
array \code{ps}.  Then the log probability is incremented with the log
sum of exponentials of those values.





\chapter{Regression Models}

\noindent
\Stan supports regression models from simple linear regressions to
multilevel generalized linear models.  Coding regression models in
\Stan is very much like coding them in \BUGS.

\section{Linear Regression}

The simplest linear regression model is the following, with a single
predictor and a slope and intercept coefficient, and normally
distributed noise.  This model can be written using standard
regression notation as
%
\[
Y_n = \alpha + \beta x_n + \epsilon_n
\ \ \ \mbox{where} \ \ \ 
\epsilon_n \sim \distro{Normal}(0,\sigma).
\]
This is equivalent to the following sampling involving the
residual,
\[
Y_n - (\alpha + \beta X_n) \sim \distro{Normal}(0,\sigma),
\]
and reducing still further, to
\[
Y_n \sim \distro{Normal}(\alpha + \beta X_n, \, \sigma).
\]
%
This latter form of the model is coded in \Stan as follows.
%
\begin{quote}
\begin{Verbatim}
data {
   int(0,) N;
   real x[N];
   real y[N];
}
parameters {
    real alpha;
    real beta;
    real(0,) sigma;
}
model {
    for (n in 1:N)
        y[n] ~ normal(alpha + beta * x[n], sigma);
}
\end{Verbatim}
\end{quote}
%
There are \code{N} observations, each with predictor \code{x[n]} and
outcome \code{y[n]}.  The intercept and slope parameters are
\code{alpha} and \code{beta}.  The model assumes a normally
distributed noise term with scale \code{sigma}.  This model has
improper priors for the two regression coefficients.

\section{Coefficient and Noise Priors}

There are several ways in which this model can be generalized.  
For example, weak priors can be assigned to the coefficients as follows.
%
\begin{quote}
\begin{Verbatim}
    alpha ~ normal(0,100);
    beta ~ normal(0,100);
    sigma ~ uniform(0,100);
\end{Verbatim}
\end{quote}
%
More informative priors based the (half) Cauchy distribution are coded
as follows.
%
\begin{quote}
\begin{Verbatim}
    alpha ~ cauchy(0,2.5);
    beta ~ cauchy(0,2.5);
    sigma ~ cauchy(0,2.5);
\end{Verbatim}
\end{quote}
%
The regression coefficients \code{alpha} and \code{beta} are
unconstrained, but \code{epsilon} must be positive and properly
requires the half-Cauchy distribution, which would be expressed in
\Stan using truncation as follows.
%
\begin{quote}
\begin{Verbatim}
    sigma ~ cauchy(0,2.5) T(0,);
\end{Verbatim}
\end{quote}
%
Because the log probability function need not be normalized, the extra
factor of 2 is not needed in the model, so the simpler un-truncated
distribution suffices.

\section{Robust Noise Models}

The standard approach to linear regression is to model the noise
term $\epsilon$ as having a normal distribution.  From \Stan's
perspective, there is nothing special about normally distributed
noise.  For instance, robust regression can be accomodated by giving
the noise term a Student-$t$ distribution.  To code this in \Stan, the
sampling distribution is changed to the following.
%
\begin{quote}
\begin{Verbatim}
data {
    ...
    real(0,) nu;
}
...
model {
    for (n in 1:N)
        y[n] ~ student_t(nu,0,sigma);
}
\end{Verbatim}
\end{quote}
%
The degrees of freedom constant \code{nu} is specified as data.

\section{Logistic and Probit Regression}

For binary outcomes, logistic or probit regression can be used.  A
logistic regression with one predictor and an intercept is coded as
follows. 
%
\begin{quote}
\begin{Verbatim}
data {
    int(0,) N;
    real x[N];
    int(0,1) y[N];
}
parameters {
    real alpha;
    real beta;
}
model {
    for (n in 1:N)
        y[n] ~ bernoulli(inv_logit(alpha + beta * x[n]));
} 
\end{Verbatim}
\end{quote}
%
The noise parameter is built into the Bernoulli formulation here
rather than specified directly.

Logistic regression is a kind of generalized linear model with binary
outcomes and the log odds (logit) link function.  The inverse of the
link function appears in the model.  

Other link functions may be used in the same way.  For example, probit
regression uses the cumulative normal distribution function, which is
typically written as 
\[
\Phi(x) = \int_{-\infty}^x \distro{Normal}(y|0,1) \, dy.
\]
%
The cumulative unit normal distribution function $\Phi$ is also known
as the error function and written as \code{erf(x)}.  The probit
regression model may be coded in \Stan as follows.
%
\begin{quote}
\begin{Verbatim}
        y[n] ~ bernoulli(erf(alpha + beta * x[n]));
\end{Verbatim}
\end{quote}

\section{Multi-Logit Regression}

Multiple outcome forms of logistic regression can be coded directly in
\Stan.  For instance, suppose there are $K$ possible outcomes for each
output variable $y_n$.  Also suppose that there is a $D$-dimensional
vector $x_n$ of predictors for $y_n$.  The multi-logit model with
improper flat priors on the coefficients is coded as follows.
%
\begin{quote}
\begin{Verbatim}
data {
   int(0,) N;
   int(1,K) y[N];
   vector(D) x[N];
}
parameters {
   matrix(K,D) beta;
}
model {
  for (k in 1:K)
    for (d in 1:D)
      beta[k,d] ~ normal(0,5);
  for (n in 1:N)
     y[n] ~ categorical(softmax(beta * x[n]));
}
\end{Verbatim}
\end{quote}
%
The softmax function is defined for a $K$-vector $\gamma \in \reals^K$ by
\[
\mbox{softmax}(\gamma) = 
\left(
 \frac{\exp(\gamma_1)}
      {\sum_{k=1}^K \exp(\gamma_k)},
  \ldots,
  \frac{\exp(\gamma_K)}
       {\sum_{k=1}^K \exp(\gamma_k)}
\right).
\]
%
The result is in the unit $K$-simplex and thus appropriate to use as
the parameter for a categorical distribution.

\subsection{Identifiability}

Because softmax is invariant under adding a constant to each component
of its input, the model is typically only identified if there is a
suitable prior on the coefficients.

An alternative is to use $K-1$ vectors by fixing one of them to be
zero.  See \refsection{partially-known-parameters} for an example of
how to mix known quantities and unknown quantities in a vector.


\section{Ordered Logistic and Probit Regression}\label{ordered-logistic.section}

Ordered regression for an outcome $y_n \in \setlist{1,\ldots,K}$ with
predictors $x_n \in \reals^D$ is determined by a single coefficient
vector $\beta \in \reals^D$ along with a sequence of cutpoints $c \in
\reals^{D-1}$ sorted so that $c_d < c_{d+1}$.  The discrete output is
$k$ if the linear predictor $x_n \beta$ falls between $c_{k-1}$ and
$c_k$, assuming $c_0 = -\infty$ and $c_K = \infty$.  The noise term is
fixed by the form of regression, with examples for ordered logistic
and ordered probit models.  

\subsection{Ordered Logistic Regression}

The ordered logistic model can be coded in \Stan using the
\code{ordered} data type for the cutpoints and the built-in
\code{ordered\_logistic} distribution.
%
\begin{quote}
\begin{Verbatim} 
data {
  int(1,) D;
  int(2,) K;
  int(0,) N;
  int(1,K) y[N];
  row_vector(D) x[N];
} 
parameters {
  matrix(D) beta;
  ordered(K-1) c;
} 
model {
  for (n in 1:N)
      y[n] ~ ordered_logistic(x[n] * beta, c);
}
\end{Verbatim}
\end{quote}
% 
The vector of cutpoints \code{c} is declared as \code{ordered(K-1)},
which guarantees that \code{c[k]} is less than \code{c[k+1]}.

\subsubsection{Ordered Probit}

An ordered probit model could be coded in a manner similar to the
\BUGS encoding of an ordered logistic model.
%
\begin{quote}
\begin{Verbatim}
data {
  int(1,) D;
  int(2,) K;
  int(0,) N;
  int(1,K) y[N];
  row_vector(D) x[N];
}
parameters {
  matrix(D) beta;
  ordered(K-1) c;
}
model {
  vector(K) theta;
  for (n in 1:N) {
      real eta;
      eta <- x[n] * beta;
      theta[1] = 1 - erf(eta - c(1));
      for (k in 2:(K-1))
          theta[k] = erf(eta - c(k-1)) - erf(eta - c(k));
      theta[K] = erf(eta - c(K-1));
      y[n] ~ categorical(theta);
  }
}
\end{Verbatim}
\end{quote}
%
The logistic model could also be coded this way by replacing \code{erf}
with \code{inv\_logit}, though the built-in encoding is more efficient
and more numerically stable.  A small efficiency gain could be
achieved by computing the values \code{1 - erf(eta - c(d))} once and
storing them for re-use.

\section{Hierarchical Logistic Regression}

The simplest multilevel model is a hierarchical model in which the
data is grouped into $L$ distinct categories (or levels).  An extreme approach would be to
completely pool all the data and estimate a common vector of
regression coefficients $\beta$.  At the other extreme, an approach
would no pooling assigns each level $l$ its own coefficient vector
$\beta_l$ that is estimated separately from the other levels.  A
hierarchical model is an intermediate solution where the degree of
pooling is determined by the data and a prior on the amount of
pooling.

Suppose each binary outcome $y_n \in \setlist{0,1}$ has an associated
level, $ll_n \in \setlist{1,\ldots,L}$.  Each outcome will also have
an associated predictor vector $x_n \in \reals^D$.  Each level $l$
gets its own coefficient vector $\beta_l \in \reals^D$.  The
hierarchical structure involves drawing the coefficients $\beta_{l,d}
\in \reals$ from a prior that is also estimated with the data.  This
hierarchically estimated prior determines the amount of pooling; if
the data in each level are very similar, the pooling will be strong,
but if it is very dissimilar, the pooling will be weak.

The following model encodes a hierarchical logistic regression model
with a hierarchical prior on the regression coefficients.
%
\begin{quote}
\begin{Verbatim}
data {
    int(1,) D;
    int(0,) N;
    int(1,) L;
    int(0,1) y[N];
    int(1,L) ll[N];
}
parameters {
    real mu[D];
    real(0,) sigma[D];
    vector(D) beta[L];
}
model {
    for (d in 1:D) {
        mu[d] ~ normal(0,100);
        sigma[d] ~ uniform(0,1000);
        for (l in 1:L)
            beta[l,d] ~ normal(mu[d],sigma[d]);
    }
    for (n in 1:N)
        y[n] ~ bernoulli(inv_logit(x[n] * beta[ll[n]]));
}
\end{Verbatim}
\end{quote}   


\section{Autoregressive Models}

A first-order autoregressive model (AR(1)) with normal noise takes
each point $y_n$ in a sequence $y$ to be generated according to
%
\[
y_n \sim \distro{Normal}(\alpha + \beta y_{n-1}, \sigma).
\]
%
That is, the expected value of $y_n$ is $\alpha + \beta y_{n-1}$, with
noise scaled as $\sigma$.

\subsection{AR(1) Models}

With improper flat priors on the regression coefficients for slope
($\beta$), intercept ($\alpha$), and noise scale ($\sigma$),
the \Stan program for the AR(1) model is as follows.
%
\begin{quote}
\begin{Verbatim}
data {
    int(0,) N;
    real y[N];
}
parameters {
    real alpha;
    real beta;
    real sigma;
}
model {
    for (n in 2:N)
       y[n] ~ normal(alpha + beta * y[n-1], sigma);
}
\end{Verbatim}
\end{quote}
%
The first observed data point, \code{y[n]}, is not modeled here.  

\subsection{Extensions to the Model} 

Proper priors of a range of different families may be added for the
regression coefficients and noise scale.  The normal noise model can
be changed to a Student-$t$ distribution or any other distribution
with unbounded support.  The model could also be made hierarchical if
multiple series of observations are available.  

To enforce the estimation of a stationary AR(1) process, the slope
coefficient \code{beta} may be constrained with bounds as follows.
%
\begin{quote}
\begin{Verbatim}
    real(-1,1) beta;
\end{Verbatim}
\end{quote}
%
In practice, such a constraint is not recommended.  If the data is not
stationary, it is best to discover this while fitting the model.
Stationary parameter estimates can be encouraged with a prior favoring
values of \code{beta} near zero.


\subsection{AR(2) Models}

Extending the order of the model is also straightforward.  For
example, an AR(2) model could be coded with the second-order
coefficient \code{gamma} and the following model statement.
%
\begin{quote}
\begin{Verbatim}
for (n in 3:N)
  y[n] ~ normal(alpha + beta * y[n-1] + gamma * y[n-2], sigma);
\end{Verbatim}
\end{quote}


\subsection{AR($K$) Models}

A general model where the order is itself given as data can be coded
by putting the coefficients in an array and computing the linear
predictor in a loop.
%
\begin{quote}
\begin{Verbatim}
data {
    int(0,) K;
    ...
parameters {
    real beta[K];
    ...
model {
    for (n in (K+1):N)
        real mu;
        mu <- alpha;
        for (k in 1:K)
            mu <- mu + beta[k] * x[n-k];
        y[n] ~ normal(mu,sigma);
    }
}
\end{Verbatim}
\end{quote}





\chapter{Custom Probability  Functions}%
\label{custom-probability-functions.chapter}

\noindent
Custom distributions may also be implemented directly within \Stan's
programming language.  A simple example is the triangle distribution,
whose density is shaped like an isosceles triangle with corners at
specified bounds and height determined by the constraint that a
density integrate to 1.  If $\alpha \in \reals$ and $\beta \in \reals$
are the bounds, with $\alpha < \beta$, then $y \in (\alpha,\beta)$ has
a density defined as follows.
\[
\distro{Triangle}(y | \alpha,\beta)
= 
\frac{2}{\beta - \alpha}
\
\left(
1 - 
\left|
y - \frac{\alpha + \beta}{\beta - \alpha}
\right|
\right)
\]
%
If $\alpha = -1$, $\beta = 1$, and $y \in (-1,1)$, this reduces to
\[
\distro{Triangle}(y|-1,1) = 1 - |y|.
\]
The file \url{src/models/basic_distributions/triangle.stan} contains
the following \Stan implementation of a sampler from 
$\distro{Triangle}(-1,1)$.
%
\begin{quote}
\begin{Verbatim}
parameters {
    real(-1,1) y;
}
model {
    lp__ <- lp__ + log1m(fabs(y));
}
\end{Verbatim}
\end{quote}
%
The single scalar parameter \code{y} is declared as lying in the
interval \code{(-1,1)}.  The log probability variable \code{lp\_\_} is
incremented with the joint log probabilty of all parameters, i.e.,
$\log \distro{Triangle}(y|-1,1)$.  This value is coded in \Stan as
\code{log1m(fabs(y))}.  The function \code{log1m} is is defined so
that \code{log1m(x)} has the same value as \code{log(1.0-x)}, but the
computation is faster, more accurate, and more stable.

The log probability variable \code{lp\_\_} is incremented in this
program rather than being set.  This is because the transform involved
for the bounded variable \code{y} of type \code{real(-1,1)} implicitly
adds a term to \code{lp\_\_} to adjust the log probability for the
transform (adding the log absolute derivative of the inverse
transform).

The constrained type \code{real(-1,1)} declared for \code{y} is
critical for correct sampling behavior.  If the constraint on \code{y}
is removed from the program, say by declaring \code{y} as having the
unconstrained scalar type \code{real}, the program would compile, but
it would produce arithmetic exceptions at run time when the sampler
explored values of \code{y} outside of $(-1,1)$.

Now suppose the log probability function were extended to all of
$\reals$ as follows by defining the probability to be \code{log(0.0)},
i.e., $-\infty$, for values outside of $(-1,1)$.
%
\begin{quote}
\begin{Verbatim}
    lp__ <- log(fmax(0.0,1 - fabs(y)));
\end{Verbatim}
\end{quote}
%
With the constraint on \code{y} in place, this is just a less
efficient, slower, and less arithmetically stable version of the
original program.  But if the constraint on \code{y} is removed, 
the model will compile and run without arithmetic errors, but will not
sample properly.%
%
\footnote{The problem is the (extremely!) light tails of the triangle
  distribution.  The standard \HMC and \NUTS samplers cannot get into the
  corners of the triangle properly.  Because the actual code declares
  \code{y} to be of type \code{real(-1,1)}, the inverse logit
  transform is applied to the unconstrained variable and its log
  absolute derivative added to the log probability.  The resulting
  distribution on the logit-transformed \code{y} is well behaved.  See
  \refchapter{variable-transforms} for more information on the actual
  transforms used.}



\chapter{Optimizing \Stan Code}\label{optimization.chapter}
\noindent

\section{Exploiting Sufficient Statistics}

In some cases, models can be recoded to exploit sufficient statistics
in estimation.  This can lead to large efficiency gains compared to an
expanded model.  For example, consider the following Bernoulli
sampling model.
%
\begin{quote}
\begin{Verbatim}
data {
    int(0,) N;
    int(0,1) y[N];
    real(0,) alpha;
    real(0,) beta;
}
parameters {
    real(0,1) theta;
}
model {
    theta ~ beta(alpha,beta);
    for (n in 1:N) 
        y[n] ~ bernoulli(theta);
}
\end{Verbatim}
\end{quote}
%
In this model, the sum of positive outcomes in \code{y} is a
sufficient statistic for the chance of success \code{theta}.  The
model may be recoded using the binomial distribution as follows.
%
\begin{quote}
\begin{Verbatim}
    theta ~ beta(alpha,beta);
    sum(y) ~ binomial(N,theta);
\end{Verbatim}
\end{quote}
%
Because truth is represented as one and falsehood as zero, the sum
\code{sum(y)} of a binary vector \code{y} is equal to the number of
positive outcomes out of a total of \code{N} trials.  



\section{Exploiting Conjugacy}


Continuing the model from the previous section, the conjugacy of the
beta prior and binomial sampling distribution allow the model to be
further optimized to the following equivalent form.
%
\begin{quote}
\begin{Verbatim}
    theta ~ beta(alpha + sum(y), beta + N - sum(y));
\end{Verbatim}
\end{quote}
%
To make the model even more efficient, a transformed data variable
defined to be \code{sum(y)} could be used in the place of \code{sum(y)}.


% \section{Using Forward Sampling}

% In fact, at this point, simple (non Markov chain) Monte Carlo may be
% used because the parameters to the beta distribution are specified and
% there are no variables that depend on \code{theta}.  Thus this model
% could be even further optimized by replacing the declaration of
% \code{theta} as a parameter with a declaration as a generated quantity
% and then generating the quantity directly.
% %
% \begin{quote}
% \begin{Verbatim}
% generated quantities {
%     real(0,1) theta;
%     theta ~ random_beta(alpha + sum(y), beta + N - sum(y));
% }
% \end{Verbatim}
% \end{quote}
% %
% When used in the generated quantities block, sampling statements such
% as that for \code{theta} are executed by taking a sample from the
% specified distribution directly. The result is a Monte Carlo estimate
% of \code{theta} in which every sample is independent (up to the limits
% of the pseudorandom number generator, of course).  Thus the effective
% sample size should be estimated as being roughly equal to the sample
% size.
