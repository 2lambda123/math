
\part*{Appendices}
\addcontentsline{toc}{part}{Appendices}


\chapter{Licensing}

\noindent
\Stan and its two dependent libraries, Boost and Eigen, are
distributed under liberal licenses approved by the Open Source
Initiative.  In particular, these licenses have no ``copyleft''
provisions requiring applications of \Stan to be open source if they
are redistributed.

This chapter describes the licenses for the tools that are distributed
with \Stan.  The next chapter explains some of the build tools that
are not distributed with \Stan, but are required to build and run
\Stan models.


\section{\Stan's License}

\Stan is distributed under the BSD 3-clause license (BSD New).
%
\begin{quote}
\url{http://www.opensource.org/licenses/BSD-3-Clause}
\end{quote}

\section{Boost License}

Boost is distributed under the Boost Software
License version 1.0.
%
\begin{quote}
\url{http://www.opensource.org/licenses/BSL-1.0}
\end{quote}

\section{Eigen License} 
%
Eigen is distributed under the GNU Lesser General Public
License, version 3.0 (LGPL-3.0). 
%
\begin{quote}
\url{http://www.opensource.org/licenses/LGPL-3.0}
\end{quote}

\section{Google Test License}

\Stan uses Google Test for unit testing; it is not required to compile
or execute models.  Google Test is distributed under the BSD 2-clause
license.
%
\begin{quote}
\url{http://www.opensource.org/licenses/BSD-License}
\end{quote}

% \noindent
% \rule{\textwidth}{1pt}
% \noindent
% Copyright \copyright 2011--2012, Columbia University
% \\
% All rights reserved.
% \\[12pt]
% Redistribution and use in source and binary forms, with or without
% modification, are permitted provided that the following conditions are
% met:
% \begin{itemize}
% \item Redistributions of source code must retain the above copyright
%   notice, this list of conditions and the following disclaimer.
% \item Redistributions in binary form must reproduce the above
%   copyright notice, this list of conditions and the following
%   disclaimer in the documentation and/or other materials provided with
%   the distribution.
% \item Neither the name of Columbia University nor the names of its
%   contributors may be used to endorse or promote products derived from
%   this software without specific prior written permission.
% \end{itemize}
% {\small \noindent
%  THIS SOFTWARE IS PROVIDED BY THE COPYRIGHT HOLDERS AND
%   CONTRIBUTORS ``AS IS'' AND ANY EXPRESS OR IMPLIED WARRANTIES,
%   INCLUDING, BUT NOT LIMITED TO, THE IMPLIED WARRANTIES OF
%   MERCHANTABILITY AND FITNESS FOR A PARTICULAR PURPOSE ARE
%   DISCLAIMED. IN NO EVENT SHALL THE COPYRIGHT HOLDER OR CONTRIBUTORS
%   BE LIABLE FOR ANY DIRECT, INDIRECT, INCIDENTAL, SPECIAL, EXEMPLARY,
%   OR CONSEQUENTIAL DAMAGES (INCLUDING, BUT NOT LIMITED TO, PROCUREMENT
%   OF SUBSTITUTE GOODS OR SERVICES; LOSS OF USE, DATA, OR PROFITS; OR
%   BUSINESS INTERRUPTION) HOWEVER CAUSED AND ON ANY THEORY OF
%   LIABILITY, WHETHER IN CONTRACT, STRICT LIABILITY, OR TORT (INCLUDING
%   NEGLIGENCE OR OTHERWISE) ARISING IN ANY WAY OUT OF THE USE OF THIS
%   SOFTWARE, EVEN IF ADVISED OF THE POSSIBILITY OF SUCH DAMAGE.
% }
% \noindent
% \rule{\textwidth}{1pt}



\chapter{Installation and Compatibility}\label{install.appendix}

\noindent
This appendix describes the hardware and software required to run
\Stan.  The software includes \Stan and its libraries, as well as a
contemporary \Cpp compiler.  \Stan requires hardware powerful enough
to build and execute the models.  Ideally, that will be a 64-bit
computer with at least 4GB of memory and mulitple processor cores.

\section{Operating System}

\Stan is written in portable \Cpp without {\Cpp}11 features, as are the
libraries on which it depends.  Therefore, \Stan should run on any machine
for which a suitable \Cpp compiler is avaialble.  In practice, \Stan,
like the Boost and Eigen libraries on which it depends, is very hard
on the compiler and linker.

\Stan has been tested on the following operating systems.
%
\begin{itemize}
\item Linux (Debian, Ubuntu), 
\item Mac OS X (Snow Leopard, Lion), and
\item Windows (7).
\end{itemize}
%
\Stan should work on other versions of these operating systems if
compatible \Cpp compilers can be found.  The plan is to keep up with
new versions of these operating systems and gradually phase out
testing on older versions.


\section{Requirements}

The only two absolute requirements for running \Stan are the
\Stan source code (and dependent libraries) and a \Cpp compiler.

\subsection{\Stan Source}

In order to compile \Stan models, the \Stan source code is required.
The latest version of \Stan can be downloaded from the following link.
%
\begin{quote}
\url{http://code.google.com/p/stan/}
\end{quote}
%
The \Stan source code distribution includes \Stan's source code,
documentation, build tools, unit tests, demo models, documentation and
the source for the required libraries Eigen and Boost and the source for
an optional testing library, Google Test.

\subsubsection{Boost C++ Library Source}

\Stan's parser and some of its mathematical functions and 
template metaprogramming facilities are implemented with the Boost
\Cpp Library.  
%
\begin{itemize}
\item Home: http://www.boost.org/users/license.html
\item License: Boost Software License
\item Version: 1.49.0
\end{itemize}
%
The Boost source code is distributed with \Stan.


\subsubsection{Eigen Matrix and Linear Algebra Library Source}

\Stan's matrix algebra depends on the Eigen \Cpp matrix and linear
algebra library.  
%
\begin{itemize}
\item Home: \url{http://eigen.tuxfamily.org}
\item License: LGPL3+
\item Tested Version: 3.0.5
\end{itemize}
%
The Eigen library is distributed with \Stan.


\subsection{\Cpp Compiler}

Compiling \Stan models requires a \Cpp compiler.  \Stan has been
primarily developed with \clang and \gpp and no promises are made for
other compilers.  The full set of compilers for which \Stan has been
tested is
%
\begin{itemize}
%
\item \gpp
\\
Tested Versions: Mac 4.2, 4.6, Linux 4.4--4.7, Windows 4.6.3
\\
Home: \url{http://gcc.gnu.org/}
\\
License: GPL3+
%
\item \clang, Mac 2.9--3.0, Linux 2.9--3.0
\\
Home: \url{http://clang.llvm.org/}
\\
License: BSD
%
\item mingw-64, version 2.0 (Windows 7, cross-compiled from Debian Linux)
%
\item Intel \Cpp, Linux version 12.1.3
%
\end{itemize}
%

\section{Optional Components for Developers}

\Stan is developed using the following set of tools.  The various
command examples in this manual have assumed they can be found on
the command path.  The makefile allows precise locations to be plugged
in. 

\subsection{GNU Make Build Tool}

\Stan automates the build, test, documentation and deployment tasks
using scripts in the form of makefiles to run with GNU Make.
%
\begin{itemize}
\item Home: \url{http://www.gnu.org/software/make}
\item License: GPL3+
\item Tested Versions: 3.81 (Mac OS X), 3.79 (Windows 7)
\end{itemize}
%


\subsection{Doxygen Documentation Generator}

\Stan's API documentation is generated using the Doxygen Tool.
%
\begin{itemize}
\item Home: \url{http://www.stack.nl/~dimitri/doxygen/index.html}
\item License: GPL2
\item Tested Version(s): Mac versions 1.7.3, 1.7.4, Windows version 1.8.0
\end{itemize}


\subsection{Git Version Control System}

\Stan uses the Git version control system for its software, libraries,
and documentations.  Git is required to interact with the most recent
versions of code in the version control repository.
% 
\begin{itemize}
\item Home: \url{http://git-scm.com/}
\item License: GPL2
\item Tested Version(s): Mac version 1.7.8.4, Windows version 1.7.9
\end{itemize}


\subsubsection{Google Test C++ Testing Framework}

\Stan's unit testing is based on the Google's googletest \Cpp testing
framework.  
%
\begin{itemize}
\item
Home: \url{http://code.google.com/p/googletest/}
\item
License: BSD
\item
Tested Version(s): 1.6.0
\end{itemize}
%
The Google Test framework is distributed with \Stan.






\section{Tips for Mac OS X}

\subsection{Install Xcode}

Apple's Xcode contains both compilers and make all of the tools needed 
to work with \Stan as a user.  Which version you need depends on your
operating system version.

\begin{description}
\item[Mac OS X ``Snow Leopard'' or earlier]
Xcode 3: Good luck; we couldn't find it
\item[Mac OS X ``Lion'' or lateter]
Xcode 4: \url{https://developer.apple.com/xcode/}
\end{description}

Then, once you've installed Xcode, you need to start it, then open
menu option \code{Xcode}, select \code{Preferences}, then click on the
\code{Downloads} icon and then click on the \code{Install} button next
to the option labeled ``Command Line Tools.''

At this point, you should have the make system \code{make}, the two
\Cpp compilers/linkers \gpp and \clang installed, which
is all you need to run \Stan.  You should also have the \code{git}
version control system at this point.

\subsection{More Recent Compilers}

MacPorts hosts recent versions of compilers for the Macintosh.
%
\begin{quote}
\url{https://distfiles.macports.org/MacPorts/}
\end{quote}
%
After finding the appropriate \code{.dmg} file, clicking on it, then
double clicking on the resulting \code{.pkg} file, and clicking
through some more menus, the following will need to be entered from a
terminal window to install it.
%
\begin{quote}
\code{> sudo port install {\slshape gccVersion}}
\end{quote}
%
In this command, {\slshape gccVersion} is the name of a compiler
version, such as \code{g++=mp-4.6}, for version 4.6.  Errors may arise
during the install such as the following.
%
\begin{quote}\small\tt
  Error: Target org.macports.activate returned: Image error:
  /opt/local/include/gmp.h already exists and does not belong to a
  registered port.  Unable to activate port gmp. Use 'port -f activate
  gmp' to force the activation.
\end{quote}
%
This issue can be resolved by running the following command.
%
\begin{quote}
\code{> sudo port -f activate gmp}
\end{quote}
%


\subsection{Git Installer}

A standalone version of Git for Mac OS X is available from the
following site. 
%
\begin{quote}
\url{http://code.google.com/p/git-osx-installer/}
\end{quote}
%
Although (at the time of this writing) there were only versions listed
up to OS X version ``Snow Leopard,'' they work on ``Lion''.


\section{Tips for Windows}

\subsection{Install Rtools}

The easiest way to get a complete \Cpp build environment on Windows is
to install the most recent version of Rtools.  

The latest version verified to work with \Stan is Rtools 2.15.  Rtools
2.15 includes the \gpp 4.6.3 (pre-release) compiler and many other
useful command line tools including many Unix commands, such as the
following.
%
\begin{quote}
\tt basename, cat, cmp, comm, cp, cut, date,
diff, du, echo, expr, gzip, ls, make, makeinfo, mkdir, mv, rm, rsync,
sed, sh, sort, tar, texindex, touch, uniq
\end{quote}

Rtools can be downloaded from the following location.
%
\begin{quote}
  \url{http://cran.r-project.org/bin/windows/Rtools/}
\end{quote}
%
Install it using the Windows installer.  Allow it to edit the
\code{PATH} environment variable so that commands are available from
the command tool.

To verify the installation was successful, open a command window by
selecting the following menu items.
%
\begin{quote}
  \code{Start} 
  $\rightarrow$ Accessories 
  $\rightarrow$ Command Prompt
\end{quote}
%
To verify that \gpp is installed, use the following command.
%
\begin{quote}
  \Verb|> g++ -v|
\end{quote}
%
This should report version information for \gpp.  Next, verify that
\code{make} is installed with the following command.
%
\begin{quote}
  \Verb|> make -v|
\end{quote}
%
This should print version information for make.

\subsection{Install Git}

There are a number of Git clients for Windows that will work.  The
official Git installer for Windows can be found at the following
location.
%
\begin{quote}
\url{http://code.google.com/p/msysgit/downloads}
\end{quote}
%
Select the latest full installer and install it. 


% \chapter{User-Defined Functions and Gradients}\label{user-defined-functions.appendix}




