
\part*{Appendices}
\addcontentsline{toc}{part}{Appendices}


\chapter{Licensing}

\Stan is licensed under the 3-clause BSD License.  This license has
been approved by the Open Source Initiative.  For more information,
see
%
\begin{quote}
\url{http://opensource.org/licenses/BSD-3-Clause}
\end{quote}
%
\Stan's 3-clause BSD license is the following.
\vspace*{6pt}

\noindent
\rule{\textwidth}{1pt}

\noindent
Copyright \copyright 2011--2012, Columbia University
\\
All rights reserved.
\\[12pt]
Redistribution and use in source and binary forms, with or without
modification, are permitted provided that the following conditions are
met:

\begin{itemize}
\item Redistributions of source code must retain the above copyright
  notice, this list of conditions and the following disclaimer.
\item Redistributions in binary form must reproduce the above
  copyright notice, this list of conditions and the following
  disclaimer in the documentation and/or other materials provided with
  the distribution.
\item Neither the name of Columbia University nor the names of its
  contributors may be used to endorse or promote products derived from
  this software without specific prior written permission.
\end{itemize}

{\small \noindent
 THIS SOFTWARE IS PROVIDED BY THE COPYRIGHT HOLDERS AND
  CONTRIBUTORS ``AS IS'' AND ANY EXPRESS OR IMPLIED WARRANTIES,
  INCLUDING, BUT NOT LIMITED TO, THE IMPLIED WARRANTIES OF
  MERCHANTABILITY AND FITNESS FOR A PARTICULAR PURPOSE ARE
  DISCLAIMED. IN NO EVENT SHALL THE COPYRIGHT HOLDER OR CONTRIBUTORS
  BE LIABLE FOR ANY DIRECT, INDIRECT, INCIDENTAL, SPECIAL, EXEMPLARY,
  OR CONSEQUENTIAL DAMAGES (INCLUDING, BUT NOT LIMITED TO, PROCUREMENT
  OF SUBSTITUTE GOODS OR SERVICES; LOSS OF USE, DATA, OR PROFITS; OR
  BUSINESS INTERRUPTION) HOWEVER CAUSED AND ON ANY THEORY OF
  LIABILITY, WHETHER IN CONTRACT, STRICT LIABILITY, OR TORT (INCLUDING
  NEGLIGENCE OR OTHERWISE) ARISING IN ANY WAY OUT OF THE USE OF THIS
  SOFTWARE, EVEN IF ADVISED OF THE POSSIBILITY OF SUCH DAMAGE.
}
\noindent
\rule{\textwidth}{1pt}


\chapter{Installation and Compatibility}\label{install.appendix}

Running \Stan requires the \Stan source code along with fairly
complete \Cpp environment.  This appendix describes the hardware and
software required to run \Stan.  

\Stan requires a computer powerful enough to compile and run \Cpp
code.  A 64-bit computer with multiple cores is preferable.  64-bit
machines process double-precision floating-point arithmetic faster and
multiple cores allow several Markov chains to be simulated in parallel.

\section{Operating System}

\Stan is written in portable \Cpp without {\Cpp}11 features, as are the
libraries on which it depends.  Therefore, \Stan should run on any machine
for which a suitable \Cpp compiler is avaialble.  In practice, \Stan,
like the Boost and Eigen libraries on which it depends, is very hard
on the compiler and linker.

\Stan is tested on and maintained for
%
\begin{itemize}
\item Linux (Debian, Ubuntu), 
\item Mac OS X (Snow Leopard, Lion), and
\item Windows (7).
\end{itemize}
%

\section{Software}

The only two absolute requirements for running \Stan are the
\Stan source code (and dependent libraries) and a \Cpp compiler.

\subsection{\Stan Source Code}

In order to compile \Stan models, the \Stan source code is required.
The latest version of \Stan can be downloaded from the following link.
%
\begin{quote}
\url{http://code.google.com/p/stan/}
\end{quote}
%
The \Stan source code distribution includes \Stan's source code,
documentation, build tools, unit tests, demo models, documentation and
the source for the libraries Eigen and Boost.

\subsection{googletest C++ Testing Framework}

\Stan's unit testing is based on the Google's googletest \Cpp testing
framework.  
%
\begin{itemize}
\item
Home: \url{http://code.google.com/p/googletest/}
\item
License: BSD
\item
Tested Version(s): 1.6.0
\end{itemize}
%
The googletest framework is distributed with \Stan.

\subsection{Eigen Matirx and Linear Algebra Library}

\Stan's matrix algebra depends on the Eigen \Cpp matrix and linear
algebra library.  
%
\begin{itemize}
\item Home: \url{http://eigen.tuxfamily.org}
\item License: LGPL3+
\item Tested Version: 3.0.5
\end{itemize}
%
The Eigen library is distributed with \Stan.

\subsection{Boost C++ Library}

\Stan's parser and some of its mathematical functions and 
template metaprogramming facilities are implemented with the Boost
\Cpp Library.  
%
\begin{itemize}
\item Home: http://www.boost.org/users/license.html
\item License: Boost Software License
\item Version: 1.49.0
\end{itemize}
%
The Boost source code is distributed with \Stan.

\subsection{C++ Compiler}

Compiling \Stan models requires a \Cpp compiler.  \Stan has been
primarily developed with \clang and \gpp.  The full set of compilers
for which \Stan has been tested is
%
\begin{itemize}
%
\item \gpp, Mac versions 4.2, 4.6, Linux versions 4.4--4.7
\\
Home: \url{http://gcc.gnu.org/}
\\
License: GPL3+
%
\item \clang, Mac and Linux versions 2.9--3.0
\\
Home: \url{http://clang.llvm.org/}
\\
License: BSD
%
\item mingw-64, version 2.0 (Windows 7, cross-compiled from Debian Linux)
%
\item Intel \Cpp, Linux version 12.1.3
%
\end{itemize}
%

\section{Optional Components for Developers}

\Stan is developed using the following set of tools.  For convenience,
the executables may be made accessible from the \code{PATH}
environment variable.  If they are not directly accessible, their
executables must be referenced with a fully qualified path.


\subsection{GNU Make Build Tool}

\Stan automates the build, test, documentation and deployment tasks
using scripts in the form of makefiles.  These can be run using the
GNU make tool, which may be downloaded from the following
location.
\begin{itemize}
\item Home: \url{http://www.gnu.org/software/make/}
\item License: GPL3+
\item Tested Versions: 3.81 (Mac OS X)
\end{itemize}
%


\subsection{Doxygen Documentation Generator}

\Stan's API documentation is generated using the Doxygen Tool.
%
\begin{itemize}
\item Home: \url{http://www.stack.nl/~dimitri/doxygen/index.html}
\item License: GPL2
\item Tested Version(s): 1.7.3
\end{itemize}


\subsection{Git Version Control System}

\Stan uses the Git version control system for its software, libraries,
and documentations.  
% 
\begin{itemize}
\item Home: \url{http://git-scm.com/}
\item License: GPL2
\item Tested Version(s): 1.7.8.4
\end{itemize}



\section{Tips for Mac OS X}

\subsection{Install Xcode}

Apple's Xcode contains both compilers and make all of the tools needed 
to work with \Stan as a user.  Which version you need depends on your
operating system version.

\begin{description}
\item[Mac OS X ``Snow Leopard'' or earlier]
Xcode 3: Good luck; we couldn't find it
\item[Mac OS X ``Lion'' or lateter]
Xcode 4: \url{https://developer.apple.com/xcode/}
\end{description}

Then, once you've installed Xcode, you need to start it, then open
menu option \code{Xcode}, select \code{Preferences}, then click on the
\code{Downloads} icon and then click on the \code{Install} button next
to the option labeled ``Command Line Tools.''

At this point, you should have the make system \code{make}, the two
\Cpp compilers/linkers \gpp and \clang installed, which
is all you need to run \Stan.  You should also have the \code{git}
version control system at this point.

\subsection{Use Macports}

\begin{Verbatim}
How to install MacPorts g++

1.  From the download page:

https://distfiles.macports.org/MacPorts/

2.  Snow Leopard:

    MacPorts-2.0.3-10.6-SnowLeopard.dmg

3.  Click on .dmg file

4.  Double click on the .pkg file.

5.  Click through menus (you'll need to enter your password)

6.  Pull up a NEW terminal and run from anywhere:

    > sudo port install gcc46

    You'll need to enter your password again

    NOTE:  On my notebook, I got a lot of errors like:

               Error: Target org.macports.activate returned: Image
	       error: /opt/local/include/gmp.h already exists and does
	       not belong to a registered port.  Unable to activate
	       port gmp. Use 'port -f activate gmp' to force the
	       activation.

           and had to do what it suggested for several packages:

           >  sudo port -f activate gmp

7.  Installed as:

    > g++-mp-4.6

8.  Run Stan with

    > make CC=g++-mp-4.6

    (or make a link)
\end{Verbatim}


\subsection{Git Installer}

If you want to download a standalone version of Git, use

\begin{quote}
\url{http://code.google.com/p/git-osx-installer/}
\end{quote}

Although (at the time of this writing) there were only versions listed
up to OS X version ``Snow Leopard,'' they work on ``Lion''.

\chapter{User-Defined Functions and Gradients}\label{user-defined-functions.appendix}


\chapter{BNF Grammars}

This appendix provides a modular Backus-Naur form (\BNF) grammar for
the \Stan modeling language.  Literal strings are indicated in single
quotes (\code{'}).  Grammar non-terminals are unquoted strings.  A
prefix question mark (\code{?}) indicates optionality.  A postfixed
Kleene star (\code{*}) indicates zero or more occurrences.
Parentheses are used for grouping.

\section{Programs}

\Stan's grammar for programs is
%
\begin{quote}
\begin{Verbatim}
program ::= ?data ?tdata ?params ?tparams model ?generated

data ::= 'data' var_decls
tdata ::= 'transformed data' var_decls_statements
params ::= 'parameters' var_decls
tparams ::= 'transformed parameters' var_decls_statements
model ::= 'model' statement
generated ::= 'generated quantities' var_decls_statements

var_decls ::= '{' var_decl* '}'
var_decls_statements ::= '{' var_decl* statement* '}'
\end{Verbatim}
\end{quote}

\section{Variable Declarations}

\Stan's grammar for variable declarations is
%
\begin{quote}
\begin{Verbatim}
var_decl ::= var_type var_name ?dims

var_type ::= int ?int_range
           | real ?real_range
           | vector '(' int_expression ')'
           | pos_ordered '(' int_expression ')'
           | simplex '(' int_expression ')'
           | row_vector '(' int_expression ')'
           | matrix '(' int_expression ',' int_expression ')'
           | corr_matrix '(' int_expression ')'
           | cov_matrix '(' int_expression ')'

int_range ::= '(' int_expression ',' ?int_expression ')'
            | '('  ','  int_expression ')'

real_range ::= '(' real_expression ',' ?real_expression ')'
             | '('  ','  real_expression ')'


dims ::= '['  int_expression (',' int_expression)*  ']'
\end{Verbatim}
\end{quote}

\section{Expressions}

\Stan's grammar for expressions is 
%
\begin{quote}
\begin{Verbatim}
expression ::= literal
             | variable
             | expression infixOp expression
             | prefixOp expression
             | expression postfixOp
             | expression '[' expressions ']'
             | function '(' ?expressions ')'
             | '(' expression ')'

expressions ::= expression
              | expression ',' expressions
\end{Verbatim}
\end{quote}

\section{Statements}

\Stan's grammar for statements is
%
\begin{quote}
\begin{Verbatim}
statement 
::= lhs '<-' expression ';'
  | expression '~' identifier '(' ?expressions ')' ?truncation ';'
  | 'for' '(' identifier 'in' expression ':' expression ')' statement
  | '{' var_decl* statement+ '}'
  | ';'

truncation ::= 'T' '(' ?expression ',' ?expression ')' 

lhs ::= identifier
      | identifier '[' expressions ']'

\end{Verbatim}
\end{quote}


