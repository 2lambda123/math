
\part{Introduction}

\chapter{What is \Stan?}

This document is a reference manual and getting started guide for
using \Stan's probabilistic modeling language.  After describing the
overall system in this introduction and providing a hands-on
quick-start guide in the following chapter, the remainder of the
document is devoted to fully documenting the behavior of
Stan's modeling language.

\Stan's modeling language and its execution behavior are similar to
that of its progenitors, \BUGS and \JAGS.  It differs in many
particulars of the modeling language, which is more like an
impertative programming language than the declarative specifications
of \BUGS and \JAGS.  For instance, \Stan statements are executed in
the order they are written in model specifications and local variables
may be reassigned as in a procedural programming language.
Furthermore, variables must be declared before they are used.


\section{\Stan's Modeling Language}

\noindent
\Stan's modeling language allows users to code a Bayesian model
specifying a joint probability function
\[
p(y,\theta),
\]
where 
\begin{itemize}
\item
  $y$ is a vector of known values, such as 
  constants, hyperparameters, and modeled data, and
\item
 $\theta$ is a vector of unknown values, such as estimated parameters,
  missing data, and simulated values.
\end{itemize}
%
To simplify terminology, $y$ will be called the data vector and
$\theta$ the parameter vector.  The probability function $p(y,\theta)$
need only be specified up to a multiplicative constant with respect to
any fixed data vector $y$.  This ensures proportionality of the
posterior to the specified joint probability,
\[
p(\theta|y) \propto p(y,\theta) = p(\theta|y) \, p(y).
\]

Stan's language is more imperative than its declarative 
predecessors, \BUGS and \JAGS.  Statements are executed in the order 
they are specified and variables and expressions are strongly typed 
and declared as data or parameters in the model rather than by a 
calling function.  \Stan also supports a broader range of arithmetic, 
matrix, and linear algebra operations than \BUGS or \JAGS.  Users may 
manipulate log probability functions directly and are not required 
to use proper priors.

\section{\Stan's Compiler}

\Stan's compiler, \stanc, reads a user program in \Stan's modeling
language and generates a \Cpp class implementing the model specified
by that program.  \Stan automatically applies a multivariate transform
(and its Jacobian determinant) to free any constrained parameters,
such as deviations (constrained to be positive), unit simplexes (a vector
constrained to be positive and to sum to 1), and covariance matrices
(positive definiteness).  The result is an unconstrained sampling (or
optimization) problem from the perspective of the sampler.  From the
user's perspective, this transform happens behind the scenes, driven
by the types declared for each of the parameters.

The generated \Cpp class can be plugged into \Stan's continuous and
discrete samplers to read the data $y$ and then draw a sequence of
sample parameter vectors $\theta^{(m)}$ according to the posterior,
\[
p(\theta|y) = \frac{p(y,\theta)}{p(y)} \propto p(y,\theta).
\]
The resulting samples may be used for full Bayesian inference, much of
which can be carried out within \Stan.

\section{\Stan's Samplers}

For continuous variables, \Stan uses Hamiltonian Monte Carlo (\HMC)
sampling. \HMC is a Markov chain Monte Carlo (\MCMC) method based on
simulating the Hamiltonian dynamics of a fictional physical system in
which the parameter vector $\theta$ represents the position of a
particle in $K$-dimensional space and potential energy is defined to
be the negative (unnormalized) log probability.  Each sample in the
Markov chain is generated by starting at the last sample, applying a
random momentum to determine initial kinetic energy, then simulating
the path of the particle in the field.  Standard \HMC runs the
simulation for a fixed number of discrete steps of a fixed step size
and uses a Metropolis adjustment to ensure detailed balance of the
resulting Markovian system.  This adjustment treats the momentum term
of the Hamiltonian as an auxiliary variable, and the only reason for
rejecting a sample will be discretization error in computing the
Hamiltonian.

\HMC treats the position of a particle, 
log probability as a negative potential
energy function, then samples by adding random kinetic energy and
simulating the 

In addition to basic \HMC, \Stan implements an adaptive
version of \HMC, the No-U-Turn Sampler (\NUTS).  \NUTS automatically
tunes step sizes and a diagonal mass matrix during warmup and then
adapts the number of leapfrog integration steps during sampling.
Stan is expressive enough to allow most discrete variables to be
marginalized out.  For the remaining discrete parameters, \Stan uses
Gibbs sampling if there are only a few outcomes and adaptive slice
sampling otherwise.


\chapter{Getting Started}

This chapter is designed to help users get acquainted with the overall
design of the \Stan language and calling \Stan from the command line.
For installation information, see \refappendix{install}.
Later chapters are devoted to expanding on the material in this
chapter with full reference documentation.


\section{A Minimal Program}

Stan is distributed with several working models.  The simplest of
these is found in the following location relative to the top-level
distribution.
%
\begin{quote}
\begin{Verbatim}
src/models/basic_distributions/normal.stan
\end{Verbatim}
\end{quote}
%
The contents of this file are as follows.
%
\begin{quote}
\begin{Verbatim}
parameters {
  real y;
}
model {
  y ~ normal(0,1);
}
\end{Verbatim}
\end{quote}
%
The model's single parameter \code{y} is declared to take real values.
The probability model specifies that \code{y} has a normal
distribution with location 0 and scale 1.  Basically, this model will
sample a single unit normal variate.  

\section{Whitespace and Semicolons}

In \Stan, every variable declaration and atomic statement must be
terminated by a semicolon (\code{;}).  This is the convention followed
by programming languages such as \Cpp.  It is not the convention
followed by the statistical languages \R, \BUGS, or \JAGS.

The reason for the \Cpp convention is to ensure that differences in
whitespace are not meaningful.  In \R, \BUGS, and \JAGS, the following
is a complete, legal statement.
%
\begin{quote}
\begin{Verbatim}
a <- b +
     c
\end{Verbatim}
\end{quote}
%
In contrast, the usual way of typesetting mathematics and laying out
code in programming languages, with the operator continuing the
expression beginning a new line, is invalid.
%
\begin{quote}
\begin{Verbatim}
a <- b
     + c
\end{Verbatim}
\end{quote}
%
The only difference is in the kind of whitespace between \code{b} and
\code{+} and between \code{+} and \code{c}.  In \Stan, there is no
whitespace-dependent behavior.  Neither of these is a complete
statement, whereas either one terminated with a semicolon is.  The
second form is recommended for \Cpp and \Stan.


\section{Compiling  with {\tt\bfseries stanc}}

Starting at \Stan's home directory, written here as {\tt \$stan},
the model may be compiled by the \Stan compiler, \stanc, into \Cpp code
as follows.
%
\begin{quote}
\begin{Verbatim}[fontshape=sl]
> cd $stan
> stanc src/models/basic_distributions/normal.stan
\end{Verbatim}
%
\begin{Verbatim}
Model name=anon_model
Input file=src/models/basic_distributions/normal.stan
Output file=anon_model.cpp
\end{Verbatim}
\end{quote}
%
The output indicates the name of the model, here the default value
\code{anon\_model}, the input file from which the \Stan program is
read, here \code{normal.stan}, and the output file to which the
generated \Cpp code is written, here \code{anon\_model.cpp}.  See
\refchapter{stanc} for more documentation on the \stanc compiler.

\section{Compiling the Generated Code}

The file generated by \stanc must next be compiled with a \Cpp
compiler by linking to \Stan's source and library directories using
the {\tt -I} option of the compiler.  The following example 
uses the \clang compiler for \Cpp.
%
\begin{quote}
\begin{Verbatim}[fontshape=sl]
> clang++ -I src -I lib anon_model.cpp 
\end{Verbatim}
\end{quote}
%
This command invokes the \clang compiler for \Cpp to create a
platform-specific executable in the default location, which is {\tt
  a.out} by convention.  If all goes well, as above, there is no
output to the console.  More information about compiling the \Cpp code
generated by \Stan may be found in \refchapter{compiling-cpp}.
Installation information for \Cpp compilers may be found in
\refappendix{install}.

\section{Running the Sampler}

The executable resulting from compiling the generated \Cpp may be run
as follows.
%
\begin{quote}
\begin{Verbatim}[fontshape=sl]
> ./a.out
\end{Verbatim}
%
\begin{Verbatim}
STAN SAMPLING COMMAND
data = 
init = random initialization
samples = samples.csv
append_samples = 0
seed = 1331941513 (randomly generated)
chain_id=1 (default)
iter = 2000
warmup = 1000
thin = 1
leapfrog_steps = -1
max_treedepth = 10
epsilon = -1
epsilon_pm = 0
epsilon_adapt_off = 0
delta = 0.5
gamma = 0.05

Iteration: 2000 / 2000 [100%]  (Sampling)
\end{Verbatim}
\end{quote}
%
The program indicates to the standard output that the samples are
written to \code{samples.csv}.  The first few lines of this file
are comments about aspects of the run.
%
\begin{quote}
\begin{Verbatim}[fontshape=sl]
> cat samples.csv
\end{Verbatim}
\begin{Verbatim}
# Samples Generated by Stan
#
# stan_version_major=alpha
# stan_version_minor=0
# data=
# init=random initialization
# append_samples=0
# seed=1331941796
# chain_id=1
# iter=2000
# warmup=1000
# thin=1
# leapfrog_steps=-1
# max_treedepth=10
# epsilon=-1
# epsilon_pm=0
# delta=0.5
# gamma=0.05
...
\end{Verbatim}
\end{quote}
%
The ellipses notation, {\tt ...}, indicates that the output continues
beyond what's shown.  Here, what follows is the data in standard
comma-separate value ({\sc csv}) notation.
%
\begin{quote}
\begin{Verbatim}
...
lp__,treedepth__,y
-0.0126699,1,0.159185
-0.222796,1,-0.667527
-0.222796,1,-0.667527
-0.404457,1,-0.899397
...
\end{Verbatim}
\end{quote}
%
The first line consists of a header indicating the names of the
variables on the lines to follow, and each following line indicates a
single sampled value of the parameters.  The first column is reserved
for the (unnormalized) log probability (density) of the parameters,
with name {\tt lp\_\_} (the underscores are to prevent name conflicts
with user-defined model parameters).  The next values are for
reporting the behavior of the sampler.  In this case, the \NUTS
sampler was used, so there is a report of the depth of tree it
explored, with variable name {\tt treedepth\_\_}.  The remaining
values are parameters.  Here, the model has only one parameter, {\tt
  y}.  The first sampled value for {\tt y} is 0.159185, the second is
-0.667527, and so on.  

Note that the second sampled value is repeated.  This is not a bug.
Rather, it is the behavior to expect from a sampler using a Metropolis
acceptance step for proposals, as \Stan's samplers \HMC and \NUTS do.

\section{Data}

\Stan allows data to be specified in programs, used in models, and
read into compiled \Stan programs. This section provides an example of
coding and running a \Stan program with data stored in a file in the
\SPLUS/\R dump format.

The \Stan program in 
\begin{quote}
\begin{Verbatim}
src/models/basic_estimators/bernoulli.stan
\end{Verbatim}
\end{quote}
can be used to estimate a Bernoulli parameter \code{theta} from
\code{N} binary observations.  The file contains the following code.
%
\begin{quote}
\begin{Verbatim}
data {
  int(0,) N;
  int(0,1) y[N];
}
parameters {
  real(0,1) theta;
}
model {
  theta ~ beta(1,1);
  for (n in 1:N)
    y[n] ~ bernoulli(theta);
}
\end{Verbatim}
\end{quote}
%
This program declares two data variables in its \code{data} block.
The first data variable, \code{N}, is an integer encoding the number
of observations.  The declaration \code{int(0,)} indicates that
\code{N} must take on non-negative values.  The second data variable,
\code{y}, is declared as \code{y[N]}, specifying that it is an array
of \code{N} values.  Each of these values has the declared type,
\code{int(0,1)}, an integer between 0 and 1 inclusive, i.e., a binary
value.  The \code{N} individual binary values in the array \code{y}
are accessed using standard array notation, indexing from 1, as \code{y[1]},
\code{y[2]}, ..., \code{y[N]}.

The \code{parametes} block declares a single parameter, \code{theta}.
Its type is given as \code{real(0,1)}, meaning it takes on continuous
values between 0 and 1 inclusive.  The constraint is necessary in
order to ensure that \code{theta} takes on a legal value as the
success parameter in the Bernoulli distribution in which it is used in
the \code{model} block of the program.

The \code{model} block consists of a for-loop for the data.   The loop is
specified so that the body is executed for values of \code{n} between
\code{1} and \code{N} inclusive.  The body here is a sampling
statement specifying that the variable \code{y[n]} is modeled as
having a Bernoulli distribution with parameter \code{theta}.  

A sample data file for this program can be found in the file
\code{bernoulli.Rdata} in the same directory.  This data file has
the following contents.
%
\begin{quote}
\begin{Verbatim}
N <- 10
y <- c(0,1,0,0,0,0,0,0,0,1)
\end{Verbatim}
\end{quote}
%
A data file must contain appropriate values for all of the data
variables declared in the \Stan program's \code{data} block.  Here there
is a non-negative integer value for \code{N} and an array of length
\code{N} (i.e., 10) integer values between 0 and 1 inclusive.  The
array is coded using the \SPLUS sequence notation \code{c(...)}.
The dump format supported by \Stan is documented in \refchapter{dump}.

The program is compiled by \stanc and the \Cpp compiler in the same
way.  This time, the output model gets an explicitly specified name.
%
%
\begin{quote}
\begin{Verbatim}[fontshape=sl]
> stanc --name=bern src/models/basic_estimators/bernoulli.stan 
\end{Verbatim}
\begin{Verbatim}
Model name=bern
Input file=src/models/basic_estimators/bernoulli.stan
Output file=bern.cpp
\end{Verbatim}
\end{quote}
%
As before, the \Cpp compiler needs to be given the name of
generated file.
%
\begin{quote}
\begin{Verbatim}[fontshape=sl]
> clang++ -O3 -I src -I lib -o bern bern.cpp
\end{Verbatim}
\end{quote}
%
There are two new compiler options here.  The option \code{-O3} sets
optimization to level 3, which generates much faster executable
code at the expense of slower compilation.  The name of the
executable is also specified, using the option \code{-o~bern}.  Now
the code may be executed by calling its executable with the data file
specified. 
%
\begin{quote}
\begin{Verbatim}[fontshape=sl]
> ./bern --data=src/models/basic_estimators/bernoulli.Rdata
\end{Verbatim}
\end{quote}

\section{Proper and Improper Priors}

The model in the previous section does not contain a sampling
statement for \code{theta}.  The default behavior is to give
\code{theta} a uniform prior.  In this case, a uniform prior is proper
because \code{theta} is bounded to a finite interval.  Improper priors
are also allowed in \Stan programs; they arise from unconstrained
parameters without sampling statements.  The uniform prior could have
also been added explicitly by adding the following statement to the
\code{model} block of the program.
%
\begin{quote}
\begin{Verbatim} 
theta ~ uniform(0,1);
\end{Verbatim}
\end{quote}
% 
A third way to specify that \code{theta} has a uniform distribution
between 0 and 1 is with the beta distribution.
%
\begin{quote}
\begin{Verbatim}
theta ~ beta(1,1);
\end{Verbatim}
\end{quote}
%
The beta distribution is conjugate to the Bernoulli, but \Stan (at
least as of yet) does not make use of this information.  On the other hand,
these three approaches, no prior, uniform prior, and beta prior,
are equally efficient in \Stan's sampler, because their uniformity
can be determined at compile time and thus computations related to
them eliminated.  There is further discussion of \Stan optimization
in \refchapter{optimization}

\section{Comments}

\Stan supports the three major style of comments.  

If the character \code{\#} appears on a line, that character and every
character up to but not including the end of line is ignored.  This is
the style of comments used in Python, R, and the shell.  

If the character pair \code{//} appears on a line, that pair and every
character up to but not including the end of line is ignored.  This is
the style of comments used in C.  This is the preferred commenting
style for line-based comments and for commenting out code.

\Stan also supports \Cpp comment style in which any content placed
between \code{/*} and \code{*/} is ignored along with the boundary
markers.  This is the preferred style for long documentation comments.
It is difficult to use this method to comment out code, because 
an internal close comment sequence, \code{*/}, may take precedence.


\section{User-Defined Distributions and Functions}

\Stan allows new distributions to be coded directly in its modeling
language, as described in \refsection{custom-prob-functions}. 

Extending \Stan's underlying \Cpp API provides a way to add new
functions that may be implemented efficiently and used across
different \Stan programs.

\refappendix{user-defined-functions} documents the way in which new
basic functions may be added to \Stan.  The new functions may use
algorithmic differentiation to compute the partial derivatives of
their output with respect to the input variables, or the user may
directly specify the partials for increased efficiency.

