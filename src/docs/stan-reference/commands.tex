\part{Commands and Data Formats}


\chapter{Compiling \Stan Programs}

\noindent
Preparing a \Stan program to be run involves two steps,
%
\begin{enumerate}
\item translating the \Stan program to \Cpp, and
\item compiling the resulting \Cpp to an executable.
\end{enumerate}
%
This chapter discusses both steps, as well as their encapsulation into
a single make target.  

\section{Installing Stan}

Before Stan can be run, it must be installed; see
\refappendix{install} for complete platform-specific installation
details.

\section{Translating and Compiling through {\tt\bfseries make}}\label{make-models.section}

The simplest way to compile a Stan program is through the \code{make}
build tool, which encapsulates the translation and compilation step
into a single command.  The commands making up the \code{make} target
for compiling a model are described in the following sections, and the
following chapter describes how to run a compiled model.

\subsection{Translating and Compiling Test Models}

There are a number of test models distributed with Stan which unpack
into the path \code{src/models}.  To build the simple example
\nolinkurl{src/models/basic\_estimators/bernoulli.stan}, the following call
to \code{make} suffices.  First the directory is changed to Stan's home
directory by replacing \code{<stan-home>} with the appropriate path.
%
\begin{quote}
\begin{Verbatim}[fontshape=sl,fontsize=\small]
> cd <stan-home>
\end{Verbatim}
\end{quote}
%
The current directory should now contain the file named
\code{makefile}, which is the default instructions used by
\code{make}.  From within the top-level Stan directory, the following call
will build an executable form of the Bernoulli estimator.
%
\begin{quote}
\begin{Verbatim}[fontshape=sl,fontsize=\small]
> make src/models/basic_estimators/bernoulli
\end{Verbatim}
\end{quote}
%
This will translate the model \code{bernoulli.stan} to a \Cpp file and
compile that \Cpp file, putting the executable in
\nolinkurl{src/models/basic_distributions/bernoulli(.exe)}.  Although
the make command including arguments is itself portable, the target it
creates is different under Windows than in Unix-like platforms.  Under
Linux and the Mac, the executable will be called \code{bernoulli},
whereas under Windows it will be called \code{bernoulli.exe}.

\subsection{Dependencies in {\tt\bfseries make}}

A \code{make} target can depend on other \code{make} targets.  When
executing a \code{make} target, first all of the targets on which it
depends are checked to see if they are up to date, and if they are
not, they are rebuilt.  This includes the top-level target itself.  If
the \code{make} target to build the Bernoulli estimator is invoked a
second time, it will see that it is up to date, and not compile
anything.  But if one of the underlying files has changes since the
last invocation \code{make}, such as the model specification file, it
will be retranslated to \Cpp and recompiled to an executable.

There is a dependency included in the make target that will
automatically build the \code{bin/stanc} compiler and the
\code{bin/libstan.a} library whenever building a model.


%\subsection{Making Other Models}
%
%A \code{make} target can be invoked from a directory other than the
%one that contains the \code{makefile} with the \code{-f} option.  Thus
%an alternative way to compile this model from its own source directory
%would be as follows (again replacing \code{<stan-home>} with the
%appropriate path to the top-level directory in which Stan was
%unpacked).
%\begin{quote}
%\begin{Verbatim}[fontshape=sl,fontsize=\small]
%> cd <stan-home>src/models/basic_estimators
%> make -f ../../../makefile bernoulli
%\end{Verbatim}
%\end{quote}
%%
%The expression \code{../../../makefile} is a relative path to the
%file named \code{makefile}; this says to go up three directories
%(i.e., back to \code{<stan-home>}) and then look for a file named
%\code{makefile}.  This can be an absolute path, too.  The Unix-like
%forward slashes can be used under Windows as well as under Linux and
%Mac.  


\subsection{Getting Help from the {\tt makefile}}

Stan's \code{makefile}, which contains the top-level instructions to
\code{make}, provides extensive help in terms of targets and options.
It is located at the top-level of the distribution, so first change
directories to that location.
%
\begin{quote}
\begin{Verbatim}[fontshape=sl,fontsize=\small]
> cd <stan-home>
\end{Verbatim}
\end{quote} 
and then invoke make with the target \code{help},

\begin{quote}
\begin{Verbatim}[fontshape=sl,fontsize=\small]
> make help
\end{Verbatim}
\end{quote}


\subsection{Options to \code{make}}

Stan's \code{make} targets allow the user to change compilers,
library versions for Eigen and Boost, as well as compilation options
such as optimization.

These options should be placed right after the call to \code{make}
itself.  For instance, to specify the \code{clang++} compiler at
optimization level 0, use
%
\begin{quote}
\begin{Verbatim}[fontshape=sl,fontsize=\small]
> make CC=clang++ O=0 ...
\end{Verbatim}
\end{quote}


\subsubsection{Compiler Option}

The option \code{CC=g++} specifies the \code{g++} compiler and
\code{CC=clang++} specifies the \code{clang++} compiler.  Other
compilers with other names may be specified the same way.  A full path
may be used, or just the name of the program if it can be found on the
system execution path.  

\subsubsection{Optimization Option}

The option \code{O=0} (that's letter `O', equal sign, digit `0'),
specifies optimization level 0 (no optimization), whereas \code{O=3}
specifies optimization level 3 (effectively full optimization), with
levels 1 and 2 in between.

With higher optimization levels, generated executable tends to be bigger
(in terms of bytes in memory) and faster.  For best results on
computationally-intensive models, use optimization level 3 for the
Stan library and for compiling models.

\subsubsection{Library Options}

Alternative versions of Eigen, Boost, and Google Test may be specified
using the properties \code{EIGEN}, \code{BOOST}, and \code{GTEST}.
Just set them equal to a path that resolves to an appropriate library.
See the libraries distributed under \code{lib} to see which
subdirectory of the library distribution should be specified in order
for the include paths in the \Cpp code to resolve properly.


\subsection{Additional  \code{make} Targets}

All of these targets are intended to be invoked from the top-level
directory in which Stan was unpacked (i.e., the directory that
contains the file named \code{makefile}).

\subsubsection{Clean Targets}

A very useful target is \code{clean-all}, invoked as
%
\begin{quote}
\begin{Verbatim}[fontshape=sl,fontsize=\small]
> make clean-all
\end{Verbatim}
\end{quote}
%
This removes everything that's created automatically by \code{make},
including the \code{stanc} translator, the Stan libraries, and all the
automatically generated documentation.  

\subsubsection{Make Target for \code{stanc}}

To make the \code{stanc} compiler, use
%
\begin{quote}
\begin{Verbatim}[fontshape=sl,fontsize=\small]
> make bin/stanc
\end{Verbatim}
\end{quote}
%
As with other executables, the executable \code{bin/stanc} will be
created under Linux and Mac, whereas \code{bin/stanc.exe} will be
created under Windows.

\subsubsection{Make Target for Stan Library}

To build the Stan library, use the following target,
%
\begin{quote}
\begin{Verbatim}[fontshape=sl,fontsize=\small]
> make bin/libstan.a
\end{Verbatim}
\end{quote}



\section{Translating \Stan to C++ with {\tt\bfseries stanc}}\label{stanc.section}


\subsection{Building the \stanc Compiler and the Stan Library}

Before the \stanc compiler can be used, it must be built.  Use the
following command from the top-level distribution directory containing
the file named \code{makefile}.
%
\begin{quote}
\begin{Verbatim}[fontshape=sl,fontsize=\small]
> make bin/stanc
\end{Verbatim}
\end{quote}
%
This invocation produces the executable \code{bin/stanc} under Linux
and Mac, and \code{bin/stanc.exe} under Windows.  The invocation of
\code{make}, including the forward slash, is the same on both platforms.

The default compiler option is \code{CC=g++} and the default
optimization level is \code{O=3} (the letter `O');  to see how to
change these, see the previous section in this chapter on \code{make}.


\subsection{The \stanc Compiler}

The \stanc compiler converts \Stan programs to \Cpp programs.  The
first stage of compilation involves parsing the text of the \Stan
program.  If the parser is successful, the second stage of compilation
generates \Cpp code.  If the parser fails, it will provide a
diagnostic error message indicating the location in the input where
the failure occurred and reason for the failure.

The following example illustrates a fully qualified call to \stanc
to build the simple Bernoulli model; just replace \code{<stan-home>}
with the top-level directory containing Stan (i.e., the directory
containing the file named \code{makefile}). 

For Linux and Mac:
%
\begin{quote}
\begin{Verbatim}[fontshape=sl,fontsize=\small]
> cd <stan-home>
> bin/stanc --name=bernoulli --o=bernoulli.cpp \
  src/models/basic_estimators/bernoulli.stan 
\end{Verbatim}
\end{quote}
%
The backslash (\Verb|\|) indicates a continuation of the same line.

For Windows:
%
\begin{quote}
\begin{Verbatim}[fontshape=sl,fontsize=\small]
> cd <stan-home>
> bin\stanc --name=bernoulli --o=bernoulli.cpp ^
>   src/models/basic_estimators/bernoulli.stan 
\end{Verbatim}
\end{quote}
%
The caret (\Verb|^|) indicates continuation on Windows.

This call specifies the name of the model, here {\tt bernoulli}.
This will determine the name of the class implementing the model in
the \Cpp code.  Because this name is the name of a \Cpp class, it must
start with an alphabetic character (\code{a--z} or \code{A--Z}) and
contain only alphanumeric characters (\code{a--z}, \code{A--Z}, and
\code{0--9}) and underscores (\code{\_}) and should not conflict with
any \Cpp reserved keyword.  

The \Cpp code implementing the class is written to the file
\code{bernoulli.cpp} in the current directory.  The final argument,
\code{bernoulli.stan}, is the file from which to read the \Stan
program.

\subsection{Command-Line Options for {\tt\bfseries stanc}}

The model translation program \code{stanc} is called as follows.
%
\begin{quote}
\code{> stanc [options] {\slshape model\_file}}
\end{quote}
%
The argument \code{\slshape model\_file} is a path to a Stan model file ending
in suffix \code{.stan}.  The options are as follows.
%
\begin{description}
%
\item[\tt {-}-help] 
\mbox{ } \\ 
Displays the manual page for \stanc.  If this option is selected,
nothing else is done.
%
\item[\tt {-}-version]
\mbox{ } \\ 
Prints the version of \stanc.  This is useful for bug reporting
and asking for help on the mailing lists.
%
\item[\tt {-}-name={\slshape class\_name}]
\mbox{ } \\ 
Specify the name of the class used for the implementation of the
\Stan model in the generated \Cpp code.  
\\[2pt]
Default: {\tt {\slshape class\_name = model\_file}\_model}
%
\item[\tt {-}-o={\slshape cpp\_file\_name}]
\mbox{ } \\ 
Specify the name of the file into which the generated \Cpp is written.
\\[2pt]
Default: {\tt {\slshape cpp\_file\_name} = {\slshape class\_name}.cpp}
%
\item[\tt {-}-no\_main]
\mbox{ } \\
Include this flag to prevent the generation of a main function in the
output.
\\[2pt]
Default: generate a main function
\end{description}



\section{Compiling C++ Programs}\label{compiling-cpp.section}

\noindent
As shown in the previous section (\refsection{stanc}), \Stan converts
a program in the \Stan modeling language to a \Cpp program.  This \Cpp
program must then be compiled using a \Cpp compiler.  

The \Cpp compilation step described in this chapter, the model
translation step described in the last chapter, and the compilation of
the dependent binaries \code{bin/stanc} and \code{bin/libstan.a} may
be automated through make; see \refsection{make-models} for details.

\subsection{Which Compiler?}

\Stan has been developed using two portable, open-source \Cpp
compilers, \gpp and \clang, both of which run under and generate code
for Windows, Macintosh, and Unix/Linux.%
%
\footnote{As of the current version, \Stan cannot be compiled using
  \MSVC, the Windows-specific compiler from Microsoft.  \MSVC is able
  to compile the \code{stanc} compiler, but not the templates required
  for algorithmic differentiation and the Eigen matrix library.}

The \clang compiler is almost twice as fast at low levels of
optimization, but the machine code generated by \gpp at high
optimization levels is faster.


\subsection{What the Compiler Does}

A \Cpp compiler like \gpp or \clang performs several lower-level
operations in sequence,
% 
\begin{enumerate}
\item
parsing the input \Cpp source file(s), 
\item 
generating (static or dynamically) relocatable object code, and
\item 
linking the relocatable object code into executable code.
\end{enumerate}
%
These stages may be called separately, though the examples in this
manual perform them in a single call.  The compiler invokes the
assembler to convert assembly language code to machine code, and the
linker to resolve the location of references in the relocatable object
files.

\subsection{Compiler Optimization}

\Stan was written with an optimizing compiler in mind, which allows
the code to be kept relatively clean and modular.  As a result, \Stan
code runs as much as an order of magnitude or more faster with
optimization turned on.

For development of \Cpp code for \Stan, use optimization level 0; for
sampling, use optimization level 3.  These are controlled through
\Stan's makefile using \code{O=0} and directly through \clang or \gpp
with \code{-O0}; in both cases, the first character is the letter `O'
and the second the digit `0'.


\subsection{Building the \Stan Library}

Before compiling a \Stan-generated \Cpp program, the \Stan object
library archive must be built using the makefile.  This only needs to
be done once and then the archive may be reused.  The recommended
build command for the \Stan archive is as follows (replacing
\code{<stan-home>} with the directory into which Stan was unpacked and
which contains the file named \code{makefile}).
%
\begin{quote}
\begin{Verbatim}[fontshape=sl,fontsize=\small]
> cd <stan-home>
> make CC=g++ O=3 bin/libstan.a 
\end{Verbatim}
\end{quote}
%
Please be patient and ignore the (unused function) warning messages.
Compilation with high optimization on \code{g++} takes time (as much
as 10 minutes or more) and memory (as much as 3GB). 

This example uses the \code{g++} compiler for \Cpp (makefile option
\code{CC=g++}).  The \clang compiler may be used by specifying
\code{CC=clang++}.

This example uses compiler optimization level 3 (makefile option
\code{O=3}).  Turning the optimization level down to 0 allows the code
to built in under a minute in less than 1GB of memory.  This will slow
down sampling as much as an order of magnitude or more, so it is not
recommended for running models.  It can be useful for working on
\Stan's \Cpp code.


\subsection{Compiling a \Stan Model}

Suppose following the instructions in the last chapter
(\refsection{stanc}) that a \Stan program has been converted to a \Cpp
program that resides in the source file \code{<stan-home>/my\_model.cpp}.

The following commands will produce an executable in the file
\code{my\_model} in the current working directory (\code{<stan-home>}).
%
\begin{quote}
\begin{Verbatim}[fontshape=sl,fontsize=\small]
> cd <stan-home>
> g++ -O3 -Lbin -Isrc -Ilib/boost_1.53.0 \
  -Ilib/eigen_3.1.2 my_model.cpp -o my_model -lstan
\end{Verbatim}
\end{quote} %
The backslash (\Verb|\|) is used to indicate that the command is
continued; it should be entered all one one line.
%
The options used here are as follows.
\begin{quote}
\begin{description}
\item[\code{-O3}] sets optimization level 3,
\item[\code{-Lbin}] specifies that the archive is in the \code{bin}
  directory,
\item[\code{-Isrc}] specifies that the directory \code{src} should be
  searched for code (it contains the top-level \Stan headers),
\item[\code{-Ilib/eigen\_3.1.2}] specifies the directory for
  the Eigen library,
\item[\code{-Ilib/boost\_1.53.0}] specifies the directory for the 
  Boost library,
\item[\code{my\_model.cpp}] specifies the name of the source file to
  compile, and 
\item[\code{-o my\_model}] is the name of the resulting executable
  produced by the command (suffixed by \code{.exe} in Windows).
\item[\code{-lstan}] specifies the name of the archived library (not
  the name of the file in which it resides),
\end{description}
\end{quote}
%
The library binary and source specifications are required, as is the
name of the \Cpp file to compile.  User-supplied directories may be
included in header or archive form by specifying additional \code{-L},
\code{-l}, and \code{-I} options.
 
A lower optimization level may be specified.  If there is no
executable name specified using the \code{-o} option, then the model
is written into a file named \code{a.out}.
 

\subsection{Library Dependencies}

\Stan depends on two open-source libraries,
%
\begin{enumerate}
\item the Boost general purpose \Cpp libraries, and 
\item the Eigen matrix and linear algebra \Cpp libraries.
\end{enumerate}
%
These are both distributed along with \Stan in the directory
\code{<stan-home>/lib}.  

The code for \Stan itself is located in the directory
\code{<stan-home>/src}.  Because not all of \Stan is included in the
archive \code{bin/libstan.a}, the \code{src} directory must also be
included for compilation.





\chapter{Running a \Stan Program}\label{stan-cmd.chapter}

\noindent 
Once a \Stan program defining a model has been converted to a \Cpp
program for that model (see \refsection{stanc}) and the resulting \Cpp
program compiled to a platform-specific executable (see
\refsection{compiling-cpp}), the model is ready to be run.

\section{Simple Example}

Suppose the executable is in file \code{my\_model} and the data
is in file \code{my\_data}, both in the current working directory, and
you want to draw samples from the model.
Then the \Stan executable may be run on Windows using
%
\begin{quote}
\begin{Verbatim}[fontshape=sl,fontsize=\small]
> my_model sample data=my_data
\end{Verbatim}
\end{quote}
%
This will read the data from file \code{my\_data}, run warmup tuning for
1000 iterations, which are discarded, then run the fully-adaptive
\NUTS sampler for 1000 iterations, writing the parameter (and other)
values to the file \code{samples.csv} in the current working
directory.  A random number generation seed will be derived from
the system time automatically. Note that on Mac or Linux, it is 
necessary to instead execute
%
\begin{quote}
\begin{Verbatim}[fontshape=sl,fontsize=\small]
> ./my_model sample data=my_data
\end{Verbatim}
\end{quote}
%
\section{Parallel Example}

The previous example executes one chain, which can be repeated to 
generate multiple chains. However, users may want to execute chains
in parallel on a multicore machine. To do so with four chains using
a Bash shell on Mac or Linux execute
%
\begin{quote}
\begin{Verbatim}[fontshape=sl,fontsize=\small]
> for i in {1..4}
> do
>   ./my_model sample random seed=12345 \ 
      id=$i data=my_data \
>     output file=samples$i.csv refresh=0 &
> done
>
\end{Verbatim}
\end{quote}
%
Note that there is blank line at the end that returns control to the
prompt. The \& at the end pushes each process into the background, so
that the loop can continue without waiting for the previous chain to 
finish. This example requires several command-line options that are
explained in further detail in the next subsection. 

On Windows, the following is functionally equivalent to the Bash
snippet above
%
\begin{quote}
\begin{Verbatim}[fontshape=sl,fontsize=\small]
> for /l %x in (1, 1, 4) do start /b model  sample ^
< random seed=12345 id=%x data=my_data ^
>  output file=samples%x.csv refresh=0
\end{Verbatim}
\end{quote}
%
If the grep and sed programs are installed, then the following will
combine the four comma-separated values files into a single
comma-separated values file.
%
\begin{quote}
\begin{Verbatim}[fontshape=sl,fontsize=\small]
> grep lp__ samples1.csv > combined.csv
> sed '/^[#l]/d' samples*.csv >> combined.csv 
\end{Verbatim}
\end{quote}
%
\section{Command-Line Options}\label{stan-command-line-options.section}

\Stan executables are highly configurable, allowing the user to specify
and customize not only the calculation method but also the data, output,
initialization, and random number generation.  The arguments are defined
hierarchically so that, for example, optimization settings are not necessary
when sampling.  

The atomic element of the hierarchy are \textit{categorical arguments}
which define self-contained categories of arguments.  Categories and
arguments within each category are unordered so that, for example,
%
\begin{quote}
\begin{Verbatim}[fontshape=sl,fontsize=\small]
> ./model sample output file=samples.csv diagnostic_file=diagnostics.csv random seed=1
> ./model sample output diagnostic_file=diagnostics.csv file=samples.csv random seed=1
> ./model sample random seed=1 output file=samples.csv diagnostic_file=diagnostics.csv 
> ./model sample random seed=1 output diagnostic_file=diagnostics.csv file=samples.csv 
\end{Verbatim}
\end{quote}
%
all define the same configuration.

Categorical arguments can appear in isolation, such as \code{random} and
\code{output}, or as values to other arguments (Figure \ref{fig:hierarchy}).  In the latter case the arguments
are mutually exclusive and specifying conflicting arguments causes the execution
to immediately terminate (Figure \ref{fig:configuration}).

\begin{figure}
\setlength{\unitlength}{0.01in} 
\centering
\begin{picture}(500, 480)
%
\put(10, 460) { \makebox(50, 10)[l]{id, data, init} }
%
\put(0, 410) { \framebox(500, 45) }
\put(10, 440) { \makebox(50, 10)[l]{random} }
\put(50, 420) { \makebox(50, 10)[l]{seed} }
%
\put(0, 360) { \framebox(500, 45) }
\put(10, 390) { \makebox(65, 10)[l]{output} }
\put(50, 370) { \makebox(500, 10)[l]{file, append\_sample, diagnostic\_file, append\_diagnostic, refresh, save\_warmup} }
%
\put(10, 340) { \makebox(50, 10)[l]{method} }
\put(20, 117.5) { \line(0, 1){212.5} }
\put(20, 117.5) { \vector(1, 0){15} }
\put(20, 257.5) { \vector(1, 0){15} }
\put(20, 307.5) { \vector(1, 0){15} }
%
\put(40, 285) { \framebox(460, 45) }
\put(50, 315) { \makebox(65, 10)[l]{ diagnose } }
\put(90, 295) { \makebox(50, 10)[l]{ $\ldots$ } }
%
\put(40, 235) { \framebox(460, 45) }
\put(50, 265) { \makebox(63, 10)[l]{ optimize } }
\put(90, 245) { \makebox(50, 10)[l]{ $\ldots$ } }
%
\put(40, 5) { \framebox(460, 225) }
\put(50, 215) { \makebox(55, 10)[l]{sample} }
\put(90, 195) { \makebox(350, 10)[l]{num\_samples, num\_warmup, save\_warmup, thin} }
%
\put(80, 140) { \framebox(410, 45) }
\put(90, 170) { \makebox(55, 10)[l]{adapt} }
\put(130, 150) { \makebox(50, 10)[l]{$\ldots$} }
%
\put(90, 120) { \makebox(55, 10)[l]{algorithm} }
\put(100, 37.5) { \line(0, 1){70} }
\put(100, 87.5) { \vector(1, 0){15} }
\put(100, 37.5) { \vector(1, 0){15} }
%
\put(120, 65) { \framebox(370, 45) }
\put(130, 95) { \makebox(55, 10)[l]{hmc} }
\put(170, 75) { \makebox(50, 10)[l]{$\ldots$} }
%
\put(120, 15) { \framebox(370, 45) }
\put(130, 45) { \makebox(115, 10)[l]{ rw\_metropolis } }
\put(170, 25) { \makebox(50, 10)[l]{ $\ldots$ } }
\end{picture}
\caption{In the hierarchical argument structure, certain arguments, such as \code{random} and \code{output}, introduce new categories of arguments.  These categorical arguments can also appear as values of other arguments, such as \code{diagnose}, \code{optimize}, and \code{sample}, which define the mutually exclusive values for the argument {method}. }
\label{fig:hierarchy}
\end{figure}

\begin{figure}
\setlength{\unitlength}{0.01in} 
\centering
\begin{picture}(500, 480)
%
\put(10, 460) { \makebox(50, 10)[l]{id, data, init} }
%
\put(0, 410) { \framebox(500, 45) }
\put(10, 440) { \makebox(50, 10)[l]{random} }
\put(50, 420) { \makebox(50, 10)[l]{seed} }
%
\put(0, 360) { \framebox(500, 45) }
\put(10, 390) { \makebox(65, 10)[l]{output} }
\put(50, 370) { \makebox(500, 10)[l]{file, append\_sample, diagnostic\_file, append\_diagnostic, refresh, save\_warmup} }
%
\put(10, 340) { \makebox(50, 10)[l]{method} }
\put(20, 117.5) { \line(0, 1){212.5} }
\put(20, 117.5) { \vector(1, 0){15} }
\put(20, 257.5) { \color{gray!30}\vector(1, 0){15} }
\put(20, 307.5) { \color{gray!30}\vector(1, 0){15} }
%
\put(40, 285) { \color{gray!30}\framebox(460, 45) }
\put(50, 315) { \makebox(65, 10)[l]{ \textcolor{gray!30}{diagnose} } }
\put(90, 295) { \makebox(50, 10)[l]{ \textcolor{gray!30}{$\ldots$} } }
%
\put(40, 235) { \color{gray!30}\framebox(460, 45) }
\put(50, 265) { \makebox(63, 10)[l]{ \textcolor{gray!30}{optimize} } }
\put(90, 245) { \makebox(50, 10)[l]{ \textcolor{gray!30}{$\ldots$} } }
%
\put(40, 5) { \framebox(460, 225) }
\put(50, 215) { \makebox(55, 10)[l]{sample} }
\put(90, 195) { \makebox(350, 10)[l]{num\_samples, num\_warmup, save\_warmup, thin} }
%
\put(80, 140) { \framebox(410, 45) }
\put(90, 170) { \makebox(55, 10)[l]{adapt} }
\put(130, 150) { \makebox(50, 10)[l]{$\ldots$} }
%
\put(90, 120) { \makebox(55, 10)[l]{algorithm} }
\put(100, 37.5) { \color{gray!30}\line(0, 1){70} }
\put(100, 87.5) { \line(0, 1){20} }
\put(100, 87.5) { \vector(1, 0){15} }
\put(100, 37.5) { \color{gray!30}\vector(1, 0){15} }
%
\put(120, 65) { \framebox(370, 45) }
\put(130, 95) { \makebox(55, 10)[l]{hmc} }
\put(170, 75) { \makebox(50, 10)[l]{$\ldots$} }
%
\put(120, 15) { \color{gray!30}\framebox(370, 45) }
\put(130, 45) { \makebox(115, 10)[l]{ \textcolor{gray!30}{rw\_metropolis} } }
\put(170, 25) { \makebox(50, 10)[l]{ \textcolor{gray!30}{$\ldots$} } }
\end{picture}
\caption{A valid argument configuration defines only one mutually exclusive argument.
If conflicting arguments are specified, for example \code{method=optimize method=sample},
then execution immediately terminates with a warning message.}
\label{fig:configuration}
\end{figure}

All commands must include at least one method, specified explicitly as
\code{method=method\_name} or implicitly with only \code{method\_name}.
All remaining configurations are optional with defaults values provided
for all arguments.

Additionally, there are various ways of displaying help output instead
of execution.  Specifying \code{help} after any argument displays a 
description and valid options for that argument, while \code{help-all}
provides a verbose presentation of that argument and all arguments
derived from it in the hierarchy.  If \code{help} is specified without any
additional arguments a usage message is displayed, while \code{help-all}
displays the entire argument hierarchy.

The full hierarchy is as follows; some typical use-case examples 
are provided in the next section.

\pagebreak

\begin{description}
\hiercmdarg{}{method}{$$list element$$}
  {Analysis method (Note that method= is optional)}
  {Valid values: sample, optimize, diagnose}
  {Defaults to sample}
%
  \hierlongcmd{\indentarrow}{sample}
    {Bayesian inference with Markov Chain Monte Carlo}
    {Valid subarguments: num\_samples, num\_warmup, save\_warmup, thin, adapt, algorithm}
%
    \hiercmdarg{\indentarrow\indentarrow}{num\_samples}{$$int$$}
      {Number of sampling iterations}
      {Valid values: 0 $\le$ num\_samples}
      {Defaults to 1000}
%
    \hiercmdarg{\indentarrow\indentarrow}{num\_warmup}{$$int$$}
      {Number of warmup iterations}
      {Valid values: 0 $\le$ warmup}
      {Defaults to 1000}
%
    \hiercmdarg{\indentarrow\indentarrow}{save\_warmup}{$$boolean$$}
      {Stream warmup samples to output?}
      {Valid values: [0, 1]}
      {Defaults to 0}
%
    \hiercmdarg{\indentarrow\indentarrow}{thin}{$$int$$}
      {Period between saved samples}
      {Valid values: 0 $<$ thin}
      {Defaults to 1}
%
    \hierlongcmd{\indentarrow\indentarrow}{adapt}
      {Warmup Adaptation}
      {Valid subarguments: engaged, gamma, delta, kappa, t0}
%
      \hiercmdarg{\indentarrow\indentarrow\indentarrow}{engaged}{$$boolean$$}
        {Adaptation engaged?}
        {Valid values: [0, 1]}
        {Defaults to 1}
%
      \hiercmdarg{\indentarrow\indentarrow\indentarrow}{gamma}{$$double$$}
        {Adaptation regularization scale}
        {Valid values: 0 $<$ gamma}
        {Defaults to 0.05}
%
      \hiercmdarg{\indentarrow\indentarrow\indentarrow}{delta}{$$double$$}
        {Adaptation target acceptance statistic}
        {Valid values: 0 $<$ delta $<$ 1}
        {Defaults to 0.65}
%
      \hiercmdarg{\indentarrow\indentarrow\indentarrow}{kappa}{$$double$$}
        {Adaptation relaxation exponent}
        {Valid values: 0 $<$ kappa}
        {Defaults to 0.75}
%
      \hiercmdarg{\indentarrow\indentarrow\indentarrow}{t0}{$$double$$}
        {Adaptation iteration offset}
        {Valid values: 0 $<$ t0}
        {Defaults to 10}
%
    \hiercmdarg{\indentarrow\indentarrow}{algorithm}{$$list element$$}
      {Sampling algorithm}
      {Valid values: hmc, rwm}
      {Defaults to hmc}
%
      \hierlongcmd{\indentarrow\indentarrow\indentarrow}{hmc}
        {Hamiltonian Monte Carlo}
        {Valid subarguments: engine, metric, stepsize, stepsize\_jitter}
%
        \hiercmdarg{\indentarrow\indentarrow\indentarrow\indentarrow}{engine}{$$list element$$}
          {Engine for Hamiltonian Monte Carlo}
          {Valid values: static, nuts}
          {Defaults to nuts}
%
          \hierlongcmd{\indentarrow\indentarrow\indentarrow\indentarrow\indentarrow}{static}
            {Static integration time}
            {Valid subarguments: int\_time}
%
            \hiercmdarg{\indentarrow\indentarrow\indentarrow\indentarrow\indentarrow\indentarrow}{int\_time}{$$double$$}
              {Total integration time for Hamiltonian evolution}
              {Valid values: 0 $<$ int\_time}
              {Defaults to 2 * pi}
%
          \hierlongcmd{\indentarrow\indentarrow\indentarrow\indentarrow\indentarrow}{nuts}
            {The No-U-Turn Sampler}
            {Valid subarguments: max\_depth}
%
            \hiercmdarg{\indentarrow\indentarrow\indentarrow\indentarrow\indentarrow\indentarrow}{max\_depth}{$$int$$}
              {Maximum tree depth}
              {Valid values: 0 $<$ max\_depth}
              {Defaults to 10}
%
        \hiercmdarg{\indentarrow\indentarrow\indentarrow\indentarrow}{metric}{$$list element$$}
          {Geometry of base manifold}
          {Valid values: unit\_e, diag\_e, dense\_e}
          {Defaults to diag\_e}
%
          \hiershortcmd{\indentarrow\indentarrow\indentarrow\indentarrow\indentarrow}{unit\_e}
            {Euclidean manifold with unit metric}
%
          \hiershortcmd{\indentarrow\indentarrow\indentarrow\indentarrow\indentarrow}{diag\_e}
            {Euclidean manifold with diag metric}
%
          \hiershortcmd{\indentarrow\indentarrow\indentarrow\indentarrow\indentarrow}{dense\_e}
            {Euclidean manifold with dense metric}
%
        \hiercmdarg{\indentarrow\indentarrow\indentarrow\indentarrow}{stepsize}{$$double$$}
          {Step size for discrete evolution}
          {Valid values: 0 $<$ stepsize}
          {Defaults to 1}
%
        \hiercmdarg{\indentarrow\indentarrow\indentarrow\indentarrow}{stepsize\_jitter}{$$double$$}
          {Uniformly random jitter of the stepsize, in percent}
          {Valid values: 0 $\le$ stepsize\_jitter $\le$ 1}
          {Defaults to 0}
%
      \hiershortcmd{\indentarrow\indentarrow\indentarrow}{rwm}
        {Random Walk Metropolis Monte Carlo}
%
  \hierlongcmd{\indentarrow}{optimize}
    {Point estimation}
    {Valid subarguments: algorithm, iter, save\_iterations}
%
    \hiercmdarg{\indentarrow\indentarrow}{algorithm}{$$list element$$}
      {Optimization algorithm}
      {Valid values: nesterov, bfgs, newton}
      {Defaults to bfgs}
%
      \hierlongcmd{\indentarrow\indentarrow\indentarrow}{nesterov}
        {Nesterov's accelerated gradient method}
        {Valid subarguments: stepsize}
%
        \hiercmdarg{\indentarrow\indentarrow\indentarrow\indentarrow}{stepsize}{$$double$$}
          {Step size for discrete evolution}
          {Valid values: 0 $<$ stepsize}
          {Defaults to 1}
%
      \hierlongcmd{\indentarrow\indentarrow\indentarrow}{bfgs}
        {BFGS with linesearch}
        {Valid subarguments: stepsize}
%
        \hiercmdarg{\indentarrow\indentarrow\indentarrow\indentarrow}{stepsize}{$$double$$}
          {Step size for discrete evolution}
          {Valid values: 0 $<$ stepsize}
          {Defaults to 1}
%
      \hiershortcmd{\indentarrow\indentarrow\indentarrow}{newton}
        {Newton's method}
%
    \hiercmdarg{\indentarrow\indentarrow}{iter}{$$int$$}
      {Total number of iterations}
      {Valid values: 0 $<$ iter}
      {Defaults to 2000}
%
    \hiercmdarg{\indentarrow\indentarrow}{save\_iterations}{$$boolean$$}
      {Stream optimization progress to output?}
      {Valid values: [0, 1]}
      {Defaults to 0}
%
  \hierlongcmd{\indentarrow}{diagnose}
    {Model diagnostics}
    {Valid subarguments: test}
%
    \hiercmdarg{\indentarrow\indentarrow}{test}{$$list element$$}
      {Diagnostic test}
      {Valid values: gradient}
      {Defaults to gradient}
%
      \hiershortcmd{\indentarrow\indentarrow\indentarrow}{gradient}
        {Check model gradient against finite differences}
%
\hiercmdarg{}{id}{$$int$$}
  {Unique process identifier}
  {Valid values: id $>$ 0}
  {Defaults to 0}
%
\hiercmdarg{}{data}{$$string$$}
  {Input data file}
  {Valid values: Path to existing file}
  {Defaults to ``''}
%
\hiercmdarg{}{init}{$$string$$}
  {Initialization method: ``x'' initializes randomly bewteen [-x, x], ``0'' initializes to 0, anything else identifies a file of values}
  {Valid values: All}
  {Defaults to ``2''}
%
\hierlongcmd{}{random}
  {Random number configuration}
  {Valid subarguments: seed}
%
  \hiercmdarg{\indentarrow}{seed}{$$unsigned int$$}
    {Random number generator seed}
    {Valid values: seed $>$ 0, if negative seed is generated from time}
    {Defaults to -1}
%
\hierlongcmd{}{output}
  {File output options}
  {Valid subarguments: file, append\_sample, diagnostic\_file, append\_diagnostic, refresh}
%
  \hiercmdarg{\indentarrow}{file}{$$string$$}
    {Output file}
    {Valid values: Path to existing file}
    {Defaults to samples.csv}
%
  \hiercmdarg{\indentarrow}{append\_sample}{$$boolean$$}
    {Append sample output to existing file?}
    {Valid values: [0, 1]}
    {Defaults to 0}
%
  \hiercmdarg{\indentarrow}{diagnostic\_file}{$$string$$}
    {Auxiliary output file for diagnostic information}
    {Valid values: Path to existing file}
    {Defaults to ``''}
%
  \hiercmdarg{\indentarrow}{append\_diagnostic}{$$boolean$$}
    {Append diagnostic output to existing file?}
    {Valid values: [0, 1]}
    {Defaults to 0}
%
  \hiercmdarg{\indentarrow}{refresh}{$$int$$}
    {Number of interations between screen updates}
    {Valid values: 0 $<$ refresh}
    {Defaults to 100}
%
\end{description}

\chapter{Print Command for Output Analysis}\label{print-command.chapter}

Stan is distributed with a print command that is able to read in the
output of one or more Markov chains and summarize the posterior fits.
This operation mimics the \code{print(fit)} command in RStan, which
itself was modeled on the print functions from R2WinBUGS and R2jags.

\section{Building the Print Command}

Stan's \code{print} command is built along with \code{stanc} into the
\code{bin} directory.  It can be compiled directly using the makefile
as follows from the home directory into which Stan was unpacked (here
written as \code{<stan-home>}).
%
\begin{quote}
\begin{Verbatim}[fontsize=\small]
> cd <stan-home>
> make bin/print
\end{Verbatim}
\end{quote}
%
All the usual compiler options from Stan's makefile apply, such as
\code{O=\farg{N}} to set optimization level to \farg{N}, and
\code{CC=clang++} to set the compilation to use clang. 

\section{Running the Print Command}

The print command is executed on one or more samples.csv files.  These
files may be provided as command-line arguments separated by spaces.
That means that wildcards may be used, as they will be replaced by
space-separated file names by the operating system's command-line
interpreter. 

Suppose there are three samples files in a directory generated by
fitting a negative binomial model to a small data set.
%
\begin{quote}
\begin{Verbatim}[fontsize=\small]
> ls samples*.csv
\end{Verbatim}
%
\begin{Verbatim}[fontshape=sl,fontsize=\small]
samples1.csv	samples2.csv	samples3.csv
\end{Verbatim}
%
\begin{Verbatim}[fontsize=\small]
> bin/print samples*.csv
\end{Verbatim}
%
{
\begin{Verbatim}[fontshape=sl,fontsize=\footnotesize]
Inference for Stan model: negative_binomial_model
1 chains: each with iter=(1000); warmup=(0); thin=(1); 1000 iterations saved.

Warmup took (0.054) seconds, 0.054 seconds total
Sampling took (0.059) seconds, 0.059 seconds total

                Mean     MCSE   StdDev    5%   50%   95%  N_Eff  N_Eff/s  R_hat
lp__             -14  6.2e-02  1.0e+00   -16   -14   -13    283     4773   1.00
accept_stat__   0.88  5.6e-03  1.8e-01  0.51  0.95   1.0   1000    16881   1.00
stepsize__      0.30  1.3e-15  8.9e-16  0.30  0.30  0.30   0.50      8.5   1.00
treedepth__      1.4  2.6e-02  8.0e-01  0.00   1.0   2.0    946    15978   1.00
alpha             17  1.8e+00  2.5e+01   1.9   9.5    50    181     3054   1.00
beta              10  1.1e+00  1.4e+01   1.2   6.2    31    181     3057    1.0

Samples were drawn using hmc with nuts.
For each parameter, N_Eff is a crude measure of effective sample size,
and R_hat is the potential scale reduction factor on split chains (at 
convergence, R_hat=1).
\end{Verbatim}
}
\end{quote}
%\end{quote}
%
The posterior is skewed to the high side, resulting in posterior means
($\alpha=17$ and $\beta=11$) that are a long way away from the posterior
medians ($\alpha=9.4$ and $\beta=6.1$);  the posterior median is the
value listed under \code{50\%}, which is the 50th percentile of the
posterior values.

For Windows, the forward slash in paths need to be converted to backslashes.

\section{Command-line Options}

In addition to the filenames, \code{print} includes three flags to customize the output.  

\begin{description}
\longcmd{help}
{Prints usage information}
{No help output by default}
%
\cmdarg{sig\_figs}{int}
{Sets the number of significant figures displayed in the output}
{Valid values: 0 \textless sig\_figs}
{default = \code{2} }
%
\cmdarg{autocorr}{int}
{Calculates and then displays the autocorrelation of the specified chain}
{Valid values: Any integer matching a chain index}
{No autocorrelation output by default}
%
\end{description}

\chapter{Dump Data Format}\label{dump.chapter}

\noindent 
For representing structured data in files, \Stan uses the dump format
introduced in \SPLUS and used in \R and \JAGS (and in \BUGS, but with
a different ordering).   A dump file is structured as a sequence of
variable definitions.  Each variable is defined in terms of its
dimensionality and its values.   There are three kinds of variable
declarations, one for scalars, one for sequences, and one for general
arrays.

\section{Scalar Variables}

A simple scalar value can be thought of as having an empty list of
dimensions.  Its declaration in the dump format follows the \SPLUS
assignment syntax.  For example, the following would constitute a
valid dump file defining a single scalar variable \code{y} with value
17.2.
%
\begin{quote}
\begin{Verbatim}[fontsize=\small]
y <- 
17.2
\end{Verbatim}
\end{quote}
%
A scalar value is just a zero-dimensional array value.

\section{Sequence Variables}

One-dimensional arrays may be specified directly using the \SPLUS
sequence notation.  The following example defines an integer-value and
a real-valued sequence.
%
\begin{quote}
\begin{Verbatim}[fontsize=\small]
n <- c(1,2,3)
y <- c(2.0,3.0,9.7)
\end{Verbatim}
\end{quote}
%
Arrays are provided without a declaration of dimensionality because
the reader just counts the number of entries to determine the size of
the array.

Sequence variables may alternatively be represented with \R's
colon-based notation.  For instance, the first example above could
equivalently be written as
%
\begin{quote}
\begin{Verbatim} 
n <- 1:3
\end{Verbatim}
\end{quote}
% 
The sequence denoted by \code{1:3} is of length 3, running from 1 to 3
inclusive.  The colon notation allows sequences going from high to
low, as in the first of the following examples, which is equivalent to
the second.
%
\begin{quote}
\begin{Verbatim}[fontsize=\small]
n <- 2:-2
n <- c(2,1,0,-1,-2)
\end{Verbatim}
\end{quote}
%


\section{Array Variables}

For more than one dimension, the dump format uses a dimensionality
specification.  For example,
%
\begin{quote}
\begin{verbatim}
y <- structure(c(1,2,3,4,5,6), .Dim = c(2,3))
\end{verbatim}
\end{quote}
%
This defines a $2 \times 3$ array.  Data is stored in column-major
order, meaning the values for \code{y} will be as follows.
%
\begin{quote}
\begin{Verbatim}[fontsize=\small]
y[1,1] = 1     y[1,2] = 3     y[1,3] = 5    
y[2,1] = 2     y[2,2] = 4     y[2,3] = 6
\end{Verbatim}
\end{quote}
%
The \code{structure} keyword just wraps a sequence of values and a
dimensionality declaration, which is itself just a sequence of
non-negative integer values.  The product of the dimensions must equal
the length of the array.

If the values happen to form a contiguous sequence of integers,
they may be written with colon notation.  Thus the example above is
equivalent to the following.
%
\begin{quote}
\begin{verbatim}
y <- structure(1:6, .Dim = c(2,3))
\end{verbatim}
\end{quote}
%
The same applies to the specification of dimensions, though it is
perhaps less likely to be used. In the above example,
c(2,3) could be written as \code{2:3}.

Arrays of more than two dimensions are written in a last-index major form.
For example, 
%
\begin{quote}
\begin{verbatim}
z <- structure(1:24, .Dim = c(2,3,4))
\end{verbatim}
\end{quote}
%
produces a three-dimensional \code{int} (assignable to \code{real})
array \code{z} with values
%
\begin{quote}
\begin{verbatim}
z[1,1,1] =  1   z[1,2,1] =  3   z[1,3,1] =  5
z[2,1,1] =  2   z[2,2,1] =  4   z[2,3,1] =  6

z[1,1,2] =  7   z[1,2,2] =  9   z[1,3,2] = 11
z[2,1,2] =  8   z[2,2,2] = 10   z[2,3,2] = 12

z[1,1,3] = 13   z[1,2,3] = 15   z[1,3,3] = 17
z[2,1,3] = 14   z[2,2,3] = 16   z[2,3,3] = 18

z[1,1,4] = 19   z[1,2,4] = 21   z[1,3,4] = 23
z[2,1,4] = 20   z[2,2,4] = 22   z[2,3,4] = 24
\end{verbatim}
\end{quote}



\section{Integer- and Real-Valued Variables}

There is no declaration in a dump file that distinguishes integer
versus continuous values.  If a value in a dump file's definition of a
variable contains a decimal point, \Stan assumes that the values are
real.  

For a single value, if there is no decimal point, it may be assigned
to an \code{int} or \code{real} variable in \Stan.  An array value may
only be assigned to an \code{int} array if there is no decimal point
in any of the values.  This convention is compatible with the way \R
writes data.

The following dump file declares an integer value for \code{y}.
%
\begin{quote}
\begin{Verbatim} 
y <- 
2
\end{Verbatim}
\end{quote}
% 
This definition can be used for a \Stan variable \code{y} declared as
\code{real} or as \code{int}.  Assigning an integer value to a real
variable automatically promotes the integer value to a real value.

Integer values may optionally be followed by \code{L} or \code{l},
denoting long integer values.  The following example, where the type is
explicit, is equivalent to the above.
%
\begin{quote}
\begin{Verbatim} 
y <- 
2L
\end{Verbatim}
\end{quote}

The following dump file provides a real value for \code{y}.
%
\begin{quote}
\begin{Verbatim}[fontsize=\small]
y <-
2.0
\end{Verbatim}
\end{quote}
%
Even though this is a round value, the occurrence of the decimal
point in the value, \code{2.0}, causes \Stan to infer that \code{y} is
real valued.  This dump file may only be used for variables \code{y}
declared as real in \Stan.


\subsection{Infinite and Not-a-Number Values}

Stan's reader supports infinite and not-a-number values for scalar
quantities (see \refsection{real-data-type} for more information).
Both infinite and not-a-number values are supported by Stan's
dump-format readers.  
%
\begin{center}
\begin{tabular}{r||c|c}
{\it Value} & {\it Preferred Form} & {\it Alternative Forms} \\ \hline \hline
positive infinity & \code{Inf} & \code{Infinity},
\code{infinity}
\\
negative infinity & \code{-Inf} & \code{-Infinity},
\code{-infinity}
\\
not a number & \code{NaN} & 
\end{tabular}
\end{center}
%
These strings are not case sensitive, so \code{inf} may also be used
for positive infinity, or \code{NAN} for not-a-number.

\section{Quoted Variable Names}

In order to support \JAGS data files, variables may be double quoted.
For instance, the following definition is legal in a dump file.
%
\begin{quote}
\begin{Verbatim}[fontsize=\small]
"y" <-
c(1,2,3)
\end{Verbatim}
\end{quote}

\section{Line Breaks}

The line breaks in a dump file are required to be consistent with
the way \R reads in data.  Both of the following declarations are
legal.
%
\begin{quote}
\begin{Verbatim}[fontsize=\small]
y <- 2
y <-
3
\end{Verbatim}
\end{quote}
%
Also following \R, breaking before the assignment arrow are not
allowed, so the following is invalid.
%
\begin{quote}
\begin{Verbatim}[fontsize=\small]
y
<- 2  # Syntax Error
\end{Verbatim}
\end{quote}

Lines may also be broken in the middle of sequences declared
using the \code{c(...)} notation., as well as between the comma
following a sequence definition and the dimensionality declaration.
For example, the following declaration of a $2 \times 2 \times 3$
array is valid.
%
\begin{quote}
\begin{Verbatim}[fontsize=\small]
y <-
structure(c(1,2,3,
4,5,6,7,8,9,10,11,
12), .Dim = c(2,2,
3))
\end{Verbatim}
\end{quote}
%
Because there are no decimal points in the values, the resulting dump
file may be used for three-dimensional array variables declared as
\code{int} or \code{real}.

\section{BNF Grammar for Dump Data}

A more precise definition of the dump data format is provided
by the following (mildly templated) Backus-Naur form grammar.

{\small 
\begin{verbatim}
 definitions ::= definition+

 definition ::= name ("<-" | '=') value optional_semicolon

 name ::= char* 
        | ''' char* ''' 
        | '"' char* '"'

 value ::= value<int> | value<double>

 value<T> ::= T 
            | seq<T>
            | 'structure' '(' seq<T> ',' ".Dim" '=' seq<int> ')'

 seq<int> ::= int ':' int
            | cseq<int>

 seq<real> ::= cseq<real>

 cseq<T> ::= 'c' '(' vseq<T> ')'

 vseq<T> ::= T
           | T ',' vseq<T>
\end{verbatim}
}
\noindent
The template parameters \code{T} will be set to either \code{int} or
\code{real}.  Because \Stan allows promotion of integer values to real
values, an integer sequence specification in the dump data format may
be assigned to either an integer- or real-based variable in \Stan.










