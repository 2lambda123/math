


\part{Built-In Functions}\label{built-in-functions.part}


\chapter{Integer-Valued Basic Functions}

\noindent
This chapter describes \Stan's built-in function that take various
types of arguments and return results of type integer.


\section{Integer-Valued Arithmetic Operators}

\Stan's arithmetic is based on standard double-precision \Cpp integer and
floating-point arithmetic.  If the arguments to an arithmetic operator
are both integers, as in \code{2+2}, integer arithmetic is used.  If
one argument is an integer and the other a floating-point value, as in
\code{2.0+2}, \code{2+2.0}, then the integer is promoted to a floating
point value and floating-point arithmetic is used.

Integer arithmetic behaves slightly differently than floating point
arithmetic.  The first difference is how overflow is treated.  If the
sum or product of two integers overflows the maximum integer
representable, the result is an undesirable wraparound behavior at the
bit level.  If the integers were first promoted to real numbers, they
would not overlflow a floating-point representation.  There are no
extra checks in \Stan to flag overflows, so it is up to the user to
make sure it does not occur.

Secondly, because the set of integers is not closed under division and
there is no special infinite value for integers, integer division
behaves differently.  For integer division by values other than zero,
the result is rounded toward 0.  For instance, positive results are
rounded down, as in \code{1/2 = 0} and \code{5/3 = 1}, whereas
negative results are rounded up, as in \code{(-1)/2 = 0} and
\code{(-5)/33 = -1}.

Unlike floating point division, where \code{1.0/0.0} produces the
special positive infinite value, integer division by zero, as in
\code{1/0}, has undefined behavior in the \Cpp standard.  For example,
the \clang compiler on Mac OS X returns 3764, whereas the \gpp
compiler throws a floating-point exception and aborts the program with
a warning.  As with overflow, it is up to the user to make sure
integer divide-by-zero does not occur.

\subsection{Binary Infix Operators}

Operators are described using the \Cpp syntax.  For instance, the
binary operator of addition, written \code{X + Y}, would have the
\Stan signature \code{int operator+(int,int)} indicating it takes
two real arguments and returns a real value.

\begin{description}
%
\fitem{int}{operator+}{int \farg{x}, int \farg{y}}{
The sum of the addends \farg{x} and \farg{y}}
%
\fitem{int}{operator-}{int \farg{x}, int \farg{y}}{
The difference between the minuend \farg{x} and subtrahend \farg{y}}
%
\fitem{int}{operator*}{int \farg{x}, int \farg{y}}{
The product of the factors \farg{x} and \farg{y}}
%
\fitem{int}{operator/}{int \farg{x}, int \farg{y}}{
The integer quotiont of the dividend \farg{x} and divisor \farg{y}}
%
\end{description}

\subsection{Unary Prefix Operators}

\begin{description}
\fitem{int}{operator-}{int \farg{x}}{
The negation of the the subtrahend \farg{x}}

\fitem{int}{operator+}{int \farg{x}}{
The promotion of addend \farg{x} to a int, which is a no-op
for int inputs}
\end{description}

\section{Absolute Functions}

\begin{description}
%
\fitem{int}{abs}{int \farg{x}}{
The absolute value of \farg{x}}
%
\fitem{int}{int\_step}{int \farg{x}}
1 if \farg{x} is strictly greater than 0, and 0 otherwise
%
\end{description}
%

\section{Bound Functions}
%
\begin{description}
\fitem{int}{min}{int \farg{x}, int \farg{y}}{
The minimum of \farg{x} and \farg{y}}
%
\fitem{int}{max}{int \farg{x}, int \farg{y}}{
The maximum of \farg{x} and \farg{y}}
%
\end{description}


\chapter{Real-Valued Basic Functions}

\noindent
This chapter describes built-in functions that take zero or more real
or integer arguments and return real values.  Constants are
represented as functions with no arguments and must be called as such.

\section{Mathematical Constants}\label{built-in-constants.section}

%
\begin{description}
%
\fitem{real}{pi}{}{
  $\pi$, the ratio of a circle's circumference to its diameter}
%
\fitem{real}{e}{}{
 $e$, the base of the natural logarithm}
%
\fitem{real}{sqrt2}{}{
The square root of 2}
%
\fitem{real}{log2}{}{
The natural logarithm of 2}
%
\fitem{real}{log10}{}{
The natural logarithm of 10}
%
\end{description}

\section{Special Values}

\begin{description}
\fitem{real}{nan}{}{
Not-a-number, a special non-finite real value returned to signal an error}
%
\fitem{real}{infinity}{}{
 Positive infinity, a special non-finite real value larger than all
  finite numbers}
%
\fitem{real}{negative\_infinity}{}{ 
 Negative infinity, a special non-finite real value smaller than all
  finite numbers}
%
\fitem{real}{epsilon}{}{
The smallest positive real value representable}
%
\fitem{real}{negative\_epsilon}{}{
The largest negative real value representable}
%
\end{description}

\section{Logical Functions}

\begin{description}
%
\fitem{real}{if\_else}{int cond, real \farg{x}, real \farg{y}}{
\farg{x} if \farg{cond} is non-zero, and \farg{y} otherwise}
%
\fitem{real}{step}{real \farg{x}}{
0 if \farg{x} is negative and 1 otherwise}
%
\end{description}


\section{Real-Valued Arithmetic Operators}\label{real-valued-arithmetic-operators.section}

The arithmetic operators are presented using \Cpp notation.  For
instance \code{operator+(x,y)} refers to the binary addition operator
and \code{operator-(x)} to the unary negation operator.  In \Stan
programs, these are written using the usual infix and prefix notations
as \code{x~+~y} and \code{-x}, respectively.

\subsection{Binary Infix Operators}

\begin{description}
%
\fitem{real}{operator+}{real \farg{x}, real \farg{y}}{
The sum of the addends \farg{x} and \farg{y}}
%
\fitem{real}{operator-}{real \farg{x}, real \farg{y}}{
The difference between the minuend \farg{x} and subtrahend \farg{y}}
%
\fitem{real}{operator*}{real \farg{x}, real \farg{y}}{
The product of the factors \farg{x} and \farg{y}}
%
\fitem{real}{operator/}{real \farg{x}, real \farg{y}}{
The quotiont of the dividend \farg{x} and divisor \farg{y}}
%
\end{description}

\subsection{Unary Prefix Operators}

\begin{description}
\fitem{real}{operator-}{real \farg{x}}{
The negation of the the subtrahend \farg{x}}

\fitem{real}{operator+}{real \farg{x}}{
The promotion of addend \farg{x} to a real, which is a no-op
for real inputs}
\end{description}


\section{Absolute Functions}

\begin{description}
%
\fitem{real}{abs}{real \farg{x}}{
The absolute value of \farg{x}}
%
\fitem{real}{fabs}{real \farg{x}}{
The absolute value of \farg{x}}
%
\fitem{real}{fdim}{real \farg{x}, 
                 real \farg{y}}{
The positive difference between \farg{x} and \farg{y}, which is
\farg{x} - \farg{y} if \farg{x} is greater than \farg{y} and 0 otherwise}
%
\end{description}

\section{Bounds Functions}

\begin{description}
%
\fitem{real}{fmin}{real \farg{x}, real \farg{y}}{
The minimum of \farg{x} and \farg{y}}
%
\fitem{real}{fmax}{real \farg{x}, real \farg{y}}{
The maximum of \farg{x} and \farg{y}}
%
\end{description}

\section{Arithmetic Functions}
%
\begin{description}
\fitem{real}{fmod}{real \farg{x}, real \farg{y}}{
The real value remainder after dividing \farg{x} by \farg{y}}
%
\end{description}

\section{Rounding Functions}

\begin{description}
%
\fitem{real}{floor}{real \farg{x}}{
The floor of \farg{x}, which is the largest integer less
than or equal to \farg{x}, converted to a real value}
%
\fitem{real}{ceil}{real \farg{x}}{
The ceiling of \farg{x}, which is the smallest integer greater
than or equal to \farg{x}, converted to a real value}
%
\fitem{real}{round}{real \farg{x}}{
The nearest integer to \farg{x}, converted to a real value}
%
\fitem{real}{trunc}{real \farg{x}}{
The integer nearest to but no larger in magnitude than \farg{x},
converted to a double value}
%
\end{description}

\section{Power and Logarithm Functions}

\begin{description}
%
\fitem{real}{sqrt}{real \farg{x}}{
The square root of \farg{x}}
%
\fitem{real}{cbrt}{real \farg{x}}{
The cube root of \farg{x}}
%
\fitem{real}{square}{real \farg{x}}{
The square of \farg{x}}
%
\fitem{real}{exp}{real \farg{x}}{
The natural exponential of \farg{x}}
%
\fitem{real}{exp2}{real \farg{x}}{
The base-2 exponential of \farg{x}}
%
\fitem{real}{expm1}{real \farg{x}}{
The natural exponential of \farg{x} minus 1}
%
\fitem{real}{log}{real \farg{x}}{
The natural logarithm of \farg{x}}
%
\fitem{real}{log2}{real \farg{x}}{
The base-2 logarithm of \farg{x}}
%
\fitem{real}{log10}{real \farg{x}}{
The base-10 logarithm of \farg{x}}
%
\fitem{real}{pow}{real \farg{x}, real \farg{y}}{
\farg{x} raised to the power of \farg{y}}
%
\end{description}




\section{Link Functions}

\begin{description}
%
\fitem{real}{logit}{real \farg{x}}{
The log odds, or logit, function applied to \farg{x}}
%
\fitem{real}{inv\_logit}{real \farg{x}}{
The logistic sigmoid function applied to \farg{x}}
%
\fitem{real}{inv\_cloglog}{real \farg{x}}{
The inverse of the complement log-log function applied to \farg{x}}
%
\end{description}


\section{Trigonometric Functions}

\begin{description}
%
\fitem{real}{hypot}{real \farg{x}, real \farg{y}}{
The length of the hypoteneuse of a right triangle with sides of
length \farg{x} and \farg{y}}
%
\fitem{real}{cos}{real \farg{x}}{
The cosine of the angle \farg{x} (in radians)}
%
\fitem{real}{sin}{real \farg{x}}{
The sine of the angle \farg{x} (in radians)}
%
\fitem{real}{tan}{real \farg{x}}{
The tangent of the angle \farg{x} (in radians)}
%
\fitem{real}{acos}{real \farg{x}}{
The princpipal arc (inverse) cosine (in radians) of \farg{x}}
%
\fitem{real}{asin}{real \farg{x}}{
The principal arc (inverse) sine (in radians) of \farg{x}}
%
\fitem{real}{atan}{real \farg{x}}{
The principal arc (inverse) tangent (in radians) of \farg{x}}
%
\fitem{real}{atan2}{real \farg{x}, real \farg{y}}{
The principal arc (inverse) tangent (in radians) of \farg{x} divided
by \farg{y}}
%
\end{description}

\section{Hyperbolic Trigonometric Functions}

\begin{description}
%
\fitem{real}{cosh}{real \farg{x}}{
The hyperbolic cosine of \farg{x} (in radians)}
%
\fitem{real}{sinh}{real \farg{x}}{
The hyperbolic sine of \farg{x} (in radians)}
%
\fitem{real}{tanh}{real \farg{x}}{
The hyperbolic tangent of \farg{x} (in radians)}
%
\fitem{real}{acosh}{real \farg{x}}{
The inverse hyperbolic cosine (in radians) of \farg{x}}
%
\fitem{real}{asinh}{real \farg{x}}{
The inverse hyperbolic sine (in radians) of \farg{x}}
%
\fitem{real}{atanh}{real \farg{x}}{
The inverse hyperbolic tangent (in radians) of \farg{x}}
%
\end{description}

\section{Probability-Related Functions}

\begin{description}
%
\fitem{real}{erf}{real \farg{x}}{
The error function of \farg{x}}
%
\fitem{real}{erfc}{real \farg{x}}{
The complementary error function of \farg{x}}
%
\fitem{real}{Phi}{real \farg{x}}{
The cumulative unit normal density function of \farg{x}}
%
\fitem{real}{log\_loss}{int \farg{y}, real \farg{y\_hat}}{
The log loss of predicting probabilty \farg{y\_hat} for 
binary outcome \farg{y}}
%
\end{description}




\section{Combinatorial Functions}

\begin{description}
%
\fitem{real}{tgamma}{real \farg{x}}{
The gamma function applied to \farg{x}}
%
\fitem{real}{lgamma}{real \farg{x}}{
The natural logarithm of the gamma function applied to \farg{x}}
%
\fitem{real}{lmgamma}{int \farg{n}, real \farg{x}}{
The natural logarithm of the multinomial gamma function with \farg{n}
dimensions applied to \farg{x}}
%
\fitem{real}{lbeta}{real \farg{x}, real \farg{y}}{
The natural logarithm of the beta function applied to \farg{x}}
%
\fitem{real}{binomial\_coefficient\_log}{real \farg{x}, real \farg{y}}{
The natural logarithm of the binomial coefficient of \farg{x} choose
\farg{y}, generalized to real values via the gamma function}
%
\end{description}


\section{Composed Functions}

\begin{description}
%
\fitem{real}{fma}{real \farg{x}, 
               real \farg{y},
               real \farg{z}}{
\farg{z} plus the result of \farg{x} multiplied by \farg{y}}
%
\fitem{real}{multiply\_log}{real \farg{x}, real \farg{y}}{ 
The product of \farg{x} and the natural logarithm of \farg{y}}
%
\fitem{real}{log1p}{real \farg{x}}{
The natural logarithm of 1 plus \farg{x}}
%
\fitem{real}{log1m}{real \farg{x}}{
The natural logarithm of 1 minus \farg{x}}
%
\fitem{real}{log1p\_exp}{real \farg{x}}{ 
The natural logarithm of one plus the natural exponentiation of
\farg{x}}
%
\fitem{real}{log\_sum\_exp}{real \farg{x}, real \farg{y}}{ 
The natural logarithm of the sum of the natural exponentiation
of \farg{x} and the natural exponentiation of \farg{y}}
%
\end{description}




\chapter{Array Operations}
\noindent

\begin{description}
\fitem{real}{sum}{real \farg{x}[]}
The sum of the elements in \farg{x}
%
\fitem{real}{mean}{real \farg{x}[]}
The sample mean of the elements in \farg{x}
%
\fitem{real}{variance}{real \farg{x}[]}
The sample variance of the elements in \farg{x} (based on
dividing by \code{length - 1})
%
\fitem{real}{sd}{real \farg{x}[]} The sample standard deviation of
elements in \farg{x} (divide by \code{length - 1})
%
\fitem{real}{log\_sum\_exp}{real \farg{x}[]}
The natural logarithm of the sum of the exponentials of the elements in \farg{x}
\end{description}


\chapter{Matrix Operations}\label{matrix-operations.chapter}
\noindent

\section{Integer-Valued Matrix Size Functions}

\begin{description}
%
\fitem{int}{rows}{vector \farg{x}}{The number of
rows in the vector \farg{x}}
%
\fitem{int}{rows}{row\_vector \farg{x}}{The number of
rows in the row vector \farg{x}, namely 1}
%
\fitem{int}{rows}{matrix \farg{x}}{The number of
rows in the matrix \farg{x}}
%
\fitem{int}{cols}{vector \farg{x}}{The number of columns
in the vector \farg{x}, namely 1}
%
\fitem{int}{cols}{row\_vector \farg{x}}{The number of columns
in the row vector \farg{x}}
%
\fitem{int}{cols}{matrix \farg{x}}{The number of columns
in the matrix \farg{x}}
\end{description}


\section{Matrix Arithmetic Operators}

\Stan supports the basic matrix operations using infix, prefix and
postfix operations.  This section lists the operations supported by
\Stan along with their argument and result types.

\subsection{Negation Prefix Operators}

\begin{description}
%
\fitem{vector}{operator-}{vector \farg{x}}{The negation of the vector
\farg{x}}
%
\fitem{row\_vector}{operator-}{row\_vector \farg{x}}{The negation of the row
vector \farg{x}}
%
\fitem{matrix}{operator-}{matrix \farg{x}}{The negation of the matrix
  \farg{x}}
%
\end{description}


\subsection{Infix Matrix Operators}

\begin{description}
%
\fitem{vector}{operator+}{vector \farg{x}, vector \farg{y}}{The sum of
the vectors \farg{x} and \farg{y}}
%
\fitem{row\_vector}{operator+}{row\_vector \farg{x}, row\_vector \farg{y}}{The sum of
the row vectors \farg{x} and \farg{y}}
%
\fitem{matrix}{operator+}{matrix \farg{x}, matrix \farg{y}}{The sum of
the matrices \farg{x} and \farg{y}}
%
\end{description}
\vspace*{-4pt}
\begin{description}
\fitem{vector}{operator-}{vector \farg{x}, vector \farg{y}}{The difference between
the vectors \farg{x} and \farg{y}}
%
\fitem{row\_vector}{operator-}{row\_vector \farg{x}, row\_vector \farg{y}}{The
  difference between the the row vectors \farg{x} and \farg{y}}
%
\fitem{matrix}{operator-}{matrix \farg{x}, matrix \farg{y}}{The difference between
  the matrices \farg{x} and \farg{y}}
%
\end{description}
\vspace*{-4pt}
\begin{description}
%
\fitem{vector}{operator*}{real \farg{x}, vector \farg{y}}{The product of
the scalar \farg{x} and vector \farg{y}}
%
\fitem{row\_vector}{operator*}{real \farg{x}, row\_vector \farg{y}}{The product of
the scalar \farg{x} and the row vector \farg{y}}
%
\fitem{matrix}{operator*}{real \farg{x}, matrix \farg{y}}{The product of
the scalar \farg{x} and the matrix \farg{y}}
%
%
\fitem{vector}{operator*}{vector \farg{x}, real \farg{y}}{The product of
the scalar \farg{y} and vector \farg{x}}
%
\fitem{matrix}{operator*}{vector \farg{x}, row\_vector \farg{y}}{The product
of the row vector \farg{x} and vector \farg{y}}
%
%
\fitem{row\_vector}{operator*}{row\_vector \farg{x}, real \farg{y}}{The product of
the scalar \farg{y} and row vector \farg{x}}
%
\fitem{real}{operator*}{row\_vector \farg{x}, vector \farg{y}}{The product
of the row vector \farg{x} and vector \farg{y}}
%
\fitem{row\_vector}{operator*}{row\_vector \farg{x}, matrix \farg{y}}{The product
of the row vector \farg{x} and matrix \farg{y}}
%
%
\fitem{matrix}{operator*}{matrix \farg{x}, real \farg{y}}{The product of
the scalar \farg{y} and matrix \farg{x}}
%
\fitem{vector}{operator*}{matrix \farg{x}, vector \farg{y}}{The
  product of the matrix \farg{x} and vector \farg{y}}
%
\fitem{matrix}{operator*}{matrix \farg{x}, matrix \farg{y}}{The product of 
  the matrices \farg{x} and \farg{y}}
%
\end{description}
%



\subsection{Broadcast Infix Operators}

\begin{description}
%
\fitem{matrix}{operator+}{real \farg{x}, vector \farg{y}}{The result of
adding \farg{x} to every entry in the vector \farg{y}}
%
\fitem{matrix}{operator+}{real \farg{x}, row\_vector \farg{y}}{The result of
adding \farg{x} to every entry in the row vector \farg{y}}
%
\fitem{matrix}{operator+}{real \farg{x}, matrix \farg{y}}{The result of
adding \farg{x} to every entry in the matrix \farg{y}}
%
\fitem{matrix}{operator+}{vector \farg{x}, real \farg{y}}{The result of
adding \farg{y} to every entry in the vector \farg{x}}
%
\fitem{matrix}{operator+}{row\_vector \farg{x}, real \farg{y}}{The result of
adding \farg{y} to every entry in the row vector \farg{x}}
%
\fitem{matrix}{operator+}{matrix \farg{x}, real \farg{y}}{The result of
adding \farg{y} to every entry in the matrix \farg{x}}
%
\end{description}
\vspace*{-4pt}
\begin{description}
%
\fitem{vector}{operator-}{vector \farg{x}, real \farg{y}}{The result of
subtracting \farg{y} from every entry in the vector \farg{x}}
%
\fitem{vector}{operator-}{real \farg{x}, vector \farg{y}}{The result of
adding \farg{x} to every entry in the negation of the vector \farg{y}}
%
\fitem{row\_vector}{operator-}{row\_vector \farg{x}, real \farg{y}}{The result of
subtracting \farg{y} from every entry in the row vector \farg{x}}
%
\fitem{row\_vector}{operator-}{real \farg{x}, row\_vector \farg{y}}{The result of
adding \farg{x} to every entry in the negation of the row vector \farg{y}}
%
\fitem{matrix}{operator-}{matrix \farg{x}, real \farg{y}}{The result of
subtracting \farg{y} from every entry in the matrix \farg{x}}
%
\fitem{matrix}{operator-}{real \farg{x}, matrix \farg{y}}{The result of
adding \farg{x} to every entry in negation of the matrix \farg{y}}
%
\end{description}

\subsection{Elementwise Products}

\begin{description}
%
\fitem{vector}{operator.*}{vector \farg{x}, vector \farg{y}}{The
elementwise product of \farg{y} and \farg{x}}
%
\fitem{row\_vector}{operator.*}{row\_vector \farg{x}, row\_vector \farg{y}}{The
elementwise product of \farg{y} and \farg{x}}
%
\fitem{matrix}{operator.*}{matrix \farg{x}, matrix \farg{y}}{The
elementwise product of \farg{y} and \farg{x}}
\end{description}
\vspace*{-4pt}
\begin{description}
\fitem{vector}{operator./}{vector \farg{x}, vector \farg{y}}{The
elementwise quotient of \farg{y} and \farg{x}}
%
\fitem{row\_vector}{operator./}{row\_vector \farg{x}, row\_vector \farg{y}}{The
elementwise quotient of \farg{y} and \farg{x}}
%
\fitem{matrix}{operator./}{matrix \farg{x}, matrix \farg{y}}{The
elementwise quotient of \farg{y} and \farg{x}}
%
\end{description}
\vspace*{-4pt}

\subsection{Elementwise Logarithms}

\begin{description}
%
\fitem{vector}{log}{vector \farg{x}}{
The elementwise natural logarithm of \farg{x}}
%
\fitem{row\_vector}{log}{row\_vector \farg{x}}{
The elementwise natural logarithm of \farg{x}}
%
\fitem{matrix}{log}{matrix \farg{x}}{
The elementwise natural logarithm of \farg{x}}
%
\fitem{vector}{exp}{vector \farg{x}}{
The elementwise exponential of \farg{x}}
%
\fitem{row\_vector}{exp}{row\_vector \farg{x}}{
The elementwise exponential of \farg{x}}
%
\fitem{matrix}{exp}{matrix \farg{x}}{
The elementwise exponential of \farg{x}}
\end{description}



\subsection{Dot Products}

\begin{description}
%
\fitem{real}{dot\_product}{vector \farg{x}, vector \farg{y}}{
The dot product of \farg{x} and \farg{y}}
%
\fitem{real}{dot\_product}{vector \farg{x}, row\_vector \farg{y}}{
The dot product of \farg{x} and \farg{y}}
%
\fitem{real}{dot\_product}{row\_vector \farg{x}, vector \farg{y}}{
The dot product of \farg{x} and \farg{y}}
%
\fitem{real}{dot\_product}{row\_vector \farg{x}, row\_vector \farg{y}}{
The dot product of \farg{x} and \farg{y}}
%
\end{description}


\section{Reductions}

\subsection{Minimum and Maximum}

\begin{description}
%
\fitem{real}{min}{vector \farg{x}}{
The minimum value in \farg{x}}
%
\fitem{real}{min}{row\_vector \farg{x}}{
The minimum value in \farg{x}}
%
\fitem{real}{min}{matrix \farg{x}}{
The minimum value in \farg{x}}
%
\fitem{real}{max}{vector \farg{x}}{
The maximum value in \farg{x}}
%
\fitem{real}{max}{row\_vector \farg{x}}{
The maximum value in \farg{x}}
%
\fitem{real}{max}{matrix \farg{x}}{
The maximum value in \farg{x}}
%
\end{description}

\subsection{Sums and Products}

\begin{description}
%
\fitem{real}{sum}{vector \farg{x}}{
The sum of the values in \farg{x} (zero if size zero)}
%
\fitem{real}{sum}{row\_vector \farg{x}}{
The sum of the values in \farg{x} (zero if size zero)}
%
\fitem{real}{sum}{matrix \farg{x}}{
The sum of the values in \farg{x} (zero if size zero)}
%
%
\fitem{real}{prod}{vector \farg{x}}{
The product of the values in \farg{x} (one if size zero)}
%
\fitem{real}{prod}{row\_vector \farg{x}}{
The product of the values in \farg{x} (one if size zero)}
%
\fitem{real}{prod}{matrix \farg{x}}{
The product of the values in \farg{x} (one if size zero)}
%
\end{description}



\subsection{Sample Moments}

\begin{description}
%
\fitem{real}{mean}{vector \farg{x}}{
The sample mean of the values in \farg{x}}
%
\fitem{real}{mean}{row\_vector \farg{x}}{
The sample mean of the values in \farg{x}}
%
\fitem{real}{mean}{matrix \farg{x}}{
The sample mean of the values in \farg{x}
(divide by length minus 1)}
%
\vspace*{4pt}
%
\fitem{real}{variance}{vector \farg{x}}{
The sample variance of the values in \farg{x}
(divide by size minus 1)}
%
\fitem{real}{variance}{row\_vector \farg{x}}{ 
The sample variance of the
values in \farg{x} (divide by size minus 1)}
%
\fitem{real}{variance}{matrix \farg{x}}{
The sample variance of the values in 
farg{x} (divide by size minus 1)}
%
\vspace*{4pt}
%
\fitem{real}{sd}{vector \farg{x}}{
The sample standard deviation of the values in \farg{x}
(divide by size minus 1)}
%
\fitem{real}{sd}{row\_vector \farg{x}}{ 
The sample standard deviation of the
values in \farg{x} (divide by size minus 1)}
%
\fitem{real}{sd}{matrix \farg{x}}{
The sample standard deviation of the values in 
\farg{x} (divide by size minus 1)}
%
\end{description}


\section{Slice and Package Functions}

\subsection{Diagonal Matrices}

\begin{description}
%
\fitem{vector}{diagonal}{matrix \farg{x}}{The diagonal
of the matrix \farg{x}}
%
\fitem{matrix}{diag\_matrix}{vector \farg{x}}{The diagonal
matrix with diagonal \farg{x}}
%
\fitem{vector}{col}{matrix \farg{x}, int \farg{n}}{The \farg{n}-th column
of matrix \farg{x}}
%
\fitem{row\_vector}{row}{matrix \farg{x}, int \farg{m}}{The \farg{m}-th row
of matrix \farg{x}}
%
\end{description}


\subsection{Transposition Postfix Operator}

\begin{description}
%
\fitem{matrix}{operator'}{matrix \farg{x}}{The transpose of the matrix
 \farg{x}, written as \code{x'}}
%
\fitem{row\_vector}{operator'}{vector \farg{x}}{The transpose of the vector
 \farg{x}, written as \code{x'}}
%
\fitem{vector}{operator'}{row\_vector \farg{x}}{The transpose of the vector
 \farg{x}, written as \code{x'}}
%
\end{description}


\section{Linear Algebra Functions and Solvers}

\subsection{Matrix Division Infix Operators}
%
\begin{description}
%
\fitem{row\_vector}{operator/}{row\_vector \farg{b}, matrix \farg{A}}{
The right division of \farg{b} by \farg{A}; equivalently
\code{\farg{b} * inv(\farg{A})}}
%
\fitem{vector}{operator\textbackslash}{matrix \farg{A}, vector \farg{b}}
The left division of \farg{b} by \farg{A}; equivalently
\code{inv(\farg{A}) * \farg{b}}
%
\end{description}

\subsection{Special Matrix Functions}

The softmax function maps $\reals^K$ to the $K$-simplex by
\[
\mbox{softmax}(y)
 = \frac{\exp(y)}
        {\sum_{k=1}^K \exp(y_k)},
\]
%
where $\exp(y)$ is the componentwise exponentiation of $y$.

\begin{description}
\fitem{vector}{softmax}{vector \farg{x}}{
The softmax of \farg{x}}
\end{description}
%

\subsection{Linear Algebra Functions}

\begin{description}
%
\fitem{real}{trace}{matrix \farg{A}}{
The trace of \farg{A}}
%
\fitem{real}{determinant}{matrix \farg{A}}{
The determinant of \farg{A}}
%
\fitem{matrix}{inverse}{matrix \farg{A}}{
The inverse of \farg{A}}
%
\fitem{vector}{eigenvalues}{matrix \farg{A}}{
The vector of eigenvalues of \farg{A} in descending order}
%
\fitem{matrix}{eigenvalues}{matrix \farg{A}}{
The matrix with the eigenvectors of \farg{A} as columns}
%
\fitem{vector}{eigenvalues\_sym}{matrix \farg{A}}{
The vector of eigenvalues of a symmetric view of \farg{A} 
in descending order}
%
\fitem{vector}{eigenvectors\_sym}{matrix \farg{A}}{
The matrix with the eigenvectors of a symmetric view of
\farg{A}}
%
\fitem{matrix}{cholesky}{matrix \farg{A}}{
The lower-triangular Cholesky factor of \farg{A}}
%
\fitem{vector}{singular\_values}{matrix \farg{A}}{
The singular values of \farg{A} in descending order}
%


\end{description}



\section{Integer-Valued Matrix Size Functions}

\begin{description}
%
\fitem{int}{rows}{vector \farg{x}}{The number of
rows in the vector \farg{x}}
%
\fitem{int}{rows}{row\_vector \farg{x}}{The number of
rows in the row vector \farg{x}, namely 1}
%
\fitem{int}{rows}{matrix \farg{x}}{The number of
rows in the matrix \farg{x}}
%
\fitem{int}{cols}{vector \farg{x}}{The number of columns
in the vector \farg{x}, namely 1}
%
\fitem{int}{cols}{row\_vector \farg{x}}{The number of columns
in the row vector \farg{x}}
%
\fitem{int}{cols}{matrix \farg{x}}{The number of columns
in the matrix \farg{x}}
\end{description}



\chapter{Discrete Probabilities}\label{discrete-prob-functions.chapter}

\noindent
The various discrete probability functions are organized by their
support.

\section{Binary Discrete Probabilities}

Binary probability functions have support on $\setlist{0,1}$, where 1
represents the value true and 0 the value false.

\subsection{Bernoulli Distribution}

If $\theta \in [0,1]$, then for $c \in \setlist{0,1}$, 
\[
\distro{Bernoulli}(c|\theta)
=
\left\{
\begin{array}{ll}
\theta & \mbox{if } y = 1, \mbox{ and}
\\
1 - \theta & \mbox{if } y = 0.
\end{array}
\right.
\]
\begin{description}
%
\fitem{real}{bernoulli\_log}{int \farg{y}, real \farg{theta}}{The
  log Bernoulli probability mass of \farg{y} given chance of success
  \farg{theta}}
%
\end{description}

\subsection{Bernoulli Distribution, Logit Parameterization}

\Stan also supplies a direct parameterization in terms of a
logit-transformed chance-of-success parameter.  This parameterization
is more numerically stable if the chance-of-success parameter is on
the logit scale, as with the linear predictor in a logistic
regression.  

If $\alpha \in \reals$, then for $c \in \setlist{0,1}$,
\[
\distro{BernoulliLogit}(c|\alpha)
=
\distro{Bernoulli}(c|\mbox{logit}^{-1}(\alpha))
= 
\left\{
\begin{array}{ll}
\mbox{logit}^{-1}(\alpha) & \mbox{if } y = 1, \mbox{ and}
\\
1 - \mbox{logit}^{-1}(\alpha) & \mbox{if } y = 0.
\end{array}
\right.
\]
\begin{description}
%
\fitem{real}{bernoulli\_logit\_log}{int \farg{y}, real \farg{theta}}{The
  log Bernoulli probability mass of \farg{y} given logit chance of success
  \farg{theta}}
%
\end{description}

\section{Bounded Discrete Probabilities}\label{betafun.section}

Bounded discrete probability functions have support on
$\setlist{0,\ldots,N}$ for some upper bound $N$.

\subsection{Binomial Distribution}

If $N \in \nats$ and $\theta \in [0,1]$, then for $n \in
\{0,\ldots,N\}$,
\[
\distro{Binomial}(n|N,\theta)
= {N \choose n} \theta^n (1 - \theta)^{N - n}.
\]

\begin{description}
%
  \fitem{real}{beta\_binomial\_log}{int \farg{n}, int \farg{N}, real
    \farg{theta}}{The log binomial probability mass of \farg{n}
    successes in \farg{N} trials given chance of success \farg{theta}}
%
\end{description}


\subsection{Beta-Binomial Distribution}

If $N \in \nats$, $\alpha \in \posreals$, and $\beta \in \posreals$,
then for $n \in \setlist{0,\ldots,N}$,
\[
\distro{BetaBinomial}(n|N,\alpha,\beta)
= 
{N \choose n} \frac{\Betafun(n+\alpha, N -n +
  \beta)}{\Betafun(\alpha,\beta)},
\]
%
where the beta function $\Betafun(x,y)$ is defined for $x \in
\posreals$ and $y \in \posreals$ by
%
\[
\Betafun(x,y)
= \frac{\Gamma(x) \ \Gamma(y)}{\Gamma(x + y)}.
\]

\begin{description}
%
\fitem{real}{beta\_binomial\_log}{int \farg{n}, int \farg{N}, real
  \farg{alpha}, real \farg{beta}}{The log beta-binomial probability mass
  of \farg{n} successes in \farg{N} trials given prior success count
  (plus one) of \farg{alpha} and prior failure count (plus one) of
  \farg{beta}}
%
\end{description}


\subsection{Hypergeometric Distribution}

If $a \in \nats$, $b \in \nats$, and $N \in \setlist{0,\ldots,a+b}$,
then for $n \in \setlist{0,\ldots,\min(a,N)}$, 
\[
\distro{Hypergeometric}(n|N,a,b)
=
\frac{\normalsize{a \choose n} {b \choose {N - n}}}
     {\normalsize{{m + n} \choose N}}.
\]

\begin{description}
%
  \fitem{real}{hypergeometric\_log}{int \farg{n}, int \farg{N}, int
    \farg{a}, int \farg{b}}{The log hypergeometric probability mass of
    \farg{n} successes in \farg{N} trials given total success count of
    \farg{a} and total failure count of \farg{b}}

%
\end{description}

\subsection{Categorical Distribution}

If $N \in \nats$ and $\theta \in \mbox{$N$-simplex}$, then for $n \in
\setlist{1,\ldots,N}$, 
\[
\distro{Categorical}(n|\theta) = \theta_n.
\]

\begin{description}
%
\fitem{real}{categorical\_log}{int \farg{y}, vector \farg{theta}}{The
  log categorical probability mass function with outcome \farg{y} in 
$1:N$ given $N$-simplex distribution parameter \farg{theta}}
%
\end{description}

\subsection{Ordered Logistic Distribution}

If $K \in \nats$ with $K > 2$, $c \in \reals^{K-1}$ such that $c_k <
c_{k+1}$ for $k \in \setlist{1,\ldots,K-2}$, and $\eta \in \reals$, then for $k \in
\setlist{1,\ldots,K}$,
\[
\distro{OrderedLogistic}(k|\eta,c)
=
\left\{
\begin{array}{ll}
1 - \mbox{logit}^{-1}(\eta - c_1) & \mbox{if } k = 1,
\\[4pt]
\mbox{logit}^{-1}(\eta - c_{k-1}) - \mbox{logit}^{-1}(\eta -
c_{k})
& \mbox{if } 1 < k < K, \mbox{and}
\\[4pt]
\mbox{logit}^{-1}(\eta - c_{K-1}) - 0
& \mbox{if } k = K.
\end{array}
\right.
\]
%
The $k=1$ and $k=K$ edge cases can be subsumed into the general definition
by setting $c_0 = -\infty$ and $c_K = +\infty$ with
$\mbox{logit}^{-1}(-\infty) = 0$ and $\mbox{logit}^{-1}(\infty) = 1$.
%
\begin{description}
  \fitem{real}{ordered\_logistic\_log}{int \farg{k}, real \farg{eta},
    vector \farg{c}}{The log ordered logistic probability mass of
    \farg{k} given linear predictor \farg{eta} and cutpoints
    \farg{c}.}
\end{description}


\section{Unbounded Discrete Distributions}

The unbounded discrete distributions have support over the natural
numbers (i.e., the non-negative integers).

\subsection{Negative Binomial Distribution}

If $\alpha \in \posreals$ and $\beta \in \posreals$, then for $n \in
\nats$,
\[
\distro{NegativeBinomial}(n|\alpha,\beta)
= 
{{n + \alpha - 1} \choose {\alpha - 1}}
\,
\left( \frac{\beta}{\beta+1} \right)^{\!\alpha}
\,
\left( \frac{1}{\beta + 1} \right)^{\!n} \!.
\]

\begin{description}
%
 \fitem{real}{negative\_binomial\_log}{int \farg{n}, real
   \farg{alpha}, real \farg{beta}}{The log negative binomial probability
   mass of \farg{n} given shape \farg{alpha} and inverse scale \farg{beta}}
%
\end{description}


\subsection{Poisson Distribution}

If $\lambda \in \posreals$, then for $n \in \nats$,
\[
\distro{Poisson}(n|\lambda)
= 
\frac{\exp(-\lambda)}{\lambda!}
\,
\lambda^n.
\]

\begin{description}
\fitem{real}{poisson\_log}{int \farg{n}, real \farg{lambda}}{ The
  log Poisson probability mass of \farg{n} given rate \farg{lambda}}
\end{description}



\section{Multivariate Discrete Probabilities}

\subsection{Multinomial Distribution}

If $K \in \nats$, $N \in \nats$, and $\theta \in \mbox{$K$-simplex}$,
then for $y \in \nats^K$ such that $\sum_{k=1}^K y_k = N$,
%
\[
\distro{Multinomial}(y|\theta,N)
= {N \choose {y_1,\ldots,y_K}} 
\prod_{k=1}^K \theta_k^{y_k},
\]
where the multinomial coefficient is defined by
\[
{N \choose {y_1,\ldots,y_k}}
= \frac{N!}{\prod_{k=1}^K y_k!}.
\]

\begin{description}
 \fitem{real}{multinomial\_log}{int[] \farg{y}, vector
    \farg{theta}}{The log multinomial probability mass function with
    outcome array \code{y} of size $K$ given the $K$-simplex
    distribution parameter \farg{theta} and (implicit) total count
    \code{N = sum(\farg{y})}}
\end{description}


\chapter{Continuous Probabilities}\label{continuous-prob-functions.chapter}

\noindent
The various continuous probability functions are organized by their
support.

\section{Univariate Continuous Probabilities}

The univariate continous probability distributions have support on all
real numbers.

\subsection{Normal Distribution}

If $\mu \in \reals$ and $\sigma \in \posreals$, then for $y \in
\reals$,
\[
\distro{Normal}(y|\mu,\sigma)
=
\frac{1}{\sqrt{2 \pi} \ \sigma}
\exp\left( - \, \frac{1}{2}
           \left(  \frac{\theta - \mu}{\sigma} \right)^2
    \right)
\!.
\]

\begin{description}
  \fitem{real}{normal\_log}{real \farg{y}, real \farg{mu}, real
    \farg{sigma}}{The log of the normal density of \farg{y} given
    location \farg{mu} and scale \farg{sigma}}
%
\fitem{real}{normal\_log}{real[] \farg{y}, real \farg{mu}, real
  \farg{sigma}}{The sum for \code{n} from 1 to
  \code{size(\farg{y})} of the log of the normal density of
  \code{\farg{y}[n]} given location \farg{mu} and scale \farg{sigma}}
%
\fitem{real}{normal\_p}{real \farg{y}, real \farg{mu}, real
  \farg{sigma}}{The cumulative normal distribution of \farg{y}
  given location \farg{mu} and scale \farg{sigma}}
\end{description}


\subsection{Student-$t$ Distribution}

If $\nu \in \posreals$, $\mu \in \reals$, and $\sigma \in \posreals$,
then for $y \in \reals$,
\[
\distro{StudentT}(y|\nu,\mu,\sigma)
=
\frac{\Gamma\left((\nu + 1)/2\right)}
     {\Gamma(\nu/2)}
\
\frac{1}{\sqrt{\nu \pi} \ \sigma}
\
\left(
1 + \frac{1}{\nu} \left(\frac{\theta - \mu}{\sigma}\right)^2
\right)^{-(\nu + 1)/2}
\! .
\]

\begin{description}
  \fitem{real}{student\_t\_log}{real \farg{y}, real \farg{nu}, real
    \farg{mu}, real \farg{sigma}}{The log of the Student-$t$ density of \farg{y}
    given degrees of freedom \farg{nu}, location \farg{mu}, and scale
    \farg{sigma}}
\end{description}

\subsection{Cauchy Distribution}

If $\mu \in \reals$ and $\sigma \in \posreals$, then for $y \in \reals$,
\[
\distro{Cauchy}(y|\mu,\sigma)
= 
\frac{1}{\pi \sigma}
\
\frac{1}{1 + \left((y - \mu)/\sigma\right)^2}
.
\]

\begin{description}
%
\fitem{real}{cauchy\_log}{real \farg{y}, real \farg{mu}, real
 \farg{sigma}}{ The log of the Cauchy density of \farg{y} given location
 \farg{mu} and scale \farg{sigma}}
%
\end{description}

\subsection{Double Exponential (Laplace) Distribution}

If $\mu \in \reals$ and $\sigma \in \posreals$, then for $y \in \reals$, 
\[
\distro{DoubleExponential}(y|\mu,\sigma)
= \frac{1}{2\sigma} 
  \exp \left( - \, \frac{|y - \mu|}{\sigma} \right)
.
\]

\begin{description}
\fitem{real}{double\_exponential\_log}{real \farg{y}, real \farg{mu},
real \farg{sigma}}{ The log of the double exponential density of \farg{y} given
location \farg{mu} and scale \farg{sigma}}
\end{description}

\subsection{Logistic Distribution}

If $\mu \in \reals$ and $\sigma \in \posreals$, then for $y \in \reals$,
\[
\distro{Logistic}(y|\mu,\sigma)
=
\frac{1}{\sigma}
\
\exp\!\left( - \, \frac{y - \mu}{\sigma} \right)
\
\left(1 + \exp \!\left( - \, \frac{y - \mu}{\sigma} \right) \right)^{\!-2}
\! .
\]

\begin{description}
\fitem{real}{logistic\_log}{real \farg{y}, real \farg{mu},
real \farg{sigma}}{ The log of the logistic density of \farg{y} given
location \farg{mu} and scale \farg{sigma}}
\end{description}


\section{Positive Continuous Probabilities}

The positive continuous probability functions have support on the
positive real numbers.

\subsection{Log-Normal Distribution}

If $\mu \in \reals$ and $\sigma \in \posreals$, then for $y \in
\posreals$, 
\[
\distro{LogNormal}(y|\mu,\sigma)
=
\frac{1}{\sqrt{2 \pi} \ \sigma}
\,
\frac{1}{y}
\
\exp \! \left(
       - \, \frac{1}{2}
       \, \left( \frac{\log y - \mu}{\sigma} \right)^2
     \right)
.
\]

\begin{description}
 \fitem{real}{lognormal\_log}{real \farg{y}, real \farg{mu}, real
  \farg{sigma}}{The log of the lognormal density of \farg{y} given location
  \farg{mu} and scale \farg{sigma}}
%
\fitem{real}{lognormal\_p}{real \farg{y}, real \farg{mu}, real
  \farg{sigma}}{The cumulative lognormal distribution of \farg{y}
  given location \farg{mu} and scale \farg{sigma}}
\end{description}

\subsection{Chi-Square Distribution}

If $\nu \in \posreals$, then for $y \in \posreals$,
\[
\distro{ChiSquare}(y|\nu)
=
\frac{2^{-\nu/2}}
    {\Gamma(\nu / 2)}
\,
y^{\nu/2 - 1}
\,
\exp \! \left( -\, \frac{1}{2} \, y \right)
.
\]

\begin{description}
\fitem{real}{chi\_square\_log}{real \farg{y}, real \farg{nu}}{The
  log of the Chi-square density of \farg{y} given degrees of freedom
  \farg{nu}}
\end{description}

\subsection{Inverse Chi-Square Distribution}

If $\nu \in \posreals$, then for $y \in \posreals$,
\[
\distro{InvChiSquare}(y|\nu)
= 
\frac{2^{-\nu/2}}
   {\Gamma(\nu / 2)}
\,
y^{-(\nu/2 - 1)}
\,
\exp\! \left( \! - \, \frac{1}{2} \, \frac{1}{y} \right)
.
\]

\begin{description}
\fitem{real}{inv\_chi\_square\_log}{real \farg{y}, real
  \farg{nu}}{The log of the inverse Chi-square density of \farg{y} given
  degrees of freedom \farg{nu}}
\end{description}

\subsection{Scaled Inverse Chi-Square Distribution}

If $\nu \in \posreals$ and $\sigma \in \posreals$, then for $y \in
\posreals$,
\[
\distro{ScaledInvChiSquare}(y|\nu)
= 
\frac{(\nu / 2)^{\nu/2}}
     {\Gamma(\nu / 2)}
\,
\sigma^{\nu}
\,
y^{-(\nu/2 + 1)}
\,
\exp \! \left( \!
   - \, \frac{1}{2} \, \nu \, \sigma^2 \, \frac{1}{y}
\right)
.
\]


\begin{description}
  \fitem{real}{scaled\_inv\_chi\_square\_log}{real \farg{y}, real
    \farg{nu}, real \farg{s}}{The log of the scaled inverse Chi-square
    density of \farg{y} given degrees of freedom \farg{nu} and scale
    \farg{s}}
\end{description}


\subsection{Exponential Distribution}

If $\beta \in \posreals$, then for $y \in \posreals$,
\[
\distro{Exponential}(y|\beta)
= 
\beta \, 
\exp ( - \beta \, y )
.
\]

\begin{description}
  \fitem{real}{exponential\_log}{real \farg{y}, real \farg{beta}}{The
    log of the exponential density of \farg{y} given inverse scale \farg{beta}}
%
\fitem{real}{exponential\_p}{real \farg{y}, real \farg{beta}}{The
  cumulative distribution function of \farg{y} given inverse scale
  \farg{beta}}
\end{description}

\subsection{Gamma Distribution}

If $\alpha \in \posreals$ and $\beta \in \posreals$, then for $y \in
\posreals$,
\[
\distro{Gamma}(y|\alpha,\beta)
=
\frac{\beta^{\alpha}}
     {\Gamma(\alpha)}
\,
y^{\alpha - 1}
\exp(-\beta \, y)
.
\]

\begin{description}
 \fitem{real}{gamma\_log}{real \farg{y}, real \farg{alpha}, real
  \farg{beta}}{The log of the gamma density of \farg{y} given shape
  \farg{alpha} and inverse scale \farg{beta}}
%
  \fitem{real}{gamma\_p}{real \farg{y}, real \farg{alpha}, real
   \farg{beta}}{The cumulative distribution function of \farg{y} given shape
   \farg{alpha} and inverse scale \farg{beta}}
\end{description}

\subsection{Inverse Gamma Distribution}

If $\alpha \in \posreals$ and $\beta \in \posreals$, then for $y \in
\posreals$,
\[
\distro{InvGamma}(y|\alpha,\beta)
= 
\frac{\beta^{\alpha}}
     {\Gamma(\alpha)}
\
y^{-(\alpha + 1)}
\,
\exp \! \left( \! - \beta \, \frac{1}{y} \right)
.
\]

\begin{description}
\fitem{real}{inv\_gamma\_log}{real \farg{y}, real \farg{alpha}, real
\farg{beta}}{The log of the inverse gamma density of \farg{y} given shape
\farg{alpha} and scale \farg{beta}}
\end{description}

\subsection{Weibull Distribution}

% k = alpha   lambda = sigma
If $\alpha \in \posreals$ and $\sigma \in [0,\infty)$, then for $y \in
\posreals$, 
\[
\distro{Weibull}(y|\alpha,\sigma)
= 
\frac{\alpha}{\sigma}
\,
\left( \frac{y}{\sigma} \right)^{\alpha - 1}
\,
\exp \! \left( \! - \left( \frac{y}{\sigma} \right)^{\alpha}  \right)
.
\]

\begin{description}
\fitem{real}{weibull\_log}{real \farg{y}, real \farg{alpha}, real
\farg{sigma}}{The log of the Weibull density of \farg{y} given shape
\farg{alpha} and scale \farg{sigma}}
%
\fitem{real}{weibull\_p}{real \farg{y}, real \farg{alpha}, real
\farg{sigma}}{The Weibull cumulative distribution of \farg{y} given shape
\farg{alpha} and scale \farg{sigma}}
\end{description}



\section{Positive Lower-Bounded Probabilities}

The positive lower-bounded probabilities have support on real values
above some positive minimum value.


\subsection{Pareto Distribution}

If $y_0 \in \posreals$ and $\alpha \in \posreals$, then for
$y \in \posreals$,
\[
\distro{Pareto}(y|y_0,\alpha)
=
\alpha 
\ y_0
\, \left( \frac{1}{y} \right)^{\!\alpha+1}
\! .
\]

\begin{description}
  \fitem{real}{pareto\_log}{real \farg{y}, real \farg{y\_min}, real
    \farg{alpha}}{The log of the Pareto density of \farg{y} given
    positive minimum value \farg{y\_min} and shape \farg{alpha}}
\end{description}



\section{$[0,1]$ Continuous Probabilities}

\subsection{Beta Distribution}

If $\alpha \in \posreals$ and $\beta \in \posreals$, then for $\theta
\in [0,1]$,
\[
\distro{Beta}(\theta|\alpha,\beta)
=
\frac{1}{\Betafun(\alpha,\beta)}
\,
\theta^{\alpha - 1}
\,
(1 - \theta)^{\beta - 1}
,
\] 
where the beta function $\Betafun()$ is as defined in \refsection{betafun}


\begin{description}
%
  \fitem{real}{beta\_log}{real \farg{theta}, real \farg{alpha}, real
    \farg{beta}}{The log of the beta density of \code{theta} in $[0,1]$
    given positive prior successes (plus one)
    \farg{alpha} and prior failures (plus one) \farg{beta}}
%
\end{description}

\section{Bounded Continuous Probabilities}

The bounded continuous probabilities have support on a finite interval
of real numbers.

\subsection{Uniform Distribution}

If $\alpha \in \reals$ and $\beta \in (\alpha,\infty)$, then for $y
\in [\alpha,\beta]$,
\[
\distro{Uniform}(y|\alpha,\beta)
= 
\frac{1}{\beta - \alpha}
.
\]

\begin{description}
\fitem{real}{uniform\_log}{real \farg{y}, real \farg{alpha}, real
  \farg{beta}}{The log of the uniform density of \farg{y} given lower bound
  \farg{alpha} and upper bound \farg{beta}}
\end{description}

\section{Simplex Probabilities}

The simplex probabilities have support on the unit $K$-simplex for a
specified $K$.  A $K$-dimensional vector $\theta$ is a unit
$K$-simplex if $\theta_k \geq 0$ for $k \in \setlist{1,\ldots,K}$ and
$\sum_{k = 1}^K \theta_k = 1$.

\subsection{Dirichlet Distribution}

If $K \in \nats$ and $\alpha \in (\posreals)^{K}$, then for
$\theta \in \mbox{$K$-simplex}$, 
\[
\distro{Dirichlet}(\theta|\alpha)
= 
\frac{\Gamma \! \left( \sum_{k=1}^K \alpha_k \right)}
     {\prod_{k=1}^K \Gamma(\alpha_k)}
\
\prod_{k=1}^K \theta_k^{\alpha_k - 1}
.
\]

\begin{description}
%
  \fitem{real}{dirichlet\_log}{vector \farg{theta}, vector
    \farg{alpha}}{ The log of the Dirichlet density for simplex
    \farg{theta} given prior counts (plus one) \farg{alpha}}
%
\end{description}


\section{Vector Probabilities}

The vector probability distributions have support on all of
$\reals^K$ for some fixed $K$.

\subsection{Multivariate Normal Distribution}

If $K \in \nats$, $\mu \in \reals^K$, and $\Sigma \in \reals^{K \times
  K}$ is symmetric and positive definite, then for $y \in \reals^K$,
\[
\distro{MultiNormal}(y|\mu,\Sigma)
=
\frac{1}{\left( 2 \pi \right)^{K/2}}
\
\frac{1}{\sqrt{|\Sigma|}}
\
\exp \! \left( \!
- 
\frac{1}{2}
(y - \mu)^{\top} \, \Sigma^{-1} \, (y - \mu)
\right)
\! ,
\]
%
where $|\Sigma|$ is the absolute determinant of $\Sigma$.

\begin{description}
%
\fitem{real}{multi\_normal\_log}{vector \farg{y}, vector \farg{mu},
  matrix \farg{S}}{The log of the multivariate normal density of vector \farg{y}
 given location \farg{mu} and covariance matrix \farg{S}}
%
\end{description}

\subsection{Multivariate Normal Distribution (Cholesky Parameterization)}

If $K \in \nats$, $\mu \in \reals^K$, and $L \in \reals^{K \times K}$ is lower
triangular and such that $LL^{\top}$ is positive definite, then for $y
\in \reals^K$,
\[
\distro{MultiNormalCholesky}(y|\mu,L)
=
\distro{MultiNormal}(y|\mu,LL^{\top}).
\]

\begin{description}
%
\fitem{real}{multi\_normal\_cholesky\_log}{vector \farg{y}, vector
  \farg{mu}, matrix \farg{L}}{The log of the multivariate normal density of vector
  \farg{y} given location \farg{mu} and lower-triangular Cholesky
  factor of the covariance matrix \farg{L}}
%
\end{description}


\subsection{Multivariate Student-$t$ Distribution}

If $K \in \nats$, $\nu \in \posreals$, $\mu \in \reals^K$, and $\Sigma
\in \reals^{K \times K}$ is symmetric and positive definite, then for
$y \in \reals^K$,
\[
\begin{array}{l}
\distro{MultiStudentT}(y|\nu,\mu,\Sigma)
\\[8pt]
\displaystyle
\hspace*{8pt}
=
\frac{1}{\pi^{K/2}}
\
\frac{1}{\nu^{K/2}}
\
\frac{\Gamma\!x\left((\nu + K)/2\right)}
     {\Gamma(\nu/2)}
\
\frac{1}{\sqrt{\left| \Sigma \right|}}
\
\left(
1 + \frac{1}{\nu} \, \left(y - \mu\right)^{\top} \, \Sigma^{-1} \, \left(y - \mu\right)
\right)^{-(\nu + K)/2}
\! .
\end{array}
\]
\vspace*{4pt}

\begin{description}
%
\fitem{real}{multi\_student\_t\_log}{vector \farg{y}, real \farg{nu},
  vector \farg{mu}, matrix \farg{S}}{The log of the multivariate Student-$t$
  density of vector \farg{y} given degrees of freedom \farg{nu},
  location \farg{mu}, and scale matrix \farg{S}}
%
\end{description}



\section{Covariance Matrix Distributions}

The covariance matrix distirbutions have support on symmetric,
positive-definite $K \times K$ matrices.

\subsection{Wishart Distribution}

If $K \in \nats$, $\nu \in (K-1,\infty)$, and $S \in \reals^{K \times K}$ is symmetric
and positive definite, then for symmetric and positive-definite $W \in
\reals^{K \times K}$,
\[
\distro{Wishart}(W|\nu,S)
=
\frac{1}{2^{\nu K / 2}}
\
\frac{1}{\Gamma_K \! \left( \frac{\nu}{2} \right)}
\
\left| S \right|^{-\nu/2}
\
\left| W \right|^{(\nu - K - 1)/2}
\
\exp \! \left(\frac{1}{2} \ \mbox{tr}\left( S^{-1} W \right) \right)
\! ,
\]
%
where $\mbox{tr}()$ is the matrix trace function, and $\Gamma_K()$ is
the multivariate Gamma function,
\[
\Gamma_K(x) = 
\frac{1}{\pi^{K(K-1)/4}}
\
\prod_{k=1}^K \Gamma \left( x + \frac{1 - k}{2} \right)
\!.
\]

\begin{description}
%
\fitem{real}{wishart\_log}{matrix \farg{W}, real \farg{nu}, real
 \farg{S}}{The log of the Wishart density for positive-definite matrix
 \farg{W} given degrees of freedom \farg{nu} and scale matrix
 \farg{S}}
%
\end{description}


\subsection{Inverse Wishart Distribution}

If $K \in \nats$, $\nu \in (K-1,\infty)$, and $S \in \reals^{K \times
  K}$ is symmetric and positive definite, then for symmetric and
positive-definite $W \in \reals^{K \times K}$,
\[
\distro{InvWishart}(W|\nu,S)
= 
\frac{1}{2^{\nu K / 2}}
\
\frac{1}{\Gamma_K \! \left( \frac{\nu}{2} \right)}
\
\left| S \right|^{\nu/2}
\
\left| W \right|^{-(\nu - K - 1)/2}
\
\exp \! \left( 
- \frac{1}{2}
\
\mbox{tr}(SW^{-1})
\right)
\! .
\]

\begin{description}
%
\fitem{real}{inv\_wishart\_log}{matrix \farg{W}, real \farg{nu}, real
 \farg{S}}{The log of the inverse Wishart density for positive-definite matrix
 \farg{W} given degrees of freedom \farg{nu} and scale matrix
 \farg{S}}
%
\end{description}

\subsection{LKJ Covariance Distribution}

\begin{description}
%
\fitem{real}{lkj\_cov\_log}{matrix \farg{W}, vector \farg{mu}, vector
\farg{sigma}, real \farg{eta}}{The log of the LKJ density for covariance matrix
\farg{W} given location vector \farg{mu}, scale vector \farg{sigma},
and shape \farg{eta}; see \citep{LewandowskiKurowickaJoe:2009} for definitions}
%
\end{description}



\section{Correlation Matrix Distributions}

The correlation matrix distributions have support on the (Cholesky
factors of) correlation matrices.  A Cholesky factor $L$ for a $K
\times K$ covariance matrix of dimensions $K$ has rows of unit length
so that the diagonal of $L L^{\top}$ is the unit $K$-vector.

\subsection{LKJ Cholesky Distribution}

\begin{description}
%
  \fitem{real}{lkj\_corr\_cholesky\_log}{matrix \farg{L}, real
    \farg{eta}}{The log of the LKJ density for the lower-triangular
    Cholesky factor \farg{L} of a correlation matrix given shape
    \farg{eta}; see \citep{LewandowskiKurowickaJoe:2009} for definitions}
%
\end{description}

