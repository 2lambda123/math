\part{Algorithms \& Implementations}

\chapter{Hamiltonian Monte Carlo Sampling}\label{hmc.chapter}

\noindent
This part of the manual details the algorithm implementations used by
Stan and how to configure them. This chapter presents the Hamiltonian
Monte Carlo (HMC) algorithm and its adaptive variant the no-U-turn
sampler (NUTS) along with details of their implementation and
configuration in Stan; the next two chapters present Stan's optimizers
and diagnostics.


\section{Hamiltonian Monte Carlo}

Hamiltonian Monte Carlo (HMC) is a Markov chain Monte Carlo (MCMC)
method that uses the derivatives of the density function being sampled
to generate efficient transitions spanning the posterior (see, e.g.,
\citep{Betancourt-Girolami:2013,Neal:2011} for more details). It uses
an approximate Hamiltonian dynamics simulation based on numerical
integration which is then corrected by performing a
Metropolis acceptance step.

This section translates the presentation of HMC by
\cite{Betancourt-Girolami:2013} into the notation of
\cite{GelmanEtAl:2013}.

\subsection{Target Density}

The goal of sampling is to draw from a density $p(\theta)$ for
parameters $\theta$.  This is typically a Bayesian posterior
$p(\theta|y)$ given data $y$, and in particular, a Bayesian posterior
coded as a Stan program.



\subsection{Auxiliary Momentum Variable}

HMC introduces auxiliary momentum variables $\rho$ and draws from a
joint density
%
\[
p(\rho,\theta) = p(\rho|\theta) p(\theta).
\]
%
In most applications of HMC, including Stan, the auxiliary density is
a multivariate normal that does not depend on the parameters $\theta$,
\[
\rho \sim \distro{MultiNormal}(0, \Sigma).
\]
The covariance matrix $\Sigma$ acts as a Euclidean metric to rotate
and scale the target distribution; see \citep{Betancourt-Stein:2011}
for details of the geometry.

In Stan, this matrix may be set to the identity matrix (i.e., unit
diagonal) or estimated from warmup samples and optionally restricted
to a diagonal matrix. The inverse $\Sigma^{-1}$ is known as the mass
matrix, and will be a unit, diagonal, or dense if $\Sigma$ is.

\subsection{The Hamiltonian}

The joint desnity $p(\rho,\theta)$ defines a Hamiltonian
%
\begin{eqnarray*}
H(\rho,\theta) & = & - \log p(\rho,\theta)
\\[3pt]
& = & - \log p(\rho|\theta) - \log p(\theta).
\\[3pt]
& = & T(\rho|\theta) + V(\theta),
\end{eqnarray*}
%
where the term
\[
T(\rho|\theta) = - \log p(\rho | \theta)
\]
is called the ``kinetic energy'' and the term
\[
V(\theta) = - \log p(\theta)
\]
is called the ``potential energy.''  The potential energy is specified
by the Stan program through its definition of a log density. 

\subsection{Generating Transitions}

Starting from the current value of the parameters $\theta$, a
transition to a new state is generated in two stages before being
subjected to a Metropolis accept step.  

First, a value for the momentum is drawn independently of the current
parameter values,
%
\[
\rho \sim \distro{MultiNormal}(0,\Sigma).
\]
%
Thus momentum does not persist across iterations. 

Next, the joint system $(\theta,\rho)$ made up of the current
parameter values $\theta$ and new momentum $\rho$ is evolved
via Hamilton's equations,
%
\[
\begin{array}{rcccl}
\displaystyle
\frac{d\theta}{dt} 
& = &
\displaystyle
+ \frac{\partial H}{\partial \rho} 
& = &
\displaystyle
+ \frac{\partial T}{\partial \rho}
\\[12pt]
\displaystyle
\frac{d\rho}{dt} 
& = & 
\displaystyle
- \frac{\partial H}{\partial \theta } 
& = & 
\displaystyle
- \frac{\partial T}{\partial \theta}
- \frac{\partial V}{\partial \theta}.
\end{array}
\]
%
With the momentum density being independent of the target density,
i.e., $p(\rho|\theta) = p(\rho)$, the first term in the
momentum time derivative, ${\partial T} / {\partial \theta}$ is
zero, yielding the pair time derivatives
%
\begin{eqnarray*}
\frac{d \theta}{d t} & = & +\frac{\partial T}{\partial \rho}
\\[2pt]
\frac{d \rho}{d t} & = & -\frac{\partial V}{\partial \theta}.
\end{eqnarray*}

\subsection{Leapfrog Integrator}

The last section leaves a two-state differential equation to solve.
Stan, like most other HMC implementations, uses the leapfrog
integrator, which is a numerical integration algorithm that's
specifically adapted to provide stable results for Hamiltonian systems
of equations.

Like most numerical integrators, the leapfrog algorithm takes discrete
steps of some small time interval $\epsilon$. The leapfrog algorithm
begins by drawing a fresh momentum term independently of the parameter
values $\theta$ or previous momentum value.
%
\[
\rho \sim \distro{MultiNormal}(0,\Sigma).
\]
It then alternates half-step updates of the momentum and full-step
updates of the position.
%
\vspace*{-6pt}
\begin{eqnarray*}
\rho & \leftarrow 
     & \rho \, - \, \frac{\epsilon}{2} \frac{\partial V}{\partial \theta}
\\[6pt]
\theta & \leftarrow
       & \theta \, + \, \epsilon \, \Sigma \, \rho
\\[6pt]
\rho & \leftarrow 
     & \rho \, - \, \frac{\epsilon}{2} \frac{\partial V}{\partial \theta}.
\end{eqnarray*}
%
By applying $L$ leapfrog steps, a total of $L \, \epsilon$ time is
simulated. The resulting state at the end of the simulation ($L$
repetitions of the above three steps) will be denoted
$(\rho^{*},\theta^{*})$.

The leapgrog integrator's error is on the order of $\epsilon^3$ per
step and $\epsilon^2$ globally, where $\epsilon$ is the time interval
(also known as the step size);  \cite{LeimkuhlerReich:2004} provide a
detailed analysis of numerical integration for Hamiltonian systems,
including a derivation of the error bound for the leapforg
integrator.


\subsection{Metropolis Accept Step}

If the leapfrog integrator were perfect numerically, there would no
need to do any more randomization per transition than generating a
random momentum vector. Instead, what is done in practice to account
for numerical errors during integration is to apply a Metropolis
acceptance step, where the probability of keeping the proposal 
$(\rho^{*},\theta^{*})$ generated by transitioning from $(\rho,\theta)$ is
%
\[
\min\!\left(1, \ \exp\!\left( H(\rho,\theta) - H(\rho^{*},\theta^{*})\right)\right).
\]
%
If the proposal is not accepted, the previous parameter value is
returned for the next draw and used to initialize the next iteration.


\subsection{Algorithm Summary}

The Hamiltonian Monte Carlo algorithm starts at a specified initial
set of parameters $\theta$; in Stan, this value is either
user-specified or generated randomly. Then, for a given number of
iterations, a new momentum vector is sampled and the current value of
the parameter $\theta$ is updated using the leapfrog integrator with
discretization time $\epsilon$ and number of steps $L$ according to
the Hamiltonian dynamics. Then a Metropolis acceptance step is
applied, and a decision is made whether to update to the new state
$(\theta^{*},\rho^{*})$ or keep the existing state.


\section{HMC Algorithm Parameters}

The Hamiltonian Monte Carlo algorithm has three parameters which must
be set,
%
\begin{itemize}
\item discretization time $\epsilon$, 
\item mass matrix $\Sigma^{-1}$, and
\item number of steps taken $L$.
\end{itemize}
%
In practice, sampling efficiency, both in terms of iteration speed and
iterations per effective sample, is highly sensitive to these three
tuning parameters \citep{Neal:2011,Hoffman-Gelman:2014}.

If $\epsilon$ is too large, the leapfrog integrator will be inaccurate
and too many proposals will be rejected. If $\epsilon$ is too small,
too many small steps will be taken by the leapfrog integrator leading
to long simulation times per interval. Thus the goal is to balance the
acceptance rate between these extremes. 

If $L$ is too small, the trajectory traced out in each iteration will
be too short and sampling will devolve to a random walk.  If $L$ is
too large, the algorithm will do too much work on each iteration.

If the mass matrix $\Sigma$ is poorly suited to the covariance of the
posterior, the step size $\epsilon$ will have to be decreased to
maintain arithmetic precision while at the same time, the number of
steps $L$ is increased in order to maintain simulation time to ensure
statistical efficiency.

\subsection{Automatic Parameter Tuning}

Stan is able to automatically optimize $\epsilon$ to match an
acceptance-rate target, able to estimate $\Sigma$ based on warmup
sample iterations, and able to dynamically adapt $L$ on the fly during
sampling (and during warmup) using the no-U-turn sampling (NUTS)
algorithm \citep{Hoffman-Gelman:2014}.

\begin{figure}
\setlength{\unitlength}{0.005in} 
\centering
\begin{picture}(1000, 200)
%
\footnotesize
\put(25, 20) { \framebox(75, 200)[c]{I} }
\put(100, 20) { \framebox(25, 200)[c]{II} }
\put(125, 20) { \framebox(50, 200)[c]{II} }
\put(175, 20) { \framebox(100, 200)[c]{II} }
\put(275, 20) { \framebox(200, 200)[c]{II} }
\put(475, 20) { \framebox(400, 200)[c]{II} }
\put(875, 20) { \framebox(50, 200)[c]{III} }
\put(25, 20) { \vector(1, 0){950} }
\put(800, -10) { \makebox(200, 20)[l]{{\small Iteration}} }
%
\end{picture}
\caption{ \small\it Adaptation during warmup occurs in three stages:
  an initial fast adaptation interval (I), 
  a series of expanding slow adaptation intervals (II), 
  and a final fast adaptation interval (III).  
  For HMC, both the fast and slow intervals are used for adapting the
  step size, while the slow intervals are used for learning the 
  (co)variance necessitated by the metric.  Iteration numbering starts
  at 1 on the left side of the figure and increases to the right.}%
\label{adaptation.figure}
\end{figure}

When adaptation is engaged (it may be turned off by fixing a step size
and mass matrix), the warmup period is split into three stages, as
illustrated in \reffigure{adaptation}, with two \textit{fast}
intervals surrounding a series of growing \textit{slow} intervals.
Here fast and slow refer to parameters that adapt using local and
global information, respectively; the Hamiltonian Monte Carlo
samplers, for example, define the step size as a fast parameter and
the (co)variance as a slow parameter. The size of the the initial and
final fast intervals and the initial size of the slow interval are all
customizable, although user-specified values may be modified slightly
in order to ensure alignment with the warmup period.

The motivation behind this partitioning of the warmup period is to
allow for more robust adaptation.  The stages are as follows.

\begin{enumerate}
\item[I.]
In the initial fast interval the chain is allowed to converge
towards the typical set,%
%
\footnote{The typical set is a concept borrowed from information
  theory and refers to the neighborhood (or neighborhoods in
  multimodal models) of substantial posterior probability mass through
  which the Markov chain will travel in equilibrium.}
%
with only parameters that can learn from local information adapted.
\item[II.]
After this initial stage parameters that require global
information, for example (co)variances, are estimated in a series of
expanding, memoryless windows; often fast parameters will be adapted
here as well. 
\item[III.]
Lastly, the fast parameters are allowed to adapt to the
final update of the slow parameters.
\end{enumerate}

These intervals may be controlled through the following
configuration parameters, all of which must be positive integers:
%
\begin{center}
\begin{tabular}{l|lc}
{\it parameter} & {\it description} & {\it default} 
\\ \hline
{\it initial buffer} & width of initial fast adaptation interval
                     & 75
\\
{\it term buffer} & width of final fast adaptation interval
                  &  50
\\
{\it window}  & initial width of slow adaptation interval
              & 25
\end{tabular}
\end{center}

\subsection{Discretization-Interval Adaptation Parameters}

Stan's HMC algorithms utilize dual averaging \citep{Nesterov:2009} to
optimize the step size.%
%
\footnote{This optimization of step size during
adaptation of the sampler should not be confused with running Stan's
optimization method.}
%
This warmup optimization procedure is extremely flexible and for
completeness, Stan exposes each tuning option for dual averaging,
using the notation of \cite{Hoffman-Gelman:2014}. In practice, the
efficacy of the optimization is sensitive to the value of these
parameters, but we do not recommend changing the defaults without
experience with the dual-averaging algorithm. For more information,
see the discussion of dual averaging in \citep{Hoffman-Gelman:2011,
 Hoffman-Gelman:2014}.

The full set of dual-averaging parameters are
%
\begin{center}
\begin{tabular}{c|lcc}
{\it parameter} & {\it description} & {\it constraint} & {\it default}
\\ \hline
$\delta$ & target Metropolis acceptance rate 
         & $\delta \in [0,1]$ 
         & 0.80
\\
$\gamma$ & adaptation regularization scale
         & $\gamma > 0$
         & 0.05
\\
$\kappa$ & adaptation relaxation exponent
         & $\kappa > 0$
         & 0.75
\\
$t_0$    & adaptation iteration offset
         & $t_0 > 0$
         & 10
\end{tabular}
\end{center}
%
By setting the target acceptance parameter $\delta$ to a value closer
to 1 (its value must be strictly less than 1 and its default value is
0.8), adaptation will be forced to use smaller step sizes. This can
improve sampling efficiency (effective samples per iteration) at the
cost of increased iteration times. Raising the value of $\delta$ will
also allow some models that would otherwise get stuck to overcome
their blockages.

\subsection{Step-Size Jitter}

All implementations of HMC use numerical integrators requiring a step
size (equivalently, discretization time interval). Stan allows the
step size to be adapted or set explicitly. Stan also allows the step
size to be ``jittered'' randomly during sampling to avoid any poor
interactions with a fixed step size and regions of high curvature. The
jitter is a proportion that may be added or subtracted, so the maximum
amount of jitter is 1, which will cause step sizes to be selected in
the range of 0 to twice the adapted step size. The default value is 0,
producing no jitter.

Small step sizes can get HMC samplers unstuck that would otherwise get
stuck with higher step sizes. The downside is that jittering below the
adapted value will increase the number of leapfrog steps required and
thus slow down iterations, whereas jittering above the adapted value
can cause premature rejection due to simulation error in the
Hamiltonian dynamics calculation. See \citep{Neal:2011} for further
discussion of step-size jittering.


\subsection{Euclidean Metric}

All HMC implementations in \Stan utilize quadratic kinetic energy
functions which are specified up to the choice of a symmetric,
positive-definite matrix known as a \textit{mass matrix} or, more
formally, a \textit{metric} \citep{Betancourt-Stein:2011}.

If the metric is constant then the resulting implementation is known
as \textit{Euclidean} HMC.  \Stan allows for three Euclidean HMC
implementations,
%
\begin{itemize}
\item a unit metric (diagonal matrix of ones),
\item a diagonal metric (diagonal matrix with positive diagonal
  entries), and
\item a dense metric (a dense, symmetric positive definite matrix)
\end{itemize}
%
The user may configure the form of the metric.

If the mass matrix is specified to be diagonal, then regularized
variances are estimated based on the iterations in each slow-stage
block (labeled II in \reffigure{adaptation}).  Each of these estimates
is based only on the iterations in that block.  This allows early
estimates to be used to help guide warmup and then be forgotten later 
so that they do not influence the final covariance estimate.

If the mass matrix is specified to be dense, then regularized
covariance estimates will be carried out, regularizing the estimate to 
a diagonal matrix, whihc is itself regularized toward a unit matrix.

Variances or covariances are estimated using Welford accumulators
to avoid a loss of precision over many floating point operations.


\subsection{NUTS and its Configuration}

The no-U-turn sampler (NUTS) automatically selects an appropriate
number of leapfrog steps in each iteration in order to allow the
proposals to traverse the posterior without doing unnecessary work.
The motivation is to maximize the expected squared jump distance (see,
e.g., \citep{RobertsEtAl:1997}) at each step and avoid the random-walk behavior
that arises in random-walk Metropolis or Gibbs samplers when there is
correlation in the posterior. For a precise definition of the NUTS
algorithm and a proof of detailed balance, see
\citep{Hoffman-Gelman:2011,
  Hoffman-Gelman:2014}.

NUTS generates a proposal by starting at an initial position
determined by the parameters drawn in the last iteration. It then
generates an independent unit-normal random momentum vector. It then
evolves the initial system both forwards and backwards in time to form
a balanced binary tree. At each iteration of the NUTS algorithm the
tree depth is increased by one, doubling the number of leapfrog steps
and effectively doubles the computation time. The algorithm terminates
in one of two ways, either
%
\begin{itemize}
\item the NUTS criterion (i.e., a U-turn in Euclidean space on a
  subtree) is satisfied for a new subtree or the completed tree, or
\item the depth of the completed tree hits the maximum depth allowed.
\end{itemize}
%
Rather than using a standard Metropolis step, the final parameter
value is selected with slice sampling along the final evolution step
(i.e., the second half of iterations generated).

Configuring the no-U-turn sample involves putting a cap on the depth
of the trees that it evaluates during each iteration.   This is
controlled through a maximum depth parameter.   The number of leapfrog
steps taken is then bounded by 2 to the power of the maximum depth minus 1. 

Both the tree depth and the actual number of leapfrog steps computed
are reported along with the parameters in the output as
\code{treedepth\_\_} and \code{n\_leapfrog\_\_}, respectively. Because
the final subtree may only be partially constructed, these two will
always satisfy
%
\[
2^{\mathrm{treedepth} - 1} - 1 < N_{\mathrm{leapfrog}} \le 2^{\mathrm{treedepth} } - 1.
\]

Tree depth is an important diagnostic tool for NUTS. For example, a
tree depth of zero occurs when the first leapfrog step is immediately
rejected and the initial state returned, indicating extreme curvature
and poorly-chosen step size (at least relative to the current
position). On the other hand, a tree depth equal to the maximum depth
indicates that NUTS is taking many leapfrog steps and being terminated
prematurely to avoid excessively long execution time. Taking very many
steps may be a sign of poor adaptation, may be due to targeting a very
high acceptance rate, or may simply indicate a difficult posterior
from which to sample. In the latter case, reparameterization may help
with efficiency. But in the rare cases where the model is correctly
specified and a large number of steps is necessary, the maximum depth
should be increased to ensure that that the NUTS tree can grow as
large as necessary. 


\subsection{Sampling without Parameters}

In some situations, such as pure forward data simulation in a directed
graphical model (e.g., where you can work down generatively from known
hyperpriors to simulate parameters and data), there is no need to
declare any parameters in Stan, the model block will be empty, and all
output quantities will be produced in the generated quantities block.
For example, to generate a sequence of $N$ draws from a binomial with
trials $K$ and chance of success $\theta$, the following program suffices.
%
\begin{stancode}
data {
  real<lower=0,upper=1> theta;
  int<lower=0> K;
  int<lower=0> N;
}
model { 
}
generated quantities {
  int<lower=0,upper=K> y[N];
  for (n in 1:N)
    y[n] <- binomial_rng(K, theta);
}
\end{stancode}
%
This program includes an empty model block because every Stan program
must have a model block, even if it's empty.  For this model, the
sampler must be configured to use the fixed-parameters setting because
there are no parameters.  Without parameter sampling there is no need
for adaptation and the number of warmup iterations should be set to
zero.

Most models that are written to be sampled without parameters will not
declare any parameters, instead putting anything parameter-like in the
data block.  Nevertheless, it is possible to include parameters for
fixed-parameters sampling and initialize them in any of the usual ways
(randomly, fixed to zero on the unconstrained scale, or with
user-specified values).  For exmaple, \code{theta} in the example
above could be declared as a parameter and initialized as a parameter.








\section{General Configuration Options}\label{general-config.section}

Stan's interfaces provide a number of configuration options that are
shared among the MCMC algorithms (this chapter), the optimization
algorithms (\refchapter{optimization-algorithms}), and the diagnostics
(\refchapter{diagnostic-algorithms}). 

\subsection{Random Number Generator}

The random-number generator's behavior is fully determined by the
unsigned seed (positive integer) it is started with. If a seed is not
specified, or a seed of 0 or less is specified, the system time is
used to generate a seed. The seed is recorded and included with Stan's
output regardless of whether it was specified or generated randomly
from the system time.

Stan also allows a chain identifier to be specified, which is useful
when running multiple Markov chains for sampling. The chain identifier
is used to advance the random number generator a very large number of
random variates so that two chains with different identifiers draw
from non-overlapping subsequences of the random-number sequence
determined by the seed.  When running multiple chains from a single
command, Stan's interfaces will manage the chain identifiers.

\subsubsection{Replication}

Together, the seed and chain identifier determine the behavior of the
underlying random number generator. See \refchapter{reproducibility} for a
list of requirements for replication.

\subsection{Initialization}

The initial parameter values for Stan's algorithms (MCMC,
optimization, or diagnostic) may be either specified by the user or
generated randomly. If user-specified values are provided, all
parameters must be given initial values or Stan will abort with an
error message.

\subsubsection{User-Defined Initialization}

If the user specifies initial values, they must satisfy the
constraints declared in the model (i.e., they are on the constrained
scale).

\subsubsection{System Constant Zero Initialization}

It is also possible to provide an initialization of 0, which causes
all variables to be initialized with zero values on the unconstrained
scale. The transforms are arranged in such a way that zero
initialization provides reasonable variable initializations for most
parameters, such as 0 for unconstrained parameters, 1 for parameters
constrained to be positive, 0.5 for variables to constrained to lie
between 0 and 1, a symmetric (uniform) vector for simplexes, unit
matrices for both correlation and covariance matrices, and so on. See
\refchapter{variable-transforms} for full details of the
transformations.


\subsubsection{System Random Initialization}

Random initialization by default initializes the parameter values with
values drawn at random from a $\distro{Uniform}(-2,2)$ distribution.
Alternatively, a value other than 2 may be specified for the absolute
bounds. These values are on the unconstrained scale, and are inverse
transformed back to satisfy the constraints declared for parameters.
See \refchapter{variable-transforms} for a complete description of the
transforms used.  

Because zero is chosen to be a reasonable default initial value for
most parameters, the interval around zero provides a fairly diffuse
starting point. For instance, unconstrained variables are initialized
randomly in $(-2,2)$, variables constrained to be positive are
initialized rougly in $(0.14,7.4)$, variables constrained to fall
between 0 and 1 are initialized with values roughly in $(0.12,0.88)$.


\chapter{Optimization  Algorithms}%
\label{optimization-algorithms.chapter}

\noindent
Stan provides optimization algorithms which find modes of the density
specified by a Stan program. Such modes may be used as parameter
estimates or as the basis of approximations to a Bayesian posterior;
see \refchapter{mle} for background on point estimation.

Stan provides three different optimizers, a Newton optimizer, and two
related quasi-Newton algorithms, BFGS and L-BFGS; see
\citep{NocedalWright:2006} for thorough description and analysis of
all of these algorithms. The L-BFGS algorithm is the default
optimizer. Newton's method is the least efficient of the three, but
has the advantage of setting its own stepsize.

\section{General Configuration}

All of the optimizers are iterative and allow the maximum number of
iterations to be specified;  the default maximum number of iterations
is 2000.  

All of the optimizers are able to stream intermediate output reporting
on their progress.  Whether or not to save the intermediate iterations
and stream progress is configurable.

\section{BFGS and L-BFGS Configuration}

\subsection{Convergence Monitoring}

Convergence monitoring in (L-)BFGS is controlled by a number of
tolerance values, any one of which being satisified causes the
algorithm to terminate with a solution. Any of the convergence tests
can be disabled by setting its corresponding tolerance parameter to
zero.  The tests for convergence are as follows.

\subsubsection{Parameter Convergence}

The parameters $\theta_i$ in iteration $i$ are considered to have
converged with respect to tolerance \code{tol\_param} if
%
\[
|| \theta_{i} - \theta_{i-1} || < \mbox{\code{tol\_param}}.
\]


\subsubsection{Density Convergence}
%
The (unnormalized) log density 
$\log p(\theta_{i}|y)$ for the parameters $\theta_i$ in iteration $i$
given data $y$ is considered to have converged with
respect to tolerance \code{tol\_obj} if
%
\[
\left| \log p(\theta_{i}|y) - \log p(\theta_{i-1}|y) \right| <
\mbox{\code{tol\_obj}}.
\]
%
The log density is considered to have converged to within
relative tolerance \code{tol\_rel\_obj} if
%
\[
\frac{\left| \log p(\theta_{i}|y) - \log p(\theta_{i-1}|y) \right|}{\
  \max\left(\left| \log p(\theta_{i}|y)\right|,\left| \log
      p(\theta_{i-1}|y)\right|,1.0\right)}
 < \mbox{\code{tol\_rel\_obj}} * \epsilon.
\]
%


\subsubsection{Gradient Convergence}

The gradient is considered to have converged to 0 relative to a
specified tolerance \code{tol\_grad} if 
%
\[
|| g_{i} || < \mbox{\code{tol\_grad}},
\]
where $\nabla_{\theta}$ is the gradient operator with respect to
$\theta$ and $g_{i} = \nabla_{\theta} \log p(\theta_{i}|y)$ is the gradient at
iteration $i$.  

The gradient is considered to have converged to 0 relative to a
specified relative tolerance
\code{tol\_rel\_grad} if
%
\[
\frac{g_{i}^T \hat{H}_{i}^{-1} g_{i} }{ \max\left(\left|\log p(\theta_{i}|y)\right|,1.0\right) } < \mbox{\code{tol\_rel\_grad}} * \epsilon,
\]
%
where $\hat{H}_{i}$ is the estimate of the Hessian at iteration $i$,
$|u|$ is the absolute value (L1 norm) of $u$, $||u||$ is the vector
length (L2 norm) of $u$, and $\epsilon \approx 2e-16$ is machine
precision.


\subsection{Initial Step Size}

The initial step size parameter $\alpha$ for BFGS-style optimizers may
be specified. If the first iteration takes a long time (and requires a
lot of function evaluations) initialize $\alpha$ to be the roughly
equal to the $\alpha$ used in that first iteration. The default value
is intentionally small, 0.001, which is reasonable for many problems
but might be too large or too small depending on the objective
function and initialization. Being too big or too small just means
that the first iteration will take longer (i.e., require more gradient
evaluations) before the line search finds a good step length. It's not
a critical parameter, but for optimizing the same model multiple times
(as you tweak things or with different data), being able to tune
$\alpha$ can save some real time.

\subsection{L-BFGS History Size}

L-BFGS has a command-line argument which controls the size of the
history it uses to approximate the Hessian. The value should be less than
the dimensionality of the parameter space and, in general, relatively
small values (5--10) are sufficient; the default value is 5.

If L-BFGS performs poorly but BFGS performs well, consider increasing
the history size. Increasing history size will increase the
memory usage, although this is unlikely to be an issue for typical
Stan models.


\section{General Configuration Options}

The general configuration options for optimization are the same as
those for MCMC;  see \refsection{general-config} for details.


\chapter{Diagnostic Mode}\label{diagnostic-algorithms.chapter}

\noindent
Stan's diagnostic mode runs a Stan program with data, initializing
parameters either randomly or with user-specified initial values, and
then evaluates the log probability and its gradients. The gradients
computed by the Stan program are compared to values calculated by
finite differences.

Diagnostic mode may be configured with two parameters.
%
\begin{center}
\begin{tabular}{c|lcc}
{\it parameter} & {\it description} & {\it constraints} & {\it
  default}
\\ \hline
{\it $\epsilon$} & finite difference size & $\epsilon > 0$ & 1e--6
\\
{\it error} & error threshold for matching & $\mbox{error} > 0$ & 1e--6
\end{tabular}
\end{center}
%
If the difference between the Stan program's gradient value and that
calculated by finite difference is higher than the specified
threshold, the argument will be flagged.

\section{Output}

Diagnostic mode prints the log posterior density (up to a proportion)
calculated by the Stan program for the specified initial values. For
each parameter, it prints the gradient at the initial parameter values
calculated by Stan's program and by finite differences over Stan's
program for the log probability.  

\subsection{Unconstrained Scale}

The output is for the variable values and their gradients are on the
unconstrained scale, which means each variable is a vector of size
corresponding to the number of unconstrained variables required to
define it. For example, an $N \times N$ correlation matrix, requires
$\binom{N}{2}$ unconstrained parameters. The
transformations from constrained to unconstrained parameters are based
on the constraints in the parameter declarations and described in
detail in \refchapter{variable-transforms}.

\subsection{Includes Jacobian}

The log density includes the Jacobian adjustment implied by the
constraints declared on variables; see
\refchapter{variable-transforms} for full details. The Jacobian
adjustment will be turned off if optimization is used in practice, but
there is as of yet no way to turn it off in diagnostic mode.

\section{Configuration Options}

The general configuration options for diagnostics are the same as
those for MCMC; see \refsection{general-config} for details. Initial
values may be specified, or they may be drawn at random. Setting the
random number generator will only have an effect if a random
initialization is specified.

\section{Speed Warning and Data Trimming}

Due to the application of finite differences, the computation time
grows linearly with the number of parameters. This can be require a
very long time, especially in models with latent parameters that grow
with the data size. It can be helpful to diagnose a model with smaller
data sizes in such cases.


